\cleardoublepage
\chapter{The UNIX {\cmd cc} Compiler}
\label{cc}
\index{cc@{\cmd cc}}     
     The {\cmd cc} compiler is the main compiler of the UNIX system,
and comprises  several parts.  These parts  are often  transparent to
the user,  since  the  {\cmd cc} command calls them as necessary.
Nontheless, it is useful to know a little bit about the compiler and
its constituent phases.

     First, there  is the  {\kc preprocessor\/}, {\cmd cpp}. This is a 
simple macroprocessor, which has  a small  set of commands allowing
macros to be defined, sections of program to  be conditionally 
compiled, text  from other files to be included, and so  on. {\cmd
cpp} also strips comments from files. The output from {\cmd cpp} is
passed as a text stream to the first pass of the compiler.

     The compiler  conceptually operates  in  two  passes,  although 
on  most systems these  actually occur  together. The  first pass
takes the output from {\cmd cpp} and  converts it  into an 
intermediate file. This file typically contains assembly language 
statements for  the higher  level  program  code  (such  as loops),
and  a symbolic  representation of  the lower level expressions in
the program. The  second pass copies the assembly language statements
to a further intermediate file, and generates assembly language for
the expressions. There is also an optional optimisation phase, {\cmd
c2}, after the second pass.

     The compiler  output is   given  to the  {\kc assembler\/},
{\cmd as}.  This is  the resident UNIX  assembler and  varies from 
system to  system.  The  assembler converts the  assembly language 
file into machine code, and saves the results in a file using the
 UNIX common  object file  format (or  COFF). This  format
includes various  headers, symbol  table and relocation information,
and three sections -- one for  the machine  code (called {\bf
.text}), one for the initialised data in the program (called {\bf
.data}), and one for  uninitialised data (called {\bf .bss}).

     The assembler  output file  is then passed to the {\kc linker\/}, 
{\cmd ld}. This is the standard UNIX linker. It adds any necessary
library functions to the file, and resolves references across files
if the program is split over several files.

\section{File Names}

The  name of  the C source  file should always end with  a {\fn .c}
suffix. Two of the compilation phases  create intermediate files with
the same name as the source file, but with different suffixes, namely:
 \begin{display}
\begin{tabular}{@{}ll@{}}
          {\fn .s} &  the output from the second pass \\
          {\fn .o} &  the output from the assembler 
\end{tabular}
\end{display}
\noindent
 These intermediate files are usually destroyed, unless you specify
otherwise (see later).  The final executable file is always put in a
file called {\fn a.out}, unless you specify otherwise. 

\section{Invoking the Compiler}
\index{cc@{\cmd cc}!options}
To invoke the compiler, use:
\begin{display}\ms
{\cmd cc} $[$options\/$]$ files
\end{display}
\noindent
     {\ms files\/}  can be  one or more {\fn .c}, {\fn .s} and/or
{\fn .o}  files -- {\cmd cc} will invoke only the appropriate
compilation phase(s) for each file.  Zero or  more options  may be 
specified.  The  main options are:
\begin{display}
\begin{tabular}{@{}ll@{}}
 {\cmd -c}  & Don't run the {\cmd ld} linker (that is, make only a 
 					{\fn .o} file). \\
 {\cmd -g}  & Generate debugging information\index{debugging!compiling for debuggers}. \\
 {\cmd -p}  & Generate code for execution profiles. \\
 {\cmd -O}  & Optimise the code produced by the compiler. \\
 {\cmd -S}  & Don't delete the {\fn .s} files. \\
 {\cmd -E}  & Run {\cmd cpp} only, sending output to the standard
					output. \\
 {\cmd -C}  & Stop {\cmd cpp} from removing comments. \\
 {\cmd -o} {\ms name\/} & Give the final (executable) file the name 
 							{\ms name}. 
\end{tabular}
\end{display}
\noindent
     Once a program has been compiled, assembled and linked without
errors, it can be executed by simply typing its name.


\subsection{Preprocessor Options}
\index{cpp@{\cmd cpp}!options}
     The preprocessor  has a  few options  which can  be sent  to  it 
either directly (if it is invoked as {\cmd cpp}), or via {\cmd cc}.
The options are:
\begin{display}
\begin{tabular}{@{}lp{.6\textwidth}@{}}
  {\cmd -I}{\ms dir\/}  &  Search the directory {\ms dir\/} for 
									{\cd \#include} files first. \\
  {\cmd -C}             &  Don't remove comments. \\
  {\cmd -U}{\ms id\/}   &  Undefine {\ms id\/} (equivalent to 
                          {\cd \#undef} {\ms id\/}). \\
  {\cmd -D}{\ms id\/}[={\ms string\/}] & 
       Define {\ms id\/}. If the ={\ms string\/} part is omitted, this is 
       equivalent to {\cd \#define} {\ms id\/}, otherwise it is equivalent 
       to  {\cd \#define} {\ms id string\/}. If the string {\ms string\/} 
       contains spaces, it must be quoted (the quotes will be removed by 
       {\cmd cpp}).
\end{tabular}
\end{display}


\section{Controlling Compilation with simple Shell Scripts}

     If you  often repeat  the same compilation sequence after editing
a file, you can save the sequence in a file and then execute it as a
{\kc shell script\/}. For example, suppose  you want  to compile  and
execute  several programs, but you want to  optimise all compilations
and generate execution profiles. Thus, you are typically repeating
this sequence:
\begin{display}\cmd
  cc -p -O -o {\ms output\_file\/} {\ms source\_file\/} \\
  {\ms output\_file\/}
\end{display}
\noindent
     You can  create a  shell script to generalise this sequence.
Firstly, use an editor to create a file (called say {\fn mycc}), 
containing the following two lines:
\begin{display}\cmd
 cc -p -O -o \$2 \$1 \\
 \$2
\end{display}
\noindent
Then, after having saved the file, issue the UNIX command
\begin{display}\cmd
 chmod u+x mycc
\end{display}
\noindent
 which changes the  file's access mode  setting.
   It adds  ({\cmd +}) execute ({\cmd x}) permission for the
user ({\cmd u}) of the file {\cmd mycc}.

     The {\cmd \$1}  and {\cmd \$2}  refer to the positional
arguments which you supply to {\cmd mycc}. If, for example, you use:
\begin{display}\cmd
mycc prog1.c prog1
\end{display}
\noindent
     {\cmd \$1} will  be {\cmd prog1.c}  and {\cmd \$2}  will be  {\cmd
prog1}. This  will  compile  {\cmd prog1.c} appropriately, leaving the
output in the file {\cmd prog1}, and then execute {\cmd prog1}.

     Shell scripts  are considerably  more powerful  than this  simple
example illustrates, and  can often  be an  effective aid  in
simplifying such routine tasks.  However, a much more sophisticated
way of maintaining programs is with the {\cmd make} utility, which is
described later.

