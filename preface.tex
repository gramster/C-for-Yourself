\chapter*{Preface}

C is a small language, and yet it is not simple to master. This is
because the source of the power of C, namely
the freedom it allows the programmer to do what (s)he wants,
is also the source of its weakness: it may fail catastrophically.

Programming in a language like Pascal or Modula-2 is like driving
around a city in a sophisticated car, with excellent brakes and other
safety features.  Programming in C is like flying a microlight. Your
controls are few and simple, yet you can soar above the restrictions
of roads and traffic signs, and get to your destination much more
directly.  However, when you make an error, the results are
catastrophic. Obviously, the better you can control the microlight,
the greater its advantages.  This control comes not just from a sound
training, but also from experience.  It is the aim of this book to
provide the former; it is up to the reader to obtain the latter.

The book is written for programmers who are familiar with some other
structured language (such as Pascal or more recent dialects of
Basic), and who are moving to the UNIX system and need to learn C.
It covers the entire C language, as well as the most common
library functions which give C its power.  In addition 
we have included appendices on the main UNIX editor {\cmd vi},
as well as many of the UNIX programmer's support tools for debugging
and program maintenance. These tools greatly simplify the task of
developing programs in C.

Recently, in an attempt at developing a standard for the language,
the C language has been undergoing changes.  Many of these changes
have not yet filtered down into the commercial UNIX world yet, so we
have concentrated on the most common version of C, so-called
Kernighan and Ritchie C. However, we have mentioned throughout the
text areas where changes are occurring, and  have included an
appendix listing the main enhancements that have been made. These
enhancements are very welcome, and if your compiler supports them,
you would do well to use them, in particular function
prototypes.

We have tried to make the book as comprehensive as possible. Thus,
the book is not only an introduction for programmers learning C, but
should remain a useful reference, even for experienced C programmers.
Alas, similar to programs, we know that there are errors and omissions
in this book.  We would appreciate being notified of any that you
find, regardless of how trivial you may think they are.

In conclusion, may you come to ``C for Yourself'' what can, what should, and
what should not be done with C.

\bigskip\noindent
{\em Graham Wheeler\\
Ri\"{e}l Smit}\\
Cape Town, February 1990.

\subsection*{Preface to 2nd Printing}

Apart from correcting all the errors we were aware of, we added
information on initialisation of arrays, 
input/output of structures, placing structures at absolute addresses,
bitfields, and startup code.  Furthermore, two
articles dealing with portable code and C style have been included as
appendices.  These articles have been included as they
were received electronically
(with only minor changes to
allow them to be formatted with the rest of the book).
Their typographical style is therefore not
necessarily the same as that of the rest of the book.  Neither have we
implemented all the  C style recommendations. They are only 
recommendations after all. {\cd :-)}

\bigskip\noindent
{\em GW \& RS}\\
Cape Town, February 1991.
