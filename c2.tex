
\section{Pointers and Addresses}
\index{pointers}
\index{addresses}

     Memory in the computer is accessed much like post boxes in a
street: each unit of  memory (usually a byte made up of eight bits)
has a unique numeric  {\kc address} which  can be  used to  access
it. These addresses run sequentially, so,  for example, if the string
{\cd "cat"} was in memory, the address of the  byte containing  the
{\cd 'a'} would be one more or less than the address of the byte
containing the {\cd 'c'}.

     Most high  level languages hide this level of memory access, an
exception being BASIC's  {\bf peek} and  {\bf poke}. C,  however,
provides an elegant mechanism for this task.

     We can  find the address of any variable or function by using the
{\ms address-of\/}~({\cd \&})  operator, and  we can  access the
contents of any address by using the {\ms contents-of\/}~({\cd *})
operator\index{operators!indirection}. These two operations are complementary, and in fact the
compiler  removes the  combination {\cd *\&}  wherever it  finds it.
The reverse combination, {\cd \&*},  is meaningless,  as one  will
then be attempting to find the address of  a value,  and not  of a
variable; nonetheless, the compiler removes this combination too.

     These operations open up a whole new world of types in C. For any
type {\ms t\/}, we now  have the  type {\ms address  of object  of
type  t\/}. Such  types are  called {\em pointer types}, as they
effectively point to objects of other types.

     Declaring a  pointer to  a type  {\ms t\/} in  C is  easy: we
simply declare the variable as  if it  were of  type {\ms t\/},  but
prefix its name with a {\cd *} to indicate that it is a pointer. The
following example, though simple, will illustrate basic pointer
usage:
\begin{code}
 main() \\
 \{ \+ \\
    int \=n, \tab \= /* an integer */ \\
	 \>    *p;\> /* a pointer to an integer */ \\
    n = 2; \\
    p = \&n; \>\> /* p now contains the address of n */ \\
    printf("\%d\verb+\+n",*p); \\
    {*}p = 3; \\
    printf("\%d\verb+\+n",n); \- \\
 \}
\end{code}
\noindent
     You should be able to see that the results this will produce are:
\begin{code}
2 \\
3
\end{code}
\noindent
     The {\ms contents-of\/}  operator has  a higher  precedence  than
multiplication; thus, for example, the expression:
\begin{code}
a**b**c
\end{code}
\noindent
 is grouped as:
\begin{code}
a*(*b)*(*c)
\end{code}
\noindent
     We will now demonstrate one of the primary applications of
pointers in C.



\section{Passing Arguments by Reference}

\index{arguments!passed by reference}
     The main  advantage of pointers is that they allow us to alter a
variable without knowing  what the  variable is,  just by  knowing a
value, namely, the address of  the variable.  This means  we can 
effectively pass  variables  by reference. Say  we want  to pass a
variable of type {\ms t\/} by reference. Instead of passing the 
variable, we  pass its  address.  For  the  corresponding  formal
argument, we  declare a  {\ms pointer to  t\/}. Within the called
function, we can now happily access  the  variable  (as  the  object 
pointed  to  by  our  pointer argument). Let's see an example:
suppose we want a function to return both the square and  the cube of
a  number, in  two different variables. The following program
illustrates this:
\begin{code}
 /* Square/Cube Example 4 */ \addVspace

 void calculate(x,xs,xc) \\
 \> int x, *xs, *xc; \\
 \{ \+\\
      {*}xs = x*x; \\
      {*}xc = (*xs)*x; 
		\-\\
 \} \addVspace

 main() \\
 \{ \+\\
      int n, n\_sq, n\_cb; \addVspace

      scanf("\%d",\&n); \\
      calculate(n, \&n\_sq, \&n\_cb); \\
      printf("\%d\verb+\+t\%d\verb+\+t\%d\verb+\+n", n, n\_sq, n\_cb);
		\-\\
 \}

\end{code}
\noindent
 This explains  why in  calls to {\cd scanf} we send the
address of the variable to read and not the value.

It should be clear that although we are effectively passing arguments by
reference, we are in fact simply passing the {\em addresses} of these arguments
by {\em value}. In other words, the ``arguments'' that are being passed by
reference are not actually the arguments being passed at all. Remember, in C all
arguments are passed by value; there are no exceptions to this rule. Thus we are simply achieving the effect of
passing arguments by reference; we cannot really do so at all.

     Pointers are  an  intrinsic  part  of  C,  as  will  become 
increasingly apparent. A sound understanding of them is a key to
mastering the language.

\section{More about Basic Types}

     So far  we have  only seen the basic C types such as {\cd int},
{\cd float} and {\cd char}.  We will now introduce some further
extensions to these basic types.

     {\cd ints}\index{integers}  are  signed  integers,  
usually  represented  with
{\em 2's-complement representation}. The  actual range  of integers 
allowed varies from machine to machine: on some, 16 bits are used to
represent an integer, whereas others use 32 bits.  Most C compilers
use 16-bit integers.  However, C allows  one to 
specify the size of an integer variable explicitly with the two type
specifiers\index{integers!short@{\cd short}}\index{integers!long@{\cd long}} 
\begin{display}
\begin{tabular}{@{}ll@{}}
  {\cd  short int} & 16-bit signed integer, and \\
  {\cd long int}   &    32-bit signed integer
\end{tabular}
\end{display}
\noindent
     In each case, the keyword {\cd int} is optional.

     Similarly, a {\cd float}\index{floating point numbers}  
is usually represented by 32 bits.
Should we wish to use double precision (64-bit) arithmetic, the type
{\cd double} is provided.

     Often we only need unsigned integers (for example, in  an
execution profile). While an n-bit {\em signed\/} integer 
\index{integers!signed@{\cd signed}}allows a
numeric range of $(-2^{n-1}, 2^{n-1}-1)$,  an n-bit {\em unsigned\/}
\index{integers!unsigned@{\cd unsigned}}integer allows $(0, 2^{n}-1)$.  
To  specify  an unsigned type in C,
we prefix the type with the keyword {\cd unsigned}, as in:
\begin{code}
unsigned short x;\\
static unsigned float root;
\end{code}
\noindent
     Care should be taken when mixing signed and unsigned types. The
advantage of 2's  complement representation  is that  all arithmetic
can be performed as though it  is unsigned.  The problem  that  this 
introduces  though,  is  that overflows may  occur that will not be
reported. If you are expecting output of negative numbers  and you 
get positive  ones instead, or vice versa, you have probably mixed
signed and unsigned types in an overflow situation.



\section{Characters and String Constants}
\index{constants!string}
\index{constants!character}

     We have  seen that  C provides  a character  type {\cd char}, 
and that  string constants in  {\cd printf} and  {\cd scanf} calls
are written  in double  quotes. C has a novel way  of handling 
strings, which  is confusing at first but becomes very sensible once
you properly understand it.

     To fully  understand strings,  we must  discuss arrays,  as a 
string is, essentially, an  array of  characters. However, as
mentioned earlier, pointers figure greatly in C, and we shall soon
see that arrays and pointers are almost identical in  C. Thus  we
can  introduce the  basics of  C strings  using  the pointer concept.

     When C  detects a  single quote,  it knows  that it is about to
process a character constant. The character following the quote is
the constant, and its ASCII (or  possibly other)  representation is
used. In effect, then, character constants in C are actually just
very short (8-bit) integers. In fact, when we read or  print a
character, it is only the {\cd \%c} format specifier that tells C it
must print  or read a character and not an ASCII value. A consequence
of this is that we can, for example, quite happily perform arithmetic
with C character variables. In  fact, this  is commonly  done --  for
example,  if we  wished to convert an  integer variable  {\cd num}
with  a value  from 0  to 9 into a character variable {\cd ch}
representing the corresponding ASCII character, we could use:
\begin{code}
 ch = num + '0';
\end{code}
\noindent
     {\bf Warning:} If  your system  does not use ASCII
representation, you may not be able  to do this, as it is based on
the assumption that character encodings of digits and letters run
sequentially.

     The following short program illustrates this idea:
\begin{code}
 main()  \\
 \{ \+  \\
   char c; \addVspace

   printf("Please enter a lower case letter\verb+\+n"); \\
   scanf("\%c",\&c); \\
   printf("Char \%c has representation \%d\verb+\+n", c, c); \\
   printf("Place in alphabet is \%d\verb+\+n", c-'a'+1); \\
   printf("In upper case it is \%c\verb+\+n", c-'a'+'A'); 
	\-\\
\}
\end{code}
\noindent
     Say we  enter {\cd 'f'}.  This has ASCII representation 102\@.
The first {\cd printf} will thus print {\cd Char f has representation
102}. Now, {\cd 'a'} has representation 97; thus the expression {\cd
c-'a'+1}  has value 6, which is the position of {\cd f} in  the
alphabet.  Adding {\cd 'A'} (ASCII 65) to {\cd c-'a'} (5) gives the
result 70, which is the ASCII code for {\cd F}.

     To illustrate a simple example of {\kc strings}, consider the call:
\begin{code}
 printf("Hello\verb+\+n");
\end{code}
\noindent
     When C encounters a double quote, it knows it has found a {\kc
string constant\/} (a string  being a  sequence of  zero or more
characters). During compilation, all the string constants in a
program are put in the {\bf .data} \index{data@{\bf .data}}section of the file (recall, in  UNIX,
the  {\bf .data} section  holds the  initialised data  from  your
program). Within  the program,  the string is replaced by its address.
Thus, {\em a string constant  in C  is converted to a pointer to
char}. In fact, all strings in C  are treated  as pointers  to char. 
Furthermore, the first argument of a {\cd printf} or {\cd scanf} call
must be a pointer to char. Thus we could write:
\begin{code}
 char *i,*o,ch; \addVspace

 i = "Please enter a character\verb+\+n"; \\
 o = "\%c"; \\
 printf(i); \\
 scanf(o,\&ch);
\end{code}
\noindent
     C strings  are sequences  of characters  which are terminated by
the ASCII character NUL (that is, a byte with value zero - we will
refer to this character as the null character)\index{null character}.  This
null is important, as it is the only way  C identifies  the end of
the string. When using string constants, as we have  been until  now,
the  null  byte  is inserted  automatically  by  the compiler.

     We can  extend a  string constant  over more than one line in
exactly the same way  as we did with preprocessor macros; that is, we
end all but the last line with a backslash (\verb+\+).

     Another point to note is that C pointers are usually similar to
integers. Thus, if we  were to  print {\cd i}  or {\cd o}  above as
numbers, we would get the addresses at which these  two string
constants are stored. We can print the strings pointed to by using
{\cd \%s} in {\cd printf} statements, for example:
\begin{code}
printf("Address is \%d, String at address is \%s\verb+\+n",i,i);
\end{code}
\noindent
     {\cd \%s} is  also used for reading  strings with  {\cd scanf}.
Note  that  the argument corresponding  to a  {\cd \%s} must  in
either  case be  a pointer  to char, representing the start address
of a null terminated string of characters.



\section{Numeric Constants and Character Escapes}
\index{constants!numeric}
\index{constants!character}
\index{character escapes}

     Numeric constants in C are not restricted to decimal numbers: we
can also use {\kc hexadecimal\/}  (base 16)  and {\kc octal\/} (base
8) numbers.  Octal  numbers\index{octal numbers}\index{integers!octal}  are indicated by  a zero ({\cd 0}) 
prefix and hexadecimal numbers 
\index{hexadecimal numbers}\index{integers!hexadecimal}by a 
{\cd 0x} prefix. Thus, the
following represent the same numbers:
\begin{display}
{\cd 16} (decimal) \tab  {\cd 020} (octal) \tab {\cd 0x10} (hexadecimal)
\end{display}
\noindent
     An integer constant can be followed immediately by an {\cd l} or
{\cd L} to specify explicitly that  it is  of type  
{\cd long}\index{integers!long@{\cd long}}\index{constants!long@{\cd long}}. 
All floating point  constants are of type 
{\cd double}\index{constants!double@{\cd double}}; thus this does
not apply to them.

     Octal numbers  are used  in C  to represent 
characters as  well; this is particularly useful  for control codes.
Within quotes, a backslash followed by a number  is taken  to be  a
{\kc character  escape}. The  number is  interpreted in octal, and 
can be  at most  377 (octal). For example, many terminals emit a beep
when ASCII 7 (control-G) is printed. We could implement a beep macro
as:
\begin{code}
\#define  BEEP   printf("\verb+\+7");
\end{code}
\noindent
     Floating point  numbers  can  be  expressed  using either  decimal 
or scientific notation. Some examples:
\begin{display}\cd
\begin{tabular}{@{}lll@{}}
 3.14159   &     3.14159e0  &    0.31415927e+1 \\
 31.415927e-1 &  0.31415927e1   &
\end{tabular}
\end{display}


\section{More about {\cd printf}}
\index{printf@{\cd printf}}

  We have  seen that  the function {\cd printf} takes a string
argument which may contain {\em conversion  specifiers} 
\index{conversion specifiers!printf@{\cd printf}}such as {\cd
\%d} to indicate the place and format for printing other  arguments.
Everything  in the  control string  which is  not a valid conversion 
specifier will  be printed.  The conversion  specifiers  are
considerably more powerful than we have seen so far.

     The general form of a {\cd printf} conversion specifier is:
\begin{code}
 \%$[$-$][$+$][$\ $][${\ms fieldwidth\/}$][$.$][${\ms decimal\_places}$]${\ms
conversion\_chars}
\end{code}
\noindent
     The square  brackets $[]$ indicate optional elements of the
specifier, and are not  part of  the specifier  itself. The  optional
{\cd -} indicates that  the result should  be printed left-justified
(the default is right-justified). The optional {\cd +}  indicates
that a sign should always be printed, even if the number is
positive.  Alternatively, a  space indicates that a space should be
printed before positive  numbers instead  of a  {\cd +} sign. If
neither a {\cd +} nor a space is present, no character is
printed before positive numbers.

 {\ms fieldwidth\/}  specifies the field width, i.e.\ the {\em
minimum\/} number of places to be used for printing. If the value to
be printed occupies less places than {\ms fieldwidth\/} specifies,
spaces are printed to  fill it up, otherwise the entire value is
printed with no spaces.  If {\ms fieldwidth\/} begins with a zero,
the print field is filled up with zeroes  instead of spaces. A {\cd
*} can be used instead of a number, in which case the field width
must be passed as an argument in {\cd printf}. For example:
\begin{code} 
printf("\%-*s",fw,str); 
\end{code} 
\noindent
  will print the string pointed to by {\cd str} left justified in a
field of width {\cd fw}.

     The {\cd .}\ is used simply to separate the field width and decimal
place numbers, and  is only  essential if we use a decimal place
number. {\ms decimal\_places\/} indicates, for a {\cd float} or {\cd
double}, the number of decimal places to be printed,  or, for  a
string,  the number  of characters of the string to be printed. Thus:
\begin{code}
printf("\%.3s","January");
\end{code}
\noindent
   will print {\cd Jan}.

{\ms conversion\_chars\/} can be any of the following:
\begin{display}
\begin{tabular}{@{}ll@{}}
     {\cd c}     &    character \\
     {\cd d}     &    signed decimal integer \\
     {\cd ld} or {\cd D} &  signed long decimal integer \\
     {\cd u}       &  unsigned decimal integer \\
     {\cd lu} or {\cd U} &  unsigned long decimal integer \\
     {\cd o}       &  unsigned octal integer \\
     {\cd lo} or {\cd O} &  unsigned long octal integer \\
     {\cd x}       &  unsigned hexadecimal integer \\
     {\cd lx} or {\cd X} &  unsigned long hexadecimal integer \\
     {\cd f}       &  decimal float or double \\
     {\cd e}       &  scientific (exponent) float or double \\
     {\cd g}       &  shortest of {\cd e} or {\cd f} \\
     {\cd s}       &  string
\end{tabular}
\end{display}
\noindent
 {\cd printf} returns the number of characters printed (including the
terminating null), or a negative value if an output error was
encountered.

\section{More about {\cd scanf}}
\index{scanf@{\cd scanf}}

     {\cd scanf} has  a format  very  similar to {\cd printf}.  The
control  string is now interpreted as  specifying the  format of the 
{\em input}.  Any  spaces,  tabs  or newlines in  this string  are
ignored:  C expects  such characters  (known  as {\kc whitespace
characters})  to separate  the input  items anyway,  so they are not
needed. Any  characters  in  the  control  string  other  than 
whitespace  or conversion specifiers  are expected  to be  found as 
well. As an example, the call:
\begin{code}
scanf("\%d,\%d", \&num1, \&num2);
\end{code}
\noindent
     when given the input:
\begin{code}
23,7
\end{code}
\noindent
     will return with {\cd num1} set to 23 and {\cd num2} set to 7.
However, the input:
\begin{code}
23 7
\end{code}
\noindent
     will result  in {\cd num1}  being set  to 27, after which {\cd
scanf} will find the 7 instead of  the comma,  and will  terminate.
Thus no value will have been read for {\cd num2}.  If we  had left 
out the  comma from the control string, the second case would be
fine, but the first case wouldn't, for the same reason as above. 
{\cd scanf} returns the number of arguments successfully read, so we
can test for such cases.

Conversion specifiers \index{conversion specifiers!scanf@{\cd scanf}}in 
{\cd scanf} control strings have the
following form:
\begin{code}
\%$[$*$][${\ms fieldwidth\/}$]${\ms conversion\_chars}
\end{code}
\noindent
     The optional  {\cd *} indicates  that the  specified  input  is 
simply  to  be skipped, that  is, that there is no corresponding
address in the argument list to {\cd scanf}. This is useful for
skipping over data items in the input that we are not interested  in
(for  example, when  we are reading from a file of records, rather
than  the keyboard,  and are  only interested  in some  fields  of 
the records).

     Here {\ms fieldwidth\/}  specifies the {\em maximum\/} length
of the specified item.  If we are reading a string with a field
width of 10, for example, {\cd scanf} would stop reading the  string
the  moment it  had successfully  read 10  characters. The conversion
characters for {\cd scanf} are:
\begin{display}
\begin{tabular}{@{}ll@{}}
          {\cd c}       &  read a character \\
          {\cd h}       &  read a short integer in decimal \\
          {\cd d}       &  read an integer in decimal \\
          {\cd ld} or {\cd D} &  read a long integer in decimal \\
          {\cd o}       &  read an integer in octal \\
          {\cd lo} or {\cd O} &  read a long integer in octal \\
          {\cd x}       &  read an integer in hexadecimal \\
          {\cd lx} or {\cd X} &  read a long integer in hexadecimal \\
          {\cd f}       &  read a float in decimal \\
          {\cd lf} or {\cd F} &  read a double in decimal \\
          {\cd e}       &  read a float in scientific notation \\
          {\cd le} or {\cd E} &  read a double in scientific notation \\
          {\cd s}       &  read a string
\end{tabular}
\end{display}
\noindent
     Note that if we are specifying a type {\ms t\/} in the control
string, there must be a  corresponding argument in the argument list
of type {\ms pointer to\/} (that is, {\ms address of\/}) {\ms t}.


\section{Arrays and String Variables}
\index{arrays}

     An {\kc array\/}  is an  ordered collection of data items (array
elements) of the same type  (the type  of the  array).  A 
particular  element is  accessed  by specifying the collective array
name and the particular position (array index) of the  element within
the ordered collection. Array indices in C always start at zero.

     To declare an array of type {\cd t\/}, with {\cd n} elements,  we
use the form:
\begin{code}
t  {\ms arrayname\/}[n]
\end{code}
\noindent
     The elements of the array can then be accessed individually as:
\begin{code}
{\ms arrayname\/}[0],  {\ms arrayname\/}[1], \ldots,  {\ms arrayname\/}[n-1]
\end{code}
\noindent
     We have  already come  across one type of array, namely 
strings.  Although we only  used string  constants, we  can have 
string variables \index{variables!string}\index{strings} as well!
such variables are  arrays of type {\cd char}.

     C uses  an ASCII  null (0) to indicate the end of a string. 
This is the only way that the end of a string is indicated.
Thus, to store the string  {\cd "cat"} in  a string  variable (that
is, a {\cd char} array) would require an array size of  at least  4.


\section{Pointer and Array Equivalence}
\index{pointers!and array equivalence}
\index{arrays!and pointer equivalence}

     One cannot  assign a string to  an array,  only to a pointer.
Thus  the following is not correct and should be rejected by most
compilers:
\begin{code}
char  name[10];\\
name = "P. Smith";
\end{code}
\noindent
     Older compilers might not report  an error, even though something
disastrous may have occurred here.  It is instructive to examine in detail
what might happen.

An array declaration results in two actions:
 \begin{itemize}
\item space is reserved for the array in memory, and
\item the  array name is changed into a constant of type
{\ms pointer to t\/} (where {\ms t\/} is the array element type), and is
set to the address of the reserved space.
\end{itemize}
 In other words, the declaration  of {\cd name}  in our incorrect
example reserves 10 bytes of  storage  and  associates  the address
of  this storage with the identifier {\cd name}.  However, the
assignment attempts to change the value of the pointer {\cd name}
(which is supposed to be the address of 10 bytes in the {\bf .bss}
section) to the address of the string {\cd "P. Smith"} in the {\bf
.data} section.  The only way the program has access to the 10 bytes
of storage set aside in the {\bf .bss} section, is through the
address held in {\cd name}.  If this is now changed, the storage will
be lost forever to our program.

     The essential  point here  is that C converts a declaration of
type {\ms array of t\/}  into one of a constant of type {\ms pointer
to t}. The only real difference is that an array declaration (unless 
it is  external or a formal function argument, see later) reserves
some  storage. In particular, if we do not specify the array size, we
have just declared a pointer; that is, the declarations  {\cd t 
n[];} and {\cd t *n;} are equivalent. This equivalence extends far,
as we shall see next.

\section{String Functions}
\index{strings!functions for handling} 
\index{functions!string handling}

If strings can only be assigned to pointers, then how are values
assigned to string variables?  C provides a set of
functions specifically for string manipulation in the  run-time
libraries. To use these functions, the header file for the string
library must be included, using
\begin{code}
\#include <string.h>
\end{code}
\noindent
     For now, the most important functions are:
\begin{display}
\newlength{\tmpL}\settowidth{\tmpL}{\cd strcpy(s1,s2)}
\begin{tabular}{@{}lp{.6\textwidth}@{}}
    \parbox[t]{\tmpL}{\cd
     strcpy(s1,s2)\\
     char *s1,*s2;  }  &  Copies string {\cd s2} to string {\cd s1}.\\
 & \\ 
    \parbox[t]{\tmpL}{\cd
     strlen(s1)\\
     char *s1;      }  &  Returns the length of string {\cd s1}.\\
 & \\ 
    \parbox[t]{\tmpL}{\cd
     strcmp(s1,s2)
     char *s1,*s2;  }  & Returns 0 if the strings are equal, a
                         negative number if {\cd s1} is less than {\cd s2},
                         and a positive number otherwise.
\end{tabular}
\end{display}
\noindent
     The last  function, {\cd strcmp}  (string compare),  can be 
used for  equality tests and for alphabetic ordering. ``Less'' here
is taken to mean ``comes before, alphabetically.''  Since string 
assignment is  possible  using  {\cd strcpy},  our earlier
error could now be corrected with
\begin{code}
char  name[10];\\
strcpy(name,"P. Smith");
\end{code}
\noindent
     We can then access the  letters of the name by selecting elements of
the {\cd char} array. More specifically:
\begin{display}\cd
\begin{tabular}{@{}lll@{}}
 name[0] {\rm is} 'P' & name[1] {\rm is} '.' & name[2] {\rm is} ' ' \\
 name[3] {\rm is} 'S' & name[4] {\rm is} 'm' & name[5] {\rm is} 'i'	 \\
 name[6] {\rm is} 't' & name[7] {\rm is} 'h' & name[8] {\rm is} '\verb+\+0'
\end{tabular}
\end{display}

\section{Pointer Arithmetic}
\index{pointers!arithmetic}

     We mentioned  earlier that  pointers are  usually stored  as
integers\footnote{The exception is when sizes of integers and
pionters are different, e.g.\ 16-bit integers and 32-bit pointers. 
This is system dependent.}.  This means  we  can  perform  arithmetic 
on  them. However,  C  performs  pointer arithmetic  slightly  
different to   normal  arithmetic.   All  arithmetic operations are
automatically scaled  by the  size of the type to which is being
pointed.  In fact, C provides an operator, 
{\cd sizeof}\index{sizeof@{\cd sizeof}}, 
which can be
applied to any variable or type name, and  will return the number of
bytes occupied by that variable or a variable of that type. If {\cd
n} represents an integer, and {\cd p1} and {\cd p2} are pointers,
pointer arithmetic can thus be defined as 
 \begin{display}\cd
\begin{tabular}{@{}ll@{}}
  {\rm\bf Expression}   &   {\rm\bf Is equivalent to} \addVspace
          p1+n                &     p1+n*sizeof(*p) \\
          p2-n                &     p2-n*sizeof(*p) \\
          p1-p2               &     (p1-p2)/sizeof(*p1) 
\end{tabular}
\end{display}
\noindent
     The last  case is  only defined  when the  two pointers  are
pointing  at different elements  of the  same array.  The difference 
is then the number of array elements between them.

     Note that  we apply  {\cd sizeof} to  the item pointed to, and
not to the pointer.  This is  because the  size of  the pointer  is
independent of the object being pointed to, and we are interested in 
the size of the object being pointed to. A similar point (no pun
intended) applies to
arrays -- applying {\cd sizeof} to an array name will not return the
size of an array element, but  rather the size of the entire  array.
An important point is that all  {\cd sizeof} operations are evaluated
by  the compiler, and  not  by  the running program.

   The use  of pointer  arithmetic extends the equivalence  of arrays  and
pointers. In particular, if we declare:
\begin{code}
t    a[n];
\end{code}
\noindent
     where {\cd t} is some type, then the following two expressions are equivalent, and either can
be used:  
\begin{code} 
*(a+i)    \\ 
a[i] 
\end{code} 
\noindent
     The value {\cd 0} has a special meaning to pointers. It is often
defined as the macro {\cd NULL} \index{pointers!null pointer}\index{NULL}in 
{\fn stdio.h}, and indicates that
the pointer points to nothing.  When a C program is run, the first
few bytes from address 0 usually have a signature value assigned to
them. At the end of the program, this value is checked, and if it has
changed, the error ``NULL pointer assignment'' is reported.

