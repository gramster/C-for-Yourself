\cleardoublepage
\chapter{The Run-Time Libraries}
\index{libraries}
     In the previous chapters we have  examined  the whole  of the  C
language.  However, as mentioned before,  C provides  a large  number
of functions in  the  {\kc run-time libraries\/} which  can be used
to simplify programming tasks. Under UNIX there are numerous
libraries, many of which  are concerned only with UNIX system
programming. We will rather focus on the five most common libraries,
the {\kc standard input/output\/} library, the  {\kc string\/}
library,  the {\kc character  function\/} library, the  {\kc storage
allocation\/} library,  and the  {\kc maths\/} library.  To use these
libraries, we must include special  system header  files which will
give us access to them. These are:
 \begin{display}
\begin{tabular}{@{}ll@{}}
     Standard I/O         & {\cd \#include <stdio.h>} \\
     String functions     & {\cd \#include <string.h>} \\
     Character functions  & {\cd \#include <ctype.h>} \\
     Storage allocation   & predeclared by C compiler, else {\cd malloc.h} \\
     Math functions       & {\cd \#include <math.h>}
\end{tabular}
\end{display}
\noindent
     We will deal with each of these, from the simplest to the most
complex.


\section{Character Processing Functions}
\index{libraries!character processing functions}
\index{functions!character processing}
\index{character processing functions}
     The  character   processing  functions   are  used  for
classifying  and  converting characters. Each of these functions
takes a single argument of type {\cd char}.  On some systems these
functions are implemented as macros.  Be careful when using them with
arguments that have side-efects, for example increment/decrement
operators.  The most common functions
are:
\begin{hanglist}
\hangitem{\cd isalnum(c)}
          returns true if {\cd c} is an alphabetic or numeric
          character.
\hangitem{\cd isalpha(c)}
          returns true if {\cd c} is an alphabetic character.
\hangitem{\cd isdigit(c)}
          returns true if {\cd c} represents a decimal digit.
\hangitem{\cd islower(c)}
          returns true if {\cd c} is a lower case letter.
\hangitem{\cd isupper(c)}
          returns true if {\cd c} is an upper case letter.
\hangitem{\cd isprint(c)}
          returns true if {\cd c} is printable (not a control
		    character).
\hangitem{\cd ispunct(c)}
          returns true if {\cd c} is a punctuation character.
\hangitem{\cd isspace(c)}
          returns true if {\cd c} is a whitespace character (that is,
          a space, horizontal or vertical tab, carriage return,
          newline, or form feed character).
\hangitem{\cd toint(c)}
          returns the decimal value of a character representing a
               hexadecimal digit.
\hangitem{\cd tolower(c)}
          returns {\cd c},  unless {\cd c} is an upper case letter,
			 in which case the lower case equivalent is returned.
\hangitem{\cd toupper(c)}
          returns {\cd c},  unless {\cd c} is a lower case letter,
			 in which case the upper case equivalent is returned.
\end{hanglist}

\section{String Handling Functions}
\index{libraries!string handling functions}
\index{strings!functions for handling}
\index{functions!string handling}
     All of  these functions  assume that  the strings passed as
arguments are properly terminated  by  a  null byte.  In  the
following  description,  all arguments whose  names begin  with {\cd
s} are of type {\cd char*} (i.e.\ {\ms pointer to char\/}), all those
beginning with {\cd c} are {\cd char}s, and all those beginning with
{\cd n} are {\cd int}s.
 \begin{hanglist}
\hangitem{\cd char *strcat(s1,s2), \\
       char *strncat(s1,s2,n)}
 Append string {\cd s2} to string {\cd s1}, and return a pointer to the
result. {\cd strncat} appends at most {\cd n} characters of {\cd s2}
to {\cd s1}.

\hangitem{\cd char *strchr(s,c), \\
       char *strrchr(s,c)}
  Return a pointer to the first (or last if {\cd strrchr}) occurrence
of character {\cd c} in string {\cd s}.  A null pointer is returned if
{\cd c} is not in {\cd s}.

\hangitem{\cd int strcmp(s1,s2), \\
              int strncmp(s1,s2,n)}
 Compare the two strings and return {\cd 0} if they are equal, a
negative value if {\cd s1} precedes s2 alphabetically, or a positive
value otherwise.  {\cd strncmp} compares a maximum of {\cd n}
characters.

\hangitem{\cd char *strcpy(s1,s2) \\
              char *strncpy(s1,s2,n)}
 Copy string {\cd s2} to {\cd s1}, returning a pointer to the result. 
{\cd strncmp} copies a maximum of {\cd n} characters.  Care should be
taken if {\cd s2} is longer than {\cd s1}.

\hangitem{\cd int strlen(s)}
 Return the number of characters in {\cd s} up to but not including
the terminating null byte.

\hangitem{\cd int strpos(s,c), \\
              int strrpos(s,c)}
 Return the position of the first (or last, if {\cd strrpos})
occurrence of the character {\cd c} in {\cd s}.  The first character
in {\cd s} has position 0.
\end{hanglist}
Typical implementations of {\cd strcmp} and {\cd strcpy} are:
\begin{code}
strcmp(s,t)							\\
\> char *s, *t;					\\
\{ \+ 								\\
	while( *s==*t )\{ \+			\\
		if( *s==0 )return 0;		\\
		++s;							\\
		++t; \-						\\
	\}									\\
	return (*s-*t); \-			\\
\}										\addVspace

char *strcpy(s,t);				\\
\> char *s, *t;					\\
\{ \+									\\
	char *d = s;					\\
	while( *s++=*t++ );			\\
	return d; \-					\\
\}
\end{code}
\noindent


\section{Mathematical Functions}
\index{libraries!mathematical functions}
\index{functions!mathematical}
     The  mathematical   functions  provide   a  large   set  of
mathematical operations, including  trigonometric and  hyperbolic
functions,  and a  random number generator.  The large  number of
these functions,  as  well  as  their (mostly) self-evident  names,
makes it tedious to list them here. Consult your C/UNIX manuals (for
example, the UNIX Programmer's Reference Manual volume 1 section 3)
for details.

\section{Storage Allocation Functions}
\index{libraries!storage handling functions}
\index{functions!for storage handling}
     It is  possible to  allocate and deallocate memory from the
system ``heap'' using these functions (which are predeclared by the C
compiler, and hence need no {\cd \#include} file header). The
functions are:
\begin{hanglist}
\hangitem{\cd char  *calloc(num,size) \\
       unsigned num, size;}
 Allocate enough memory to hold an array of {\cd num} elements
of size {\cd size} bytes each, clear the memory to zeroes and
return a pointer to the start. There is a variant {\cd clalloc},
which takes {\cd unsigned long} arguments, used for allocating large
areas of memory.

\hangitem{\cd cfree(mem) \\
       char *mem;}
 Free (de-allocate) the memory pointed to by {\cd mem}, where
{\cd mem} is a pointer returned by some previous {\cd calloc} call.

\hangitem{\cd char *malloc(size) \\
       unsigned size;}
 Allocate {\cd size} bytes of memory, and return a pointer to the
start. The memory is not initialised in any way, and must be
assumed to contain garbage. A {\cd long} version, {\cd mlalloc}, is
also available.

\hangitem{\cd free(mem) \\
       char *mem;}
 same as {\cd cfree}, but for memory allocated by {\cd malloc}.

\hangitem{\cd char *realloc(mem,size) \\
       char *mem;	       \\
       unsigned size;}
 Change the size (without damaging the contents) of the previously
allocated memory region pointed to by {\cd mem}.  Return a pointer to
the result.  If this returned pointer is not the same as {\cd mem},
then {\cd mem} was not a valid argument and should not be used).
There is a {\cd long} version, {\cd relalloc}.

\end{hanglist}

\section{Standard Input and Output}

     The standard  input/output header  file,  
{\fn stdio.h}\index{stdio.h@{\fn stdio.h}},  defines
some  useful macros and  variables for  use with input and output.
These include an end of file marker,  {\cd EOF}\index{EOF@{\cd EOF}}, and a new type,
{\cd FILE}\index{FILE@{\cd FILE}}, used for declaring  streams\index{streams}. A {\kc stream\/} can be a
file  or some other producer or consumer of data, such as a keyboard
or screen. In particular, {\fn stdio.h} declares three  file
pointers\index{stdin@{\cd stdin}}\index{stdout@{\cd stdout}}\index{stderr@{\cd stderr}}:
\begin{code}
extern FILE *stdin, *stdout, *stderr;
\end{code}
\noindent
     which represent  the standard streams for use by functions such
as {\cd printf} and {\cd scanf},  and system  error messages.  All
of  our examples so far have used {\fn stdin} and {\fn stdout}, which
can be easily redirected by UNIX to almost any stream.  This is
inadequate, however, if we need more than just one file each for
input and output. We can declare additional file pointers ourselves,
and use them as arguments to input and output functions.

     The following functions are for use with {\fn stdin} only:
\begin{hanglist}
\hangitem{\cd int getchar()}
	Return the next character (or {\cd EOF}) from {\fn stdin}.

\hangitem{\cd char *gets(s)   \\
       char *s;}
	Read characters from {\fn stdin} into the array pointed to by
	{\cd s} until a newline or {\cd EOF} is encountered.  The newline is
	discarded and the string is terminated with a null.  Return
	{\cd s} if successful, otherwise null.

\hangitem{\cd int scanf(f,{\ms list\_of\_pointers\/}) \\
       char *f;}\index{scanf@{\cd scanf}}
	Described previously.  Return the number of fields that were
	assigned successfully.
\end{hanglist}
     The following functions are for use with {\bf stdout} only:
\begin{hanglist}
\hangitem{\cd int putchar(c)		 \\
       char c;	}
	Write the character {\cd c} to {\fn stdout}.  Return {\cd c} as
	an {\cd int} if successful, {\cd EOF} otherwise.

\hangitem{\cd int puts(s)		    \\
       char *s;}
	Write the string {\cd s} to {\fn stdout} and append a newline.  
	Return a newline if successful, {\cd EOF} otherwise.

\hangitem{\cd int printf(f,{\ms list\_of\_expressions\/}) \\
       char *f;} \index{printf@{\cd printf}}
	Described previously.  Return the number of characters actually
	written if successful, a negative value otherwise.
\end{hanglist}
     All other  input and  output functions  take a  stream as an
argument. To open and close streams, we use:
\begin{code}
       FILE *fopen(name,mode)		 \\
       char *name,*mode;		 \addVspace

       FILE *freopen(name,mode,stream)	 \\
       char *name,*mode;		 \\
       FILE *stream;			 \addVspace

       int fclose(stream)		 \\
       FILE *stream;
\end{code}
\noindent
     {\cd fopen} opens the stream called {\cd name}, and returns a
pointer to it. The {\cd mode} specifies what the stream is to be used
for; it can be any one of:
\begin{display}
\begin{tabular}{@{}lp{.8\textwidth}@{}}
       {\cd "r"}  &   open an existing file for input \\
       {\cd "w"}  &   create a new file for output    \\
       {\cd "a"}  &   create a new file for output or append to existing file \\
       {\cd "r+"} &   open an existing file for input and output \\
       {\cd "w+"} &   create a new file for input and output	  \\
       {\cd "a+"} &   create a new file or append to an existing file, for
                    input and output
\end{tabular}
\end{display}
\noindent
     {\cd freopen} is  similar, except  it calls  {\cd fclose} first.
In other  words, it allows us  to associate  our stream  variable
with  a different  stream using a single call  to {\cd freopen} rather
than one to {\cd fclose} followed by one to {\cd fopen}. If an error
occurs in attempting to open a file, both {\cd fopen} and {\cd
freopen} return a null pointer (zero).

     {\cd fclose} closes a stream. If any error occurs, it returns {\cd EOF}, otherwise it
returns zero.

     To check the status of a stream, we use:
\begin{code}
       int feof(stream) \\
       FILE *stream;   \addVspace

       int ferror(stream) \\
       FILE *stream;	  \addVspace

       void clearerr(stream) \\
       FILE *stream;
\end{code}
\noindent
     {\cd feof} checks  whether the  end of {\cd stream} has been
reached, and returns true if  this is  the case,  false otherwise.
{\cd ferror} returns true if the last read or  write operation  to
{\cd stream} resulted in an error. {\bf clearerr} resets any
such error indication on the stream.

     The input and output operations on the streams are:
\begin{code}
       int getc(stream) \\
       FILE *stream;	\addVspace

       int fgetc(stream)   \\
       FILE *stream;	       \addVspace

       int ungetc(c,stream)	     \\
       char c;				 \\
       FILE *stream;			   \addVspace

       char *fgets(s,n,stream)		 \\
       char *s;				 \\
       int n;				 \\
       FILE *stream;			 \addVspace

       int fscanf(stream,f,{\ms pointer\_list\/}) \\
       FILE *stream;				\\
       char *f;				       \addVspace

       int putc(c,stream)		       \\
       char c;				       \\
       FILE *stream;			       \addVspace

       int fputc(c,stream)		       \\
       char c;				       \\
       FILE *stream;			       \addVspace

       int fputs(s,stream)			\\
       char *s;					 \\
       FILE *stream;				 \addVspace

       int fprintf(stream,f,{\ms argument\_list\/})  \\
       FILE *stream;				   \\
       char *f;					   \addVspace

       int fflush(stream)			   \\
       FILE *stream;
\end{code}
\noindent
     {\cd getc} and  {\cd putc} are macros, while {\cd fgetc} and {\cd
fputc} are functions. These are the equivalent of {\cd getchar} and
{\cd putchar} but for any stream. Similarly, {\cd fputs} and {\cd
fgets} are  the equivalent  of {\cd puts} and {\cd gets}, but for any
stream. One difference, however, is that the newline is not striped
or appended as with {\cd gets} and {\cd puts} respectively.  
The same goes for {\cd fprintf} and {\cd fscanf}. The only
``new'' functions here are {\cd ungetc} and {\cd fflush}.

     {\cd ungetc} attempts  to push  the character {\cd c} back into
an input stream. Thus the next  call to {\cd getc} on that stream
would return {\cd c}. This may not be possible -- if it is, {\cd c}
is returned; otherwise {\cd EOF} is returned.

     {\cd fflush} flushes the buffers associated with a stream. Most
input and output in UNIX  is buffered and thus writing  to a  stream
does  not mean  that  the  actual  device  or  file associated with
that stream  will get  the output  immediately. For  example,
terminal I/O  is buffered  on a line-by-line basis. The terminal does
not send keyboard input  or print  screen output until it receives a
character.  Say, however,  we wanted  to prompt for user input, and
have the prompt appear immediately after  our message (in other
words, we don't want a newline at the end of our message).  Using:
\begin{code}
printf("Please enter Y or N "); c=getchar();
\end{code}
\noindent
     is not  good enough,  as the  message won't  be printed -- no
newline has occurred. We can achieve the result we want with:
\begin{code}
printf("Please enter Y or N "); fflush(stdout); c=getchar();
\end{code}
\noindent
     When we  are using  files  for  input  and  output,  we  need
additional functions to  allow us  to move  around the  file, and  so
on.  The  following functions are usually provided for this purpose:
\begin{code}
       int fread(buf,size,num,stream)	      \addVspace

       int fwrite(buf,size,num,stream)	      \\
       char *buf;			      \\
       unsigned size;			      \\
       int num;				      \\
       FILE *stream;			      \addVspace

       int fseek(stream,offset,type)	     \\
       FILE *stream;			     \\
       long offset;			     \\
       int type;			     \addVspace

       long ftell(stream)		     \\
       FILE *stream;			     \addVspace

       void rewind(stream)		     \\
       FILE *stream;
\end{code}
\noindent
     {\cd fread} and  {\cd fwrite} respectively reads  and writes
blocks of binary data consisting of at most {\cd num}  items of  size
{\cd size}  from/into the buffer pointed to by {\cd buf}.
The number of items actually read or written is returned.  When
reading binary blocks, we cannot use {\cd EOF} to tell if the end of
the file has been reached; we must use {\cd feof} instead.

     {\cd fseek} seeks to a position in the specified stream,
allowing random access to the  stream. If  the seek  is successful,
{\cd fseek}  returns  zero.    {\cd type} specifies to what {\cd
offset} is  relative;  if {\cd type} is zero,  {\cd offset} is
treated as  relative to  the beginning of the stream, if one, it is
treated as relative to the current position, and if two, as relative
to the end of the file.

     {\cd ftell} returns  the current  position in  {\cd stream}.
This can then later be used in calls to {\cd fseek} to restore this
position.

     {\cd rewind} sets  a  stream  back  to  the  beginning;  it  is
equivalent  to the function call {\cd fseek(stream,0L,0)}.



\section{Executing UNIX commands from C}

     In UNIX  environments, it  is easy  to invoke  other commands or
programs from yours  -- a function 
{\cd system} \index{system()@{\cd system()} function}
\index{UNIX!executing commands from programs}has been included in
the runtime libraries for this. {\cd system}  takes a  string
argument;  this argument  is passed  to the UNIX shell and  executed,
after  which execution  of  the  program  continues.  For example, to
print the date, you can simply say:
\begin{code}
system("date");
\end{code}
\noindent
     All C environments should provide some way of invoking other
programs and commands from  within programs.  If the {\cd system}
function is not provided, a related  function called  {\cd exec} is
probably  available. Check your system documentation under  {\em
System Calls\/}  (for example,  UNIX Programmer's Reference Manual
volume 1 section 2).

