
\section{Statements}

     Statements are  the most  complex constructs  in C  other than
functions (although  a  function  body  is  simply  a  large
compound  statement).  All statements in  C, except  compound
statements,  must end with a semicolon. Any expression may be treated
as a statement; it is simply evaluated and its value discarded.

     A  {\kc compound  statement\/}  \index{statements!compound}or 
{\kc block\/}  consists  of
(optional)  declarations followed by  a (possibly empty) sequence of
statements, all enclosed in braces {\cd \{\}}. A compound statement may
appear anywhere a statement may. Any declarations that are  not {\cd
extern}al  are local  to the  block, and  cannot be accessed from
outside the block.


\subsection{The {\cd if} Statement}
\index{statements!if@{\cd if}}
\index{statements!else@{\cd else}}
     An {\cd if} statement has the form:
\begin{code}
if( {\ms expression\/} ) \\
 \> {\ms statement\/} 	 \\
else 							 \\
 \> {\ms statement\/}
\end{code}
\noindent
     The {\cd else}  {\ms statement\/} part may be omitted.  The
expression is evaluated; if it is  true, the  first statement  is
executed.  If it is false, the statement following {\cd else}  is
executed.  If there  is no  {\cd else} part and the expression is
false, the  {\cd if} statement  is skipped  and the  program
continues with the next statement.

   {\cd if} statements can be  cascaded as in:
\begin{code}
if( {\ms expression\/} )		\\
 \> {\ms statement\/}			\\
else if( {\ms expression\/} ) \\
 \> {\ms statement\/}			\\
else if( \ldots\  ) \\
 \vdots							\\
else {\ms statement\/}
\end{code}
\noindent
     An  {\cd else}  always  refers  to  the  closest  possible  {\cd
if}.  If  any  other interpretation is needed, braces can be used.
For example
\begin{code}
if({\ms e1\/}) if({\ms e2\/}) {\ms s1\/} else {\ms s2\/}
\end{code}
\noindent
     where {\ms e1\/} and {\ms e2\/} are expressions, and {\ms s1\/} and
{\ms s2\/} are statements, groups as
\begin{code}
if({\ms e1\/}) \{ if({\ms e2\/}) {\ms s1\/} else {\ms s2\/} \}
\end{code}
\noindent
     If we wanted it to group as
\begin{code}
if({\ms e1\/}) \{ if({\ms e2\/}) {\ms s1\/} \} else {\ms s2\/}
\end{code}
\noindent
     we would have to use the braces as shown.



\subsection{The {\cd while} Statement}
\label{while_st}
\index{statements!while@{\cd while}}
     The {\cd while} statement is one of C's looping statements. It has
the form:
\begin{code}
while( {\ms expression\/} ) \\
\> {\ms statement\/}
\end{code}
\noindent
     The {\ms expression\/}  is evaluated; if it  is false,  the
rest of the {\cd while} statement is skipped and  the program
continues with the next statement. If {\ms expression\/} is true, 
{\ms statement\/} is  executed, and  the program continues from the
start of the while statement. Thus, as long as {\ms expression\/} is
true,  {\ms statement\/} is repeatedly executed. Note that  if {\ms
expression\/} is false right  at  the start, {\ms statement\/} is not
executed at all.


\subsection{The {\cd do} Statement}

\index{statements!do@{\cd do}}
     This is similar to the {\cd while} statement. It has the form:
\begin{code}
do  \\
 \> {\ms statement\/} \\
while( {\ms expression\/} );
\end{code}
\noindent
     The main  difference between  a {\cd do}  and a {\cd while}
statement is that in a {\cd do} statement, {\ms statement\/} is 
executed  before the  {\ms expression\/}  is evaluated. Thus 
{\ms statement\/} is always executed at least once, even if
{\ms expression\/} is false to begin with.


\subsection{The {\cd for} Statement}
\index{statements!for@{\cd for}}

     This is C's most powerful looping command. It has the form:
\begin{code}
for( {\ms expression1\/}; {\ms expression2\/}; {\ms expression3\/} ) \\
 \> {\ms statement\/}
\end{code}
\noindent
     Each of  the three  expressions is  optional, but  the
semicolons must be present. The effect of the {\cd for} statement is
as follows:  Firstly, {\ms expression1\/} is evaluated. This  is used
to set up any variables needed by the loop. Then the following
process is repeatedly performed:  {\ms expression2\/} is  evaluated;
if  its value is  zero, the {\cd for} statement terminates and  the
program  continues with the next statement. Otherwise,  {\ms
statement\/} is executed, after which {\ms expression3\/} evaluated.

     Thus, {\ms expression2\/}  is used to test for loop
termination, while  {\ms expression3\/} is  usually used  to modify 
some  variable(s)  used  in controlling the loop.

The {\cd for} loop is equivalent to the following {\cd while} loop:
\begin{code}
{\ms expression1\/}; \\
while( {\ms expression2\/} )\{ \\
\>  {\ms statement\/}; \\
\>  {\ms expression3\/}; \\
\} 
\end{code}
\noindent
     Let's see some examples. Firstly, the BASIC for loop:
\begin{code}
FOR I=0 TO 100 STEP 10
\end{code}
\noindent
     could be written in C as:
\begin{code}
for( i=0; i<=100; i+=10 )\{ \\
\> \vdots \\
\}
\end{code}
\noindent
     The C  {\cd for} is  much more  powerful. For one thing,
sequential expressions \index{expressions!sequential}may be  used inside the {\cd for(\ldots)}, and 
furthermore, the expressions are arbitrary. For example, to loop
through all the powers of two up to 1000, we could use:
 \begin{code}
for( i=1; i<=1000; i*=2 )\{ \\
\> \vdots \\
\}
\end{code}
\noindent
     Or, to read characters until a space is entered:
\begin{code}
for( ; getchar()!=' '; );
\end{code}
\noindent
     In particular,  this last  example shows how any {\cd while}
loop could be done with a {\cd for}. The {\cd while} 
loop\index{statements!relationship between {\cd for} and {\cd while}}:
\begin{code}
while( {\ms expression\/} ) {\ms statement\/}
\end{code}
\noindent
     is the same as:
\begin{code}
for( ; {\ms expression\/}; ) {\ms statement\/}
\end{code}
\noindent
     Conversely, we have already seen how any {\cd for} loop can be
replaced by a {\cd while}. 

As a  more sophisticated  example, here  is
a function that tests whether two strings are equal:
\begin{code}
int stringeq(s1, s2) \\
\> char s1[], s2[]; \\
\{ \+\\
  char *p1, *p2; \addVspace

  for( p1=s1, p2=s2; *p1 \&\& *p2; ++p1, ++p2)\{ \\
  \>   if( *p1 != *p2) return FALSE; \\
  \} \\
  return *p1==*p2; 
  \-\\
\} 
\end{code}
\noindent
    The first expression sets the pointer variables {\cd p1} and {\cd
p2} to the start of the strings.  The second  expression is  true if 
and only  if neither  of the current characters  being compared is a
zero; in other words,  we have not yet reached  the end  of
either string. The third expression sets the pointers to point to the
next two characters for comparison.

     The {\cd if}  statement is  executed if the second expression is
true, that is, we have  not reached  the end of a string. In this
case, if the two characters being compared are different, the strings
are not equal and we return false.

     If we  reach the  end of  one of  the strings  and have  not yet 
found a mismatch, we  exit the  {\cd for} loop.  Now, either we have
reached the end of both strings and  the strings are equal, or one
string is longer than the other. We can simply  return the  value of
the comparison of the last two characters for this --  at least  one
must  be zero,  so the other is either also zero (and we return true)
or not (we return false).

Note that in the above example we did not have to use braces ({\cd
\{\}}) to delimit the body  of the {\cd for} loop, since it consists
only of one statement.  We nevertheless did so because it is good
programming practice -- if at a later stage we wish to add a
statement to the {\cd for} body, we do not have to remember to add
the braces.



\subsection{The {\cd switch} Statement}
\index{statements!switch@{\cd switch}}
     This is a multiway branch based on the value of an expression,
similar to BASIC's {\cd on...goto} and Pascal's {\cd case...of}. It
has the form:
\begin{code}
switch( {\ms expression\/} ) \\
 \> {\ms compound\_statement\/}
\end{code}
\noindent
  In the execution of the {\cd switch} statement, {\ms expression\/} 
(also known as the switch expression) is evaluated,  and depending 
on its value, a branch is made to the appropriate {\kc case label\/}
in {\ms compound\_statement\/}. As the name implies, {\ms
compound\_statement\/} must be a compound statement. Any statement
within {\ms compound\_statement\/} may  be labeled  with a  case
label of the form:  
 \begin{code} 
case  {\ms constant\_expression\/}:  
\end{code} 
\noindent
     The type  of {\ms constant\_expression\/} should be the same as
the type {\ms expression\/}, and no two {\ms constant\_expression\/}s
in the case labels of the same {\cd switch} statement may have  the
same value. If the switch expression has the same  value as the
{\ms constant\_expression\/} of  some case label, then  the compound
statement  is entered at that point. If no match is
found, the compound statement is skipped and the program
continues from the next statement.

     There is  a special  case label,  called {\cd default:},  which
will  match any value that is not matched by some other case label.

     As an example, consider:
\begin{code}
 void triangle( height ) \\
 \> unsigned short height; \\
 \{ \+ \\
    switch( height )\{ \+\\
         case 4:   printf(" \verb+*******\+n"); \\
         case 3:   printf(" \verb+ *****\+n"); \\
         case 2:   printf(" \verb+  ***\+n"); \\
         default:  printf(" \verb+   *\+n"); 
			\-\\
    \} \-\\
 \} 
\end{code}
\noindent
which will print
{\cd\begin{tabular}[t]{c}*****\\{***}\\{*}\end{tabular}}
when called with {\cd triangle(3)} and 
{\cd\begin{tabular}[t]{c}***\\{*}\end{tabular}}
when called with {\cd triangle(2)}.  A call to {\cd triangle} with
any other parameter value than 2, 3, or 4, will cause a branch to
the {\cd default} label and a single {\cd *} will be printed.

    More than one case label may refer to the same statement in the
compound statement, for example:
\begin{code}
avoid quack\_shrink() \\
\{ \+\\
   char c; \addVspace

   printf("Do you like yourself?\verb+\+n"); \\
   c=getchar(); \\
   switch( c )\{ \\
   case 'y': \\ 
	case 'Y': \\
	\> printf("You're cured!\verb+\+n"); \\
   \> break; \\
   case 'n': \\
	case 'N': \\
	\> printf("Don't be so neurotic!\verb+\+n"); \\
	\} \\
   printf("Don't forget to leave my fifty bucks. Come again!\verb+\+n"); 
	\-\\
\} 
\end{code}
\noindent
(There is an error in the above program, can you spot it?)
The {\cd break} statement is described below.



\subsection{{\cd break}, {\cd continue}, {\cd goto}  and {\cd exit()}}

     A {\cd break}  \index{statements!break@{\cd break}}statement causes 
the execution of the smallest
containing {\cd while}, {\cd do}, {\cd for}  or {\cd switch}  statement to 
be terminated.  Execution continues with the next statement after the
{\cd while}, {\cd do}, {\cd for}, or {\cd switch}.

A {\cd continue}  \index{statements!continue@{\cd continue}}statement 
causes  the execution  of the
smallest  containing {\cd while}, {\cd do} or {\cd for} loop's {\ms
statement\/} to be terminated.  Execution resumes at the start of the
loop. The use of {\cd continue} is discouraged, as it is often a sign of
a poorly structured program.

     An example of using of {\cd break} and {\cd continue} is:
\begin{code}
 for( i=0; i<5; ++i )\{ \+\\
      printf("At start of loop; i=\%d\verb+\+n",i); \\
      if (i==3) break; \\
      if (i==1) continue; \\
      printf("At end of loop; i=\%d\verb+\+n",i); \-\\
 \} \\
 printf("Out of loop - i=\%d\verb+\+n",i); 
\end{code}
\noindent
     This will result in the printout:
\begin{code}
  At start of loop; i=0 \\
  At end of loop; i=0 \\
  At start of loop; i=1 \\
  At start of loop; i=2 \\
  At end of loop; i=2 \\
  At start of loop; i=3 \\
  Out of loop - i=3 
\end{code}
\noindent
     A {\cd goto}  \index{statements!goto@{\cd goto}}statement can 
be used to transfer control from any
statement in a function to any other statement in the function. The
general form is:
\begin{code}
goto  {\ms identifier\/};
\end{code}
\noindent
     The destination  statement must  be preceded by  {\ms
identifier\/} followed by {\cd :}.  The use  of {\cd goto}  is
strongly discouraged, with one exception -- if some kind of error
condition occurs in a function or a multiple nested construct, 
it is acceptable {\cd goto} the
end of the function or to use {\cd goto} to break out of the nested
construct. For example:
\begin{code}
  funk()							\\
  \{ 								\+\\
     if( \ldots\ )\{			\+\\
       \vdots					\\
		 while( \ldots\ )\{	\+\\
		   \vdots				\\
         if( error ) goto disaster; \\
		   \vdots 				\-\\
       \} 						\\
		 \vdots 					\-\\
     \} 							\\
	  \vdots 					\\
	  return; 					\-\\
  disaster: 					\\
  \> printf("Error in funk\verb+\+n"); \\
  \} 
\end{code}
\noindent
     The appearance  of a label is taken as the declaration of the
label; thus there is no need to explicitly declare labels.

C provides  a system  call called  
{\cd exit({\ms n\/})}\index{exit()@{\cd exit()} system call}.  The
execution of an {\cd exit} call terminates the program, and returns
the argument value {\ms n\/}. This is usually 0 if the program
terminated satisfactorily, or some non-zero error code otherwise. The standard
UNIX error codes are contained in the include file {\fn errno.h}. 

\subsection{The null Statement}
\index{statements!null}
     This consists  of just  a semicolon. It is useful in two
situations: when we have  a loop  which needs  no {\ms statement\/}
following (as in some of our loop examples above),  and for  allowing
labels for {\cd goto}s at the end of blocks (for example, if we
didn't want an error message in {\cd funk} above).



\section{More About Declarations}
\index{declarations}
     The {\kc scope\/}\index{declarations!scope}  of a  declaration is  the region of the program
over which the declaration has effect. C declarations may have one of
five scopes:
\begin{itemize}
 \item {\kc top-level declarations\/} have a scope extending from the
declaration to the end of  the program, even if the program is split
up into several files (but see below);

\item {\kc arguments\/} have their function bodies as their scope;

\item {\kc block declarations\/} have a scope extending to the end of
the block;

\item {\kc labels\/} have their containing function bodies as their
scope;

\item {\kc macros\/} have scope from their declarations until the end
of the program or an {\cd \#undef} command for the same macro,
whichever comes first.

\end{itemize}
      The scope  of every  declaration is  limited to  the  file
containing it, unless it is a global ({\cd extern}, or non-static
top-level) declaration.

     A declaration  may be  hidden by  other declarations whose scope
overlaps that of the first declaration. For example:
\begin{code}
  int x; \addVspace

  main() \\
  \{ \+\\
    int x; \\
    \vdots \\
    \{ \+\\
       int x; \\
       \vdots \-\\
    \} \-\\
  \} 
\end{code}
\noindent
    Within the  function, the  top-level declaration  of {\cd x}  is hidden 
by the declaration at  the start  of {\cd main()}.  Within the
sub-block, both of the outer declarations are hidden.

     When an  identifier is  used before  it is  declared, it is
known as a {\kc forward 
reference\/}\index{forward references}\index{declarations!forward}. 
C  allows labels  to be  forward
referenced  (as well as structure, union   and enumeration tags, see
later). Referring to a function before it is declared is  acceptable,
provided  the function  returns  an  integer.  If  it returns any 
other type,  it must  have a  {\kc forward declaration\/} to enable
the compiler to know what return type to expect. For example:
\begin{code}
float area(), volume();
\end{code}
\noindent
     would tell  the compiler  that the  two functions  {\cd area}
and {\cd volume} each return a {\cd float}, not an  integer.

     Identifier names  in C  can be  {\kc overloaded\/} (that is, the
same name can be used for  different  objects)  provided  they  are 
in  different  {\kc overloading classes}\index{declarations!overloading classes}
\index{overloading classes}.  There are five overloading
classes:
\begin{itemize}
 \item {\kc preprocessor  macro names\/}  -- these  are independent
of any other names in the C program, as preprocessing occurs first
and in a separate pass. This overloading class is an exception  --
all  occurrences in the program of macro names will be replaced. Thus
macro names cannot be overloaded.

\item {\kc statement labels\/}

\item {\kc structure, union\/} and {\kc enumeration tags\/} (see later)

\item {\kc structure\/} and {\kc union component names\/}

\item all  {\kc other names\/}  fall into an overloading class that
includes variables, functions, {\cd typedef}  names (see  later) and 
enumeration constants  (see later).

\end{itemize}
     It is  illegal to  make two  declarations of  the same  name in 
the same overloading class  in the  same block  or at  the  top 
level;  we  call  such declarations {\kc conflicting  declarations\/}
as  the compiler cannot determine which declaration should be used.
The only exception is {\cd extern} declarations, which can  be used 
freely, provided  that there  is  one and only one defining 
declaration somewhere for each externally  declared identifier. For
example, our functions {\cd area} and  {\cd volume} above have not
been defined although they have been declared; thus there  must still
be defining declarations  for them somewhere in the program, and
these defining declarations would not conflict with the {\cd extern}
declarations.

     At most  one {\kc storage class specifier\/}  \index{storage classes}may appear  in a 
declaration. The available storage classes are:  
 \begin{itemize}\cd
\item auto

\item extern

\item static

\item register

\item typedef
\end{itemize}
     We have  already seen  the first  three. A {\cd register}
\index{storage classes!register@{\cd register}}
\index{register@{\cd register}}declaration is useful when a  particular 
variable  is used  heavily
in  a function which needs to be optimised for  speed (for example, a
sort routine). The {\cd register} storage class tells the  compiler
to  allocate one  of the computer's hardware registers for that
variable,  if possible.  Some compilers simply ignore this storage
class.  You may not use the {\ms address-of} operator ({\cd \&}) with
register variables.

A {\cd typedef} 
\index{typedef@{\cd typedef}}\index{storage classes!typedef@{\cd typedef}}
declaration simply declares a name as a synonym
for a type. It looks just  like a  normal declaration,  but the 
``variable'' being declared is actually just  a name  which can  be
used  to declare  other variables of that type. For example:
\begin{code}
typedef int \=*pointer, \\
         \>   (*PFint)();
\end{code}
\noindent
     declares {\cd pointer}  to be  a name  for the  type {\ms pointer
to int\/}, and {\cd PFint} to be a name for  the type  {\ms pointer to
function returning int\/}. Once a {\cd typedef} name has been declared,
it may be used wherever a type specifier is needed; thus:
\begin{code}
 pointer \= a, b; \\
 PFint \> f1, f2; 
\end{code}
\noindent
     declares {\cd a}  and {\cd b}  to be {\ms pointers to int\/}, and
{\cd f1} and {\cd f2} to be {\ms pointers to functions returning
int\/}.



\section{Initialisation Within Declarations}
\index{declarations!initialisers for}
     When we  declare a  variable, we cannot assume anything about its
initial value. Some  compilers initialise all declared variables to
zero, while others simply allocate  storage, regardless  of what
values may be in that storage at the  time.  However,  when  we 
declare  a  variable,  we  can  initialise  it immediately using
\begin{display}\ms
type identifier\/ {\cd =} initialisation
\end{display}
\noindent
     Initialisers for  {\cd static} variables must be constant
expressions, as these variables are  initialised by  the compiler
(and placed in the {\bf .data} section).  Any expression may be used
for {\cd auto} variables, as these are evaluated every time the
automatic  variables  are  created.  Formal  arguments  of functions
may not  be initialised.

     As some compilers cannot perform floating point arithmetic, {\cd
static float}s should be initialised with numbers only (not arbitrary
constant expressions).

     Some examples of {\cd int}, {\cd char} and {\cd float}
initialisers are\index{variables!initialisation of declared}:
\begin{code}
  static int count=4*200, sum=0; \addVspace

  main() \\
  \{ \+\\
       char ch=getchar(); \\
       \vdots \-\\
  \} \addVspace

  static void calculate(k) \\
  \> double k; \\
  \{ \+\\
       static double epsilon=1.0e-6; \\
       auto float leeway = k*epsilon;\tab /* illegal in some compilers */ \\
       \vdots \-\\
  \} 
\end{code}
\noindent
     Initialisers for pointers \index{pointers!initialisers of}must 
evaluate to an integer, or
an address plus or minus an integer constant. For example:
\begin{code}
  \#define NULL 0 				\\
  double *dp = NULL; 				\\
  long *population = (long *)32000000; 		\addVspace

  char buffer[100]; 		\\
  char \=*bufptr = buffer,		\\
    \>   *bufend = \&buffer[99];	\addVspace

  static short s; 		\\
  short \=*sp1 = \&s,		\\
    \>    *sp2 = \&s+3; \tab\=/* a valid initialisation, */ \\
 \> \>                        /* but not recommended */	\addVspace

  char *greet = "Press <return> to begin\verb+\+n";	\addVspace

  float powers\_of\_pi[8]; 	\\
  float \=*pi = \&powers\_of\_pi[1], 	\\
   \>     *pi\_sq = \&powers\_of\_pi[2]; 
\end{code}
\noindent
     Arrays can  be initialised as well\index{arrays!initialising}. 
The last subscript of the
array forms the ``most tight'' group.  For example,
\begin{code}
  int a[5] = \{ 0, 1, 2, 3, 4 \};	\addVspace

  int b[2][3] = \{					\\ 
  \> \{10, 20, 30\},					\\
  \> \{40, 50, 60\}					\\
  \} 										\\
  int table[2][3][4] = \{			\\
  \> \{ \{0,1,2,3\}, \{4,5,6,7\}, \{8,9,10,11\} \}, \\
  \> \{ \{11,10,9,8\}, \{7,6,5,4\}, \{3,2,1,0\} \}  \\ 
  \}; 
\end{code}
\noindent
     If the  number of initialisers is less than the number of array
elements, the remaining elements are initialised to zero. We also do
not need to specify the size  of the  array, in  which case it is
worked out from the initialiser.  For example:
\begin{code}
char *names[] = \{"John", "Mary", "Bob"\};
\end{code}
\noindent
     will declare a three-element array of pointers to chars.

     When initialising  {\cd char} arrays, remember that space must
also be reserved for the terminating {\cd '\verb+\+0'}.  For example,
\begin{code}
char word[] = \{'c','a','t','\verb+\+0'\};
\end{code}
\noindent
Recall it is not correct to say
\begin{code}
char word[] = \{"cat"\};
\end{code}
\noindent
as {\cd "cat"} has type {\ms pointer to char\/}, whereas {\cd word}
is an array whose elements are of type {\cd char}.

Finally, the following examples are all equivalent ways of declaring
and initialising a 4-{\cd char} array to contain a string, although some 
are clearer than others:
\begin{code}
char str[4] \== \{ 'U', 'C', 'T' \}; \\
char str[] \>= \{ 'U', 'C', 'T', '\verb+\+0' \}; \\
char str[] \>= "UCT"; \\
char str[] \>= \{ "UCT" \};
\end{code}
