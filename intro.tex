\chapter{Introduction}

\section{A Short History of C and UNIX}
\index{history of C}

   In the 1960's, as the use of computers changed from batch
processing systems to the new concept of time-sharing, operating
system design was a major computing issue. In England, the
Universities of London and Cambridge embarked on the Combined
Programming Language (CPL) project, an attempt to provide a single
language for all applications, including what were previously thought
to be assembly language applications.

   CPL was not entirely successful, and never came into general use,
but it did inspire the language BCPL (Basic CPL). BCPL is a system's
programming language with the main distinguishing feature that it is
typeless - the only data objects allowed are machine words; it is up
to the programmer to implement types such as characters, strings,
real numbers, and so on.

   In America, much work was being done on the Multics (MULTiplexed
Information and Computing Service) project, a joint Massachusetts
Institute of Technology, Bell Laboratories and General Electric
Corporation time-sharing project started in 1965 as part of MIT's
``Project MAC.'' After three years, no usable system had yet been
produced, and Bell Laboratories withdrew from the project. Ken
Thompson and Dennis Ritchie were two of the last Bell Labs employees
still working on Multics, and were unhappy with the idea of losing
the limited time-sharing capability Multics had thus far produced.
They began designing their own file system on paper. Thompson found
an unused PDP-7 to play with, and their file system was soon
bootstrapped onto it. They were encouraged in their task by the fact
that CPU time on the mainframe for Thompson's elaborate ``Space
Travel'' game cost about \$75 a game, while the PDP-7 was theirs to
play with and had superior graphics!

   Some essential utilities for file handling, including printing and
editing, were added, as well as a way to run system processes. In 1970,
Brian Kernighan named this rudimentary system ``UNIX'' --  a pun on
Multics.

   The UNIX system was moved onto a PDP-11 for developing a text
processing system for Bell, and the First Edition documentation written
by Ritchie and Thompson in late 1971, the Second Edition appearing in
June 1972.

   B, a direct descendant of BCPL, was used by Thompson for the Second
Edition assembler. He extended B into a language called NB, in the hope
of using it to rewrite UNIX. However, this proved unsuccessful. Ritchie,
attempting to improve the performance of NB, came up with a similar
language, which he called C. In 1973 structures were added to C, adding
considerable power, and UNIX was rewritten in C.

   Due to a government agreement, Bell Labs could not market UNIX, but
provided it free to universities, which resulted in a large growth of
software, particularly at the University of California at Berkeley.

   In 1978 Kernighan and Ritchie published {\em The C Programming
Language\/}
(Prentice-Hall), still considered the definitive reference on C. A
summary from this book is reprinted as ``The C Programming Language -
Reference Manual'' by Dennis Ritchie (see ``UNIX Programmer's Manual
volume 2'', Bell Laboratories).

   Although there is an American National Standards Institute (ANSI)
proposed standard\index{ANSI C}, C dialects abound. However, most implementations
conform closely to the 1978 version (known as Kernighan \& Ritchie, or
K\&R C)\index{K\&R C} and, more increasingly, the ANSI standard recommendations,
although these are less commonly found on UNIX systems. Steve
Johnson's Portable C Compiler (PCC), the basis of many UNIX C
compilers, is also often used as a standard reference.

   Hand-in-hand with the development of UNIX and C was the development
of their underlying philosophy, dubbed the ``Software Tools'' 
\index{software tool}philosophy.
This philosophy will be discussed in more detail later.

\section{The UNIX {\cmd cc} Compiler}
\index{cc@{\cmd cc}}

     The {\cmd cc} compiler is the main compiler of the UNIX system,
and comprises  several parts.  These parts  are often  transparent to
the user,  since  the  {\cmd cc} command calls them as necessary.
Nonetheless, it is useful to know a little bit about the compiler and
its constituent phases.

\index{cc@{\cmd cc}!components of}
    First, there  is the  {\kc preprocessor\/}\index{cpp@{\cmd cpp}}, {\cmd cpp}. This is a 
simple macroprocessor, which has  a small  set of commands allowing
macros to be defined, sections of program to  be conditionally 
compiled, text  from other files to be included, and so  on. {\cmd
cpp} also strips comments \index{comments}from files. The output from {\cmd cpp} is
passed as a text stream to the first pass of the compiler proper. 
The preprocessor is described in more detail in the next chapter.

     The compiler  conceptually operates  in  two  passes,  although 
on  most systems these  actually occur  together. The  first pass
takes the output from {\cmd cpp} and  converts it  into an 
intermediate file. This file typically contains assembly language 
statements for  the higher  level  program  code  (such  as loops),
and  a symbolic  representation of  the lower level expressions in
the program. The  second pass copies the assembly language statements
to a further intermediate file, and generates assembly language for
the expressions. There is also an optional optimisation phase, {\cmd
c2}, after the second pass.

     The compiler  output is   given  to the  {\kc assembler\/},
{\cmd as}\index{as@{\cmd as}}.  This is  the resident UNIX  assembler and  varies from 
system to  system.  The  assembler converts the  assembly language 
file into machine code, and saves the results in a file using the
 UNIX common  object file  format (or  COFF)\index{COFF}. This  format
includes various  headers, symbol  table and relocation information,
and three sections -  one for  the machine  code (called {\bf
.text}\index{text@{\bf .text}}), one for the initialised data in the program (called {\bf
.data}\index{data@{\bf .data}}), and one for  uninitialised data (called 
{\bf .bss}\index{bss@{\bf .bss}}).

     The assembler  output file  is then passed to the {\kc linker\/}, 
{\cmd ld}\index{ld@{\cmd ld}}. This is the standard UNIX linker. It adds any necessary
library functions \index{libraries}to the file, and resolves references across files
if the program is split over several files.

The final executable file is always put in a file called 
{\fn a.out}\index{a.out@{\fn a.out}}, 
unless you specify otherwise. 
The  name of  the C source  file should always end with  a {\fn .c}
suffix. Two of the compilation phases  create intermediate files with
the same name as the source file, but with different suffixes, namely:
 \begin{display}
\begin{tabular}{@{}ll@{}}
          {\fn .s} &  the output from the second pass of the compiler (assembly language)\\
          {\fn .o} &  the output from the assembler (machine code)
\end{tabular}
\end{display}
\noindent
 These intermediate files are usually destroyed, unless you specify
otherwise.  For details on the compiler options available, consult
Appendix~\ref{cc}.

