\cleardoublepage
\chapter{The {\cmd vi} Visual Editor}
\index{vi@{\cmd vi}}
When you use {\cmd vi} to edit a file, it makes a copy of the file
(if it exists)  in its  own  work  buffer.  If you want to keep the
changes which  you make  during a  {\cmd vi} session,  you must  save
the work buffer contents before you leave {\cmd vi}.

   {\cmd vi} operates in two modes, {\kc insert mode\/} (everything
typed is treated as text to be  put in  the file)  and {\kc command 
mode\/} (everything  typed is treated as a command to  {\cmd vi}).
Certain  commands cause a switch from command to insert mode; to
switch from insert mode to command mode, you simply hit the \ESC\
key.

   {\cmd vi} commands  mostly consist  of letters.  The case is
important; upper case and lower case commands typically represent
different but related operations.  In the rest of this chapter, we
use the following notation:
\begin{display}
\begin{tabular}{@{}lp{.8\textwidth}@{}}
\CR & denotes the carriage return, or enter key. \\
\CTRL{\ms key\/} & denotes the {\cd Ctrl} key pressed while pressing
{\ms key}.\\
\ESC & denotes the escape key.
\end{tabular}
\end{display}
\noindent
Note: not all the commands described in this chapter may work on your
terminal exactly as described here.  The outcome of some commands are 
dependent on the terminal settings currently in use.


\section{Entering and Leaving {\cmd vi}}
\index{vi@{\cmd vi}!entering and leaving}
   To enter {\cmd vi}, you type:
\begin{display}\ms
{\cmd vi} list\_of\_filenames
\end{display}
\noindent
 where {\ms list\_of\_filenames\/} is optional.   This will  read the
first file in the list (if there is one)
into the work buffer and display the  first screenful  of lines  from
this  file in command mode.  If the file does not exist, it will be
created as an empty work buffer.

  If there  are not enough lines to fill up the screen, {\cmd vi} will
indicate the lines that  don't exist by printing  tildes \verb+~+.

     If more than one file name has been specified, you can edit  the
next file in the list by typing {\cd :n}.

     The most common way of leaving {\cmd vi} is by entering {\cd ZZ}
in command mode. This causes the  contents of the work buffer to be
saved back to the original file.     Other  commands to leave {\cmd
vi} are the following:
 \begin{display}
\begin{tabular}{@{}lp{.8\textwidth}@{}}
   {\cd :q}\CR &  quit without overwriting the file, but only if the work 
						buffer has been saved or has not been changed. \\
   {\cd :q!}\CR & quit regardless of the state of the work buffer.\\
   {\cd :wq}\CR & write the work buffer to the file and then quit.
\end{tabular}
\end{display}
\noindent
   The colon  {\cd :} indicates to {\cmd vi} that you wish to use an
extended command. When you enter  a colon,    the  cursor moves to
the bottom of the screen, allowing you to  enter the command. You 
must terminate  an extended  command  with  a \CR.  In the rest  of
this chapter we will not show the \CR\ at the end of extended
commands -- the {\cd :} at the start of the commands should be enough
to remind you that you have to terminate the command with a \CR.

   Many {\cmd vi}  commands are immediate commands.  Such a command is
not displayed and takes effect immediately  -- no \CR\ is necessary.

   The bottom  line of  the screen is a status line. It is used
for displaying error messages, extended commands, file information,
and so on.

   The rest  of the  screen acts as a  window on the file being
edited, showing you the contents of part of the file.


\section{Moving the Window and Cursor}
\index{vi@{\cmd vi}!moving around}
   The window  movement commands  are immediate  commands using
control codes:  that is,  you hold down the {\cd Ctrl} key on the
keyboard and press the appropriate letter. Basic  window
movement commands are:
\begin{display}
\begin{tabular}{@{}lp{.8\textwidth}@{}}
  \CTRL{\cd F} & move the window forward (down) by one screen        \\
  \CTRL{\cd D} & move the window forward (down) half a screen \\
  \CTRL{\cd B} & move the window backward (up) by one screen       \\
  \CTRL{\cd U} & move the window backward (up) by half a screen 
\end{tabular}
\end{display}
\noindent
     The  cursor may  be moved within the  window, with the following
cursor movement commands:
\begin{display}
\begin{tabular}{@{}lp{.7\textwidth}@{}}
  {\ms space}      &  move one character to the right \\
  {\ms backspace}  &  move one character to the left \\
  {\cd 0}      &  move to start of the current line \\
  \CR     &  move to the first non-blank character of the next line \\
  {\cd +}            &  same as \CR \\
  {\cd -}  &  move to the first non-blank character of the previous line \\
  {\cd \$}            &  move to end of current line \\
  {\ms line\/}{\cd G} &  move to start of specified line (the
									default is the last line in the file)\\
  {\cd H}            &  move to top of screen \\
  {\cd M}            &  move to middle of screen \\
  {\cd L}            &  move to bottom of screen \\
  {\cd w}            &  move to start of next word or punctuation mark \\
  {\cd W}            &  move to start of next blank delimited word \\
  {\cd b}            &  move to start of previous word or punctuation mark \\
  {\cd B}            &  move to start of previous blank delimited word \\
  {\cd e}            &  move to end of next word or punctuation mark \\
  {\cd E}            &  move to end of next blank delimited word \\
  {\cd (}            &  move to start of previous sentence \\
  {\cd )}            &  move to start of next sentence \\
 {\cd \{}            &  move to start of previous paragraph \\
 {\cd \}}            &  move to start of next paragraph \\
  {\cd k}            &  move up one line \\
  {\cd l}            &  move right one character \\
  {\cd j}            &  move down one line \\
  {\cd h}            &  move left one character
\end{tabular}
\end{display}
\noindent
     The usual  cursor control  keys can be used for moving left or
right by a single character and up or down by a single line (that is,
they are treated as if they were {\cd h}, {\cd j}, {\cd k} and {\cd l}).

\section{Moving by Context Searching}
\index{vi@{\cmd vi}!searching}
   If you enter a slash {\cd /} in command mode, the cursor moves to the
status line and you can enter a  search pattern followed by \CR.  All
lines from (but not including) the current line will then be searched
for the specified pattern and the cursor will be positioned at the
first string that matches the pattern.   If the  search reaches the
end of the file, it will ``wrap around'' (continue from the
beginning) to the starting position.  The last search pattern is
reused if no pattern is given, (i.e.\ if you just type {\cd /}\CR).

Alternatively     the   next occurrence of the last
specified search pattern can be found by using the immediate commands:
\begin{display}
\begin{tabular}{@{}ll@{}}
          {\cd n}   &  repeat last pattern search  \\
          {\cd N}   &  repeat last pattern search, but in opposite direction 
\end{tabular}
\end{display}
\noindent
     To search backwards instead of forwards,  use {\cd ?} instead
of {\cd /}. 

\subsection{Search Patterns}
\index{vi@{\cmd vi}!search patterns}
   Search patterns  consist of  strings  of  characters,  with  the 
following special notations:  
\begin{display} 
\begin{tabular}{@{}lp{.8\textwidth}@{}}
    \caret         & represents the start of a line \\
    \verb+$+         & represents the end of a line \\
    \verb+\<+        & matches the beginning of a word \\
    \verb+\>+        & matches the end of a word \\
    \verb+.+         & matches any single character \\
    \verb+*+         & matches any sequence of zero or more characters \\
    \verb+[]+        & allows character ranges to be specified \\
    \verb+\t+        & represents a tab \\
    \verb+\b+        & represents {\ms backspace} \\
    \verb+\\+        & represents \verb+\+ \\
    \verb+\+\CR & represents a newline character \\
    \verb+\^+        & represents a caret \caret \\
    \verb+\:+        & represents a colon : \\
    \verb+\$+        & represents a dollar sign \$ \\
    \verb+\[+        & represents a left square bracket {\cd [} \\
    \verb+\*+        & represents an asterisk * \\
    \verb+\/+        & represents a slash {\cd /} 
\end{tabular}
\end{display}
\noindent
   The square  brackets {\cd []}  are used  to represent a single
character within a given range. The characters can be listed
explicitly, or given as a range. For example, {\cd [abcdefg]}  and
{\cd [a-g]}  each match a single letter in the range {\cd a} to {\cd
g}, while {\cd [aeiou]} matches any single vowel.

\subsection{Matching Parentheses in Programs}
\index{vi@{\cmd vi}!matching parentheses}
     A useful command for editing programs is the  parenthesis match
command {\cd \%}.  By positioning  the cursor  on a parenthesis or
brace, and using this command, the matching  parenthesis or brace
will be temporarily highlighted (or depending on your terminal type,
the cursor may be positioned on the matching parenthesis/brace). 


\section{Repeating Commands}
\index{vi@{\cmd vi}!repeating commands}
     Some {\cmd vi}  commands may  begin with  repetition count
specifiers indicating how many times the command is to be
performed. For example:
\begin{display}\cd
 3j
\end{display}
\noindent
will move down three lines.

     To repeat the most recent command which caused a change to the
file (that is, some  form of  insert or delete command), the .
(period) immediate command can be used.


\section{Inserting Text}
\index{vi@{\cmd vi}!inserting}
     Commands for adding text include:
\begin{display}
\begin{tabular}{@{}ll@{}}
          {\cd i}        & insert text before cursor \\
          {\cd I}        & insert text at beginning of line \\
          {\cd a}        & append text after cursor \\
          {\cd A}        & append text at end of line 
\end{tabular}
\end{display}
\noindent
     These all put {\cmd vi} into insert mode. Everything that is now
typed, up until the next \ESC, will be treated as text.

 While in  insert mode,  we can  correct text  as we enter it using 
\begin{display}
\begin{tabular}{@{}lp{.8\textwidth}@{}}
{\CTRL \cd H} & delete last  character entered ( many terminals
					will allow {\ms backspace\/} to be used as well), \\
{\CTRL \cd W} & delete last word entered, and  \\
{\cd @}  & delete last line entered.
\end{tabular}
\end{display}
\noindent
 These will  only work  on the current line being entered, and on
text that has been typed since the last time insert mode was entered.

\section{Opening and Joining Lines}
\index{vi@{\cmd vi}!opening lines}
\index{vi@{\cmd vi}!joining lines}
     To open or join lines we use:
\begin{display}
\begin{tabular}{@{}ll@{}}
          {\cd o}    &     open a new line below current line \\
          {\cd O}    &     open a new line above current line \\
          {\cd J}    &     join lines 
\end{tabular}
\end{display}
\noindent
     When we open a new line, {\cmd vi} goes into insert mode and puts
us at the start of the  newly opened  line. If  we join  lines, the 
specified number of lines (default 2), beginning with the current
line, will be joined.


\section{Replacing Text}
\index{vi@{\cmd vi}!replacing text}
     To overwrite existing text, we use:
\begin{display}
\begin{tabular}{@{}lp{.8\textwidth}@{}}
  {\cd r}    &     replace the character under the cursor with the next 
                         character typed \\
  {\cd R}    &     replace the characters from cursor onwards until the
                         next \ESC
\end{tabular}
\end{display}
\noindent
     Replacing of  characters is  done on  a line  by line basis, so
we do not have to  worry if  the replacement line is longer than the
original; this will not affect the next line.


\section{Refreshing the Screen}
\index{vi@{\cmd vi}!screen refreshing}

     Depending on the terminal type, and unless you  specifically
request  it, {\cmd vi} does not immediately refresh the screen after
deleting lines. Instead, deleted lines are replaced on the screen by
{\cd @}  symbols. This  is to speed the  editor up  a bit,
particularly for slow terminals. You can, however,  explicitly
request  {\cmd vi} to refresh  the  screen display by entering \CTRL{\cd
 R} (some terminals use \CTRL {\cd  L} instead).

\section{Deleting, Moving, Copying and Changing Text}
\index{vi@{\cmd vi}!editing text}
To {\em delete\/} text, we use:
\begin{display}
\begin{tabular}{@{}ll@{}}
  {\cd x}    &     delete the character under the cursor \\
  {\cd X}    &     delete the character to the left of the cursor \\
  {\cd D}    &     delete from the cursor to the end of the line 
\end{tabular}
\end{display}
\noindent
 as well as some others mentioned later.
 
When text is deleted, it is always placed in a  buffer from where it
can be retrieved.  There is one unnamed buffer and 26 named 
buffers, {\cd "a} to {\cd "z}.  Unless  you explicitly select a named
buffer,  the unnamed buffer is used. To  use a  named buffer, the
delete,  move or  copy command  being used should begin with the
buffer name.

Deleted text can be retrieved by using:
\begin{display}
\begin{tabular}{@{}lp{.8\textwidth}@{}}
  {\cd p} & insert (put) the text from the buffer after the cursor or
				current line \\
  {\cd P} & insert the text from the buffer before the cursor or
				current line 
\end{tabular}
\end{display}
\noindent
     To  copy text to a buffer without deleting it, the yank commands
are used:  
\begin{display} 
\begin{tabular}{@{}ll@{}}
 {\cd y}   & copy text from cursor position \\
 {\cd Y}   & copy text from current line 
\end{tabular}
\end{display}

\section{Object Specifiers}
\index{vi@{\cmd vi}!object specifiers}
\label{obj_spec}
     Commands such as the delete and yank commands should be followed by an
{\kc object specifier\/} telling {\cmd vi} what  is to  be deleted  or
copied.  If the command letter is repeated, the object is the current line.
In other words, the command applies to the current line. Other object
specifiers are:
 \begin{display}
\begin{tabular}{@{}ll@{}}
          {\cd c}      &   a single character \\
          {\cd w}      &   the next word or punctuation mark \\
          {\cd W}      &   the next word \\
     {\cd /{\ms pattern\/}}  &   the next line containing {\ms pattern} \\
          {\cd (}      &   the start of the current sentence \\
          {\cd )}      &   the end of the current sentence \\
          {\cd \{}      &   the start of the current paragraph \\
          {\cd \}}      &   the end of the current paragraph \\
          {\cd H}      &   the home (top) line of the screen \\
          {\cd L}      &   the last line of the screen 
\end{tabular}
\end{display}
\noindent
     The last two cases may be prefixed by a number, specifying how
many lines below H or above L the required line is.

     The object specifiers are used with the following commands:
\begin{display}
\begin{tabular}{@{}lp{.7\textwidth}@{}}
 {\cd c} {\ms obj text\/} \ESC &  replace (change) the text from the cursor 
 										to the object {\ms obj\/} with {\ms text}. \\
 {\cd d} {\ms obj\/} &  delete text up to object {\ms obj\/}. \\
 {\cd y} {\ms obj\/} &  copy text up to {\ms obj\/} to the given buffer. \\
 {\cd !} {\ms obj command\/}   &  execute the UNIX command {\ms
			  command\/}, using the text up to {\ms obj\/} as 
			  input, and replace this text with the ouput of
			  the UNIX command.
\end{tabular}
\end{display}
\noindent
     Some examples should help to illustrate these concepts:
\begin{display}
\begin{tabular}{@{}lp{.7\textwidth}@{}}
 {\cd dw}	& deletes the next word to the unnamed buffer.\\
 {\cd dd}	& deletes the current line to the unnamed buffer. \\
 {\cd "ayw}    &  yanks (copies) the next word to the buffer {\cd "a}. \\
 {\cd y/[a-z]} &  yanks everything between (in including) the cursor 
		position and the next lower-case letter to the unnamed 
                  buffer. \\
 {\cd 5Y}      &  yanks the current and next four lines to the 
                unnamed buffer. \\
 {\cd "ap}     &  puts the contents of the {\cd "a} buffer after current 
                position.\\
 {\cd !!date}  &  replaces the current line with the UNIX date.\\
 {\cd !\}sort}  &  sorts the words to the end of the paragraph.
\end{tabular}
\end{display}
\noindent
     Many of  the object  specifiers also  allow  repetition  counts,  so 
the commands mostly have the general form:
\begin{display}\ms
$[$buffer\_name\/$]$command\/$[$count\/$]$object\/
\end{display}
\noindent
     For example, to delete the next six words to buffer {\cd f}, we use the
command {\cd "fd6w}.

\section{String Substitution}
\index{vi@{\cmd vi}!replacing text}
     We can  combine the  search and  change commands  by using the 
substitute command.   This command  requires  both a  search  and
a replacement string. The form of the command is:
\begin{display}\ms
{\cd :}$[$range\/$]${\cd s/}search\_string\/{\cd 
/}replacement\_string\/$[${\cd /g]}
\end{display}
\noindent
     The optional  {\ms range\/} specifies which lines of the file
should be searched.  The default  is the  current line,  which is
represented by a period (.). Ranges can specify  a single  line, or a
start and  finish line  separated  by commas. A line can be specified
with an explicit line number, or with an expression  involving the 
lines .  (current line)  and \$  (last line), for example:
 \begin{display}
\begin{tabular}{@{}lp{.8\textwidth}@{}}
  {\cd .,.+4}   &  the current line plus the four lines that follow it.\\
  {\cd .,\$}     &  everything from the current line to the end of the file.
\end{tabular}
\end{display}
\noindent
     The second  example can  be abbreviated by {\cd ;}. The range {\cd
1,\$} (that is, the whole file), can be abbreviated by {\cd ,}.

     The {\ms search\_string\/} is  as before, including all the
special characters.   {\ms replacement\_string\/} is  the string with
which we wish  to replace  the  search string.  The   only 
metacharacters  allowed  here are  \verb+\+  (allowing  special
characters to  be quoted),  and {\cd \&}, which represents the text
which matched the search string.

     Usually, only  the first occurrence of {\ms search\_tring\/} in
any line in the range is substituted. To substitute  every occurrence
in the range, the {\cd g}  option should  be  used  (this means {\ms
replacement\_string\/}  must  be terminated by a {\cd /} to
distinguish it from the {\cd g}). Thus, for example, to replace all
occurrences of {\cd UNIX} in a file with  {\cd the UNIX operating
system}, we can  use 
\begin{display}\cd
 1,\$s/UNIX/the \& Operating System/g
\end{display}


\section{Undoing Commands}
\index{vi@{\cmd vi}!undoing changes}
     To undo changes made, we use:
\begin{display}
\begin{tabular}{@{}lp{.8\textwidth}@{}}
          {\cd u}    &     undo the last change made \\
          {\cd U}    &     restore the current line to what it was before 
                           editing began 
\end{tabular}
\end{display}


\section{Setting {\cmd vi} Options}
\index{vi@{\cmd vi}!options}
     Many of  {\cmd vi}'s features may be controlled by the user. To
enable, disable, set or  list options,  the {\cd :set}  command is 
used.  To  list  all  the  option settings, use:
\begin{display}\cd
 :set all
\end{display}
\noindent
     To list only those options that you have changed, use:
\begin{display}\cd
 :set
\end{display}
\noindent
     To set  an option,   enter  {\cd :set} followed by the option
name.  To clear the option,   prefix  the name  with {cd no}.  The main
options (plus acceptable abbreviations) are:
\begin{display}
\begin{tabular}{@{}lll@{}}
 {\cd autoindent}  & {\cd ai} & automatic indentation of lines \\
 {\cd autowrite}   & {\cd aw} & automatically save file before exit \\
 {\cd ignorecase}  & {\cd ic} & ignore upper/lower case in searches    \\
 {\cd list}        &          & print tabs as {\cd \verb+^+I} and end of 
 										  line as {\cd \$} \\
 {\cd number}      & {\cd nu} & display line numbers \\
 {\cd showmatch}   & {\cd sm} & show matching brackets as the are typed, by
			flashing the cursor \\
 {\cd wrapmargin} & {\cd wm} & enable automatic line splitting in
				the given zone
\end{tabular}
\end{display}
\noindent
     Examples of usage are:
\begin{display}
\begin{tabular}{@{}ll@{}}
    {\cd :set nolist}   &      don't print tabs and line feeds \\
    {\cd :set nu}       &      display line numbers 
\end{tabular}
\end{display}
\noindent
     The autoindent  option makes  {\cmd vi} indent each line to the
same left margin as the  previous line.  To ``unindent''  a line  by
one indentation, use \CTRL{\cd  D}\@.  {\cd showmatch} makes  {\cmd vi}
briefly highlight matching left brackets as you type right brackets,
provided  both are  on the  same screen.  If there  is  no  matching
bracket, {\cmd vi} beeps to indicate a possible error.


\section{Defining and Using Key Macros}
\index{vi@{\cmd vi}!macros}

     It is  possible to  define key  macros for  commonly used
sequences of {\cmd vi} commands. This  is done  with the  {\cd :map}
command.  Special  characters  can  be entered into  macros by  using
a  \CTRL{\cd  V} prefix.  For example,  suppose  we wish to enter a
number of lines of the form:
\begin{display}\cd
table[21]="example";
\end{display}
\noindent
 where only  the table  index and string contents vary. To do this
fast, we could define two macros as follows:
\begin{display}\cd
 :map t A\CTRL V\CR table[ \\
 :map s A]="";\CTRL V\ESC hi
\end{display}
\noindent
 The first line assigns a macro to the key {\cd t}.  The macro starts
a new line (by appending a carriage return to the current line) and
then inserts {\cd table[}, leaving {\cmd vi} in insert mode which will
enable us to type in the desired index.  The second line defines a
macro for the {\cd s} key.  It appends {\cd ]="";} to the current
line, goes into edit mode, positions the cursor on the second quote,
and puts {\cmd vi} in insert mode, ready for us to type in the string. 
     To remove a macro definition,  use {\cd :unmap} (for example,
{\cd :unmap t}).


\section{Reading and Writing Files}
\index{vi@{\cmd vi}!file operations}
There are ways
of reading and writing text into and from the work buffer without
exiting {cmd vi}.

     The read command has the format:
\begin{display}\cd
:r {\ms filename\/}
\end{display}
\noindent
     This reads  the specified  file and inserts it after the current
line. To read a  file into the work buffer and replace the contents
of the work buffer, the edit command should be used:
\begin{display}\cd
:e {\ms filename\/}
\end{display}
\noindent
     To write out text from the work buffer, the write command
can be used.  This has the form:
\begin{display}\cd
:$[${\ms range\/}$]$w$[${\ms option\/}$][${\ms filename\/}$]$
\end{display}
\noindent
     where the optional {\ms range\/} is  the same  as for  the 
substitute command, and the optional {\ms option\/} can be one of {cd
!} or {\cd >>}.   If no file name is specified, the work buffer is
written to file currently being edited.

If  a range is given, only the specified portion of the work buffer
will be written to the given file.  If no range is  specified, the 
entire buffer is written. {\cmd vi} will not overwrite an existing
file  if only  part of the buffer is being written.  To do this, you 
must use  the overwrite  option {\cd !}.  To append  to a file rather
than overwriting it, the {\cd >>} option must be used.

The {\cd :e} command can also be followed by an exclamation  mark.
This  forces {\cmd vi} to obey the command even if changes made to the
work buffer have   not been saved.  Be  careful about using {\cd !}; 
always  try  to perform a command without it first.

\section{Advanced Search and Replace}
\index{vi@{\cmd vi}!replacing text}
The search and replace facility of {\cmd vi} is much more powerful
than we indicated previously. Much of this power comes from being
able to group regular expressions in the search string, and then use
these groups as part of the replacement string.

The grouping facility is done using {\cd \verb+\+(} and {\cd
\verb+\+)}. Within the replacement string, the groups are referred to
as {\cd \verb+\+1}, {\cd \verb+\+2}, etc, in sequence.

Additional special characters in the replacement string are:
\begin{display}
\begin{tabular}{@{}lp{.8\textwidth}@{}}
\cd \verb+\+u &  convert the next character to upper case \\
\verb+\+l &  convert the next character to lower case \\
\verb+\+U &  turns on upper case conversion until \verb+\+e, 
					\verb+\+E or the end of the replacement string is
					encountered.\\
\verb+\+L & as for \verb+\+U, but for lower case.
\end{tabular}
\end{display}
\noindent
 Here are some examples to clarify these concepts. To convert all
occurences of ``up and over'' to ``over and up'', we can use:
\begin{code}
:s/\verb+\+(up\verb+\+)\verb+\+( and \verb+\+)\verb+\+(over\verb+\+)/%
\verb+\+3\verb+\+2\verb+\+1/g
\end{code}
\noindent
or, equivalently:
\begin{code}
:s/\verb+\+(up\verb+\+) and \verb+\+(over\verb+\+)/\verb+\+2 and
\verb+\+1/g
\end{code}
\noindent
 To convert all occurences of the words ``unix'' or ``Unix'' to
``UNIX'', we can use:
\begin{code}
:s/[uU]nix/\verb+\+U\&/g
\end{code}
\noindent
 (Recall that {\cd \&} corresponds to the matched text). Similarly, to
convert all occurences of the word ``multics'' to ``Multics'', we can
use:
\begin{code}
:s/multics/\verb+\+u\&/g
\end{code}
\noindent
 While these examples are trivial, and could as easily have be done using
simple patterns, they illustrate the concepts. Real examples will
appear when you are programming, and the true value of case
conversion and regular expression grouping will then become apparent.

\section{Miscellaneous}
\index{vi@{\cmd vi}!running UNIX commands from}
     UNIX commands  can be  sent to  the shell while in {\cmd vi}. To
do this, prefix the UNIX command  with an  exclamation mark.  For example,
to find out the time, enter {\cd !date}\CR.  See also
section~\ref{obj_spec} on
Object Specifiers

     To find  out the name and size of the file we are editing, the
{\cd :f} or or  \CTRL{\cd G} command can be used.


\section{Recovering from Crashes}
\index{vi@{\cmd vi}!recovering from UNIX crashes}
     {\cmd vi} continually saves the contents of the editing buffer in
a file. If the system crashes,  you can  recover your file up to
the last change made. {\cmd vi} normally sends you mail indicating
that a recovery file is available. When you invoke {\cmd vi} the next
time with that file name, you must invoke it with
\begin{display}\cmd
 vi -r {\ms filename\/}
\end{display}
\noindent
 after which  you can  continue from  where you left off.  If you
invoke {\cmd vi} without this option, the recovery file will be lost.


\section{Installing {\cmd vi} Options Permanently}
\index{vi@{\cmd vi}!options}
     If you  use certain  {\cd :set} options in {\cmd vi} every time you
use the editor, you can have these options automatically set for you.
To do this you must edit (or create if  necessary) the file {\fn .profile}
in your login directory and add the lines:
\begin{display}\cd
 EXINIT='set {\ms option\_list\/}' \\
 export EXINIT
\end{display}
\noindent
     Every time  you execute  {\cmd vi},  the  shell  will  send  it 
these  settings automatically.

