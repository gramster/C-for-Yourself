\cleardoublepage
\chapter{Useful UNIX Tools}

UNIX  provides us  with several tools to aid program development.
These  include
\begin{display}
\begin{tabular}{@{}ll@{}}
{\cmd lint} & a  program checker \\
{\cmd make, SCCS} &   program maintenance utilities \\
{\cmd sdb, dbx} & interactive symbolic debuggers \\
{\cmd ctrace} & a C trace utility \\
{\cmd cxref} & a cross-reference generator \\
{\cmd cb} & a  program ``beautifier'' \\
{\cmd make, SCCS} &   program maintenance utilities \\
{\cmd ar} & an archive (library) maintainer \\
{\cmd size, nm} & provide information on compiled files \\
{\cmd prof} & an execution profiler \\
{\cmd time} & time the execution of programs
\end{tabular}
\end{display}
\noindent
 We will examine some of these in the sections that follow.


\section{{\cmd lint} -- The C Program Checker }
\index{lint@{\cmd lint}}
     {\cmd lint} checks  C programs  for features  which are  likely
to be bugs, non-portable, or  wasteful. It  provides stricter  type
checking  than {\cmd cc}, and can find unreachable  statements, loops
not entered  at the top, constant logical expressions, auto variables
which  are declared  but  never  used,  functions called with varying
numbers  of arguments,  and functions returning values that are
not always (if ever) used.

     {\cmd lint} is invoked by:
\begin{display}\cmd
lint $[${\ms options\/}$]$ {\ms files\/}
\end{display}
\noindent
     {\ms files\/} is a list of files that are assumed to contain  a
single  program,  and  are checked for mutual compatibility. Note
that ``lint'' is a reserved word in {\cmd lint}, so the files should 
not use  ``lint'' as  an identifier. 

There is a large number of {\cmd lint} options, of which we will
examine only a few.   Some of the main options are:
\begin{display}
\begin{tabular}{@{}lp{.7\textwidth}@{}}
{\cmd -a} & Report assignments of longs to ints. \\
{\cmd -b} & Report break statements that are unreachable. \\
{\cmd -c} & Complain about non-portable type casts. \\
{\cmd -h} & Don't attempt to guess at likely mistakes. \\
{\cmd -n} & Attempt to check portability to IBM and GCOS C. \\
{\cmd -p} & Report possible portability problems. \\
 {\cmd -u} & Do  not complain  about functions  and variables  that are
defined and never used  or vice-versa  (useful for checking files
containing only part of a program). \\
{\cmd -v} & Do not complain about unused function arguments. \\
{\cmd -x} & Report all {\cd extern} declarations whose variables are never
used.
\end{tabular}
\end{display}
\noindent
     The {\cd exit}  system call,  and other  functions that  never return,
are not properly handled by {\cmd lint}.

     {\cmd lint} allows  you to  use comments  within your files that
will modify its behaviour from  that point.  Although their  effects
are simple (for  example,  shutting  off strict  type   checking  in
expressions), these are  likely to  be used only by experienced {\cmd
lint} users.  Refer  to  the  UNIX Programmer's Reference  Manual
volume  1 section  1 for  details, if  you  are interested in using
these.


\section{{\cmd make} -- A Program Maintenance Utility.}
\index{make@{\cmd make}}
Large C programs are usually split over numerous files. Sometimes,
some of the files are compiled and put into libraries, which are then
linked with the other files. Numerous header files can also exist,
with each C file {\cd \#include}-ing different sets of header files.
Obviously, if a file is changed, the program must be recompiled. This
may entail remaking libraries if the changed file is a library
component, or recompiling several C files if the changed file is a
header file. Keeping track of all the {\kc dependencies\/} between
files can become complex, and recompiling can be tedious. In an
earlier section we saw how we could use shell scripts to control
compilation, but these may not be adequate beyond a certain level of
complexity. {\cmd make} is a tool which handles program maintenance
automatically. It keeps track of file dependencies, and only performs
those actions that are necessary to keep a program (or group of
programs) up to date.

In order to tell {\cmd make} what to do, a {\kc makefile} must be
created. This file may have any name, but {\cmd make} will
automatically use a default file if none is explicitly given. The
allowed default filenames are {\fn makefile}, {\fn Makefile}, {\fn
s.makefile} or {\fn s.Makefile}. The makefile specifies the
dependencies between files, and the actions to be taken if a {\kc
target\/} is older than one of the files upon which it depends.

Each entry in the makefile is separated from others by one or more
blank lines, and consists of a line specifying the target and its
dependencies, followed by zero or more lines specifying the actions.
Each of the action lines should begin with a {\sc tab} character.
Consider a simple example. Say we have a program {\cd f} made up of files
{\cd f1.c}, {\cd bf f2.c} and {\cd f3.c}, each of which uses the
include file {\cd f.h}. The file {\cd f1.c} also uses an include file
{\cd config.h}. A sample makefile for this situation would be:
\begin{code}
f: f1.o f2.o f3.o	 \tab	\# Target and dependencies \\
\>	cc -o f f1.o f2.o f3.o \tab \# Action			  \addVspace

f1.o: f1.c f.h config.h								  \\
\>	cc -c f1.c												  \addVspace

f2.o: f2.c f.h											  \\
\>	cc -c f1.c												  \addVspace

f3.o: f3.c f.h											  \\
\>	cc -c f3.c												  
\end{code}
\noindent
 Notice the use of {\cd \#} to specify comments. All text from a {\cd
\#} to the end of that line is ignored by {\em make}.

When we run {\em make}, it will examine the makefile. If any of the
dependent files are newer than the targets, the actions associated with
them will be performed. For example, if {\cd config.h} has just
been edited, it is more recent than the target {\cd f1.o}. {\cd f1.c}
will thus be recompiled to create a new, up-to-date version of {\cd
f1.o}. This will then make the program {\cd f} out of date, so it
will also be recompiled. After this, all the targets will be more
recent than their dependencies, so {\em make} will exit.

If a dependency file is not found, {\em make} will try to create it.
It does this by looking for an entry which has that file as the
target. If no such entry is found, {\em make} will report that it
cannot find the file, and does not know how to make it.

Make has a set of internal default rules that it will use if no
explicit rule is given. Some of these are listed below:
\begin{display}\cd
\begin{tabular}{@{}lll@{}}
\rm\bf Target & \rm\bf Dependent on	& \rm\bf Rule \\
x.o  &	x.c  &	cc -c x.c \\
x.o  &	x.s  &	as x.s	 \\
x.a  &	x.c  &	cc -c x.c;ar rv x.a x.o;rm -f x.o 
\end{tabular}
\end{display}
\noindent
 {\em make} has an elementary macro facility. A macro can be defined by:
\begin{code}
macroname = string
\end{code}
\noindent
A macro can then be used by:
\begin{code}
\$(macroname)
\end{code}
\noindent
 Macros in {\em make} are mostly used for altering compiler options.
For example, we may want to be able to turn symbolic debugging on or off
during compilations at different stages. Instead of changing each
action line, we can use a macro. Our makefile will now look like:
\begin{code}
CFLAGS = -g -c												\addVspace

f: f1.o f2.o f3.o		\# Target and dependencies	\\
\>	cc -o f f1.o f2.o f3.o  \# Action				\addVspace

f1.o: f1.c f.h config.h									\\
\>	cc \$(CFLAGS) f1.c									\addVspace

f2.o: f2.c f.h												\\
	cc \$(CFLAGS) f1.c									\addVspace

f3.o: f3.c f.h												\\
	cc \$(CFLAGS) f3.c
\end{code}
\noindent
If we wish to disable symbolic debugging, we now need only to change the
macro definition.

{\em make} has several command-line options that can be specified.
These include:
\begin{display}
\begin{tabular}{@{}ll@{}}
\cmd -s &	Don't print command lines before executing them (silent
				mode). \\
\cmd -n &	Print commands to be performed; don't actually do them.\\
\cmd -r &	Do not use the built in rules. \\
\cmd -f{\ms file} &	Use {\em file\/} as the makefile
\end{tabular}
\end{display}

\section{{\cmd SCCS} -- The Source Code Control System}
\index{SCCS@{\cmd SCCS}}
When systems are being developed, it is often useful to keep
old versions of programs, for reference and also occasionally to
recover a working version of a file when a line of development proves
to be unsuccessful. The Source Code Control System (SCCS) is a
collection of programs which allows multiple versions of programs to
be stored in a space efficient way (rather than storing the programs,
only the most recent versions are stored, together with the sets of
changes that must be made to restore them to older versions), and
to be retrieved upon command. SCCS is a sophisticated system and a
description of its use is beyond the scope of this book. Interested
readers are referred to the UNIX manuals for details. A good
introduction to SCCS (as well as many other UNIX utilities), is the
book {\em UNIX Utilities -- A Programmers Reference}, by R S Tare
(McGraw-Hill 1987).

\section{{\cmd sdb} -- A Symbolic Debugger}
\index{debugging!sdb@{\cmd sdb}}
\index{sdb@{\cmd sdb}}
     {\cmd sdb} is  a powerful  symbolic debugger for C programs. It
makes use of the program source  files, thus  it is possible to see
at any time which C statement is being processed. In order to  use
{\cmd sdb} to debug a program, the program must have been compiled
with the {\cmd -g} option of {\cmd cc.}

     To invoke {\cmd sdb}, we use:
\begin{display}\cmd
sdb $[${\ms object\_file\/} $[${\ms core\_file\/} $[${\ms directory\_list\/}$]]]$
\end{display}
\noindent
where
\begin{itemize}
\item[]  {\ms object\_file\/} is the name of the compiled program (the
default is {\fn a.out}).
\item[]  {\ms core\_file\/}  is an optional core dump of the program.
This is a record of the state  of the  program at  some particular
point during execution. We can generate a  core dump  at any  time
while executing a program by typing \CTRL \verb+\+ (this also  causes
the program's execution to be terminated). Alternatively, if a program
crashed  during execution,  a core  dump is automatically generated
by UNIX. The default name for core files is {\fn core}.
\item[]   {\ms directory\_list\/} tells  {\cmd sdb} where  to look  for the
program  source files. If  more than  one directory is given, they
must be separated by colons (:).
\end{itemize}
 {\cmd sdb} is terminated with the command {\cd q}.

{\cmd sdb } maintains a current file name and current
line count. If  {\ms core\_file} is  present, the current file is set to
the core file, and the current line is set to the line  where the
program stopped.  Thus, if  a program crashed, one can immediately
find out where it did so.

     Variable names in {\cmd sdb} are the same as in the source
programs, except that variables  local   to  a   function   must  be
written   using   the   form {\ms function\/}:{\ms variable\/}. If no
function  name is given, the current function is used. Array and
structure elements can be accessed as usual, using {\cd []}, . and
{\cd ->}.  One can also specify variables by their addresses.

     Lines in  a  source  program can  be referred  to as  {\ms
file\/}:{\ms number\/} or {\ms function\/}.{\ms number\/}. In  either
case, {\ms number\/} is relative to the beginning of  {\ms
file\/},  or the  file containing  {\ms function\/}. If no {\ms
number\/} is given, the first line of the {\ms file\/} or {\ms
function\/} is used.

     {\cmd sdb} provides  a large  set of  commands, of which we shall
examine only a few. Firstly, commands to examine program source
files:
\begin{hanglist}
\hangitem{\cd e {\ms function\/} \\
           e {\ms file\/}  }
 Sets the  current file  to be  the specified {\ms file\/}, or the
file containing  {\ms function\/},  and sets the current line
to be the first line in the {\ms function\/} or {\ms file\/}.

\hangitem{\cd /{\ms pattern\/} }
 Search from  the current line  for a  line containing a match of
{\ms pattern}.

\item[\cd ?{\ms pattern\/}]
 Search backwards for {\ms pattern}.

\item[\cd p]
 Print the current line.

\item[\cd z]
 Advance and print the next ten lines.

\item[\cd w]
 Print the ten lines around the current line.

\item[\ms number\/]
 Set the current line to {\ms number\/}.

\item[\cd {\ms count\/}+]
 Advance by {\ms count\/} lines, and print the new current line.

\item[\cd {\ms count\/}-]
 Retreat by {\ms count\/} lines, and print the new current line.

\end{hanglist}
     The following commands allow us to examine data from the program:
\begin{hanglist}
\item[\cd t] Print a function stack trace.

\item[\cd T] Print the top line of the stack trace.

\hangitem{\cd {\ms variable\/}/lf}
 Print the  value of {\ms variable\/} according to the length {\cd l}
and  format {\cd f}.  The length  and format may be omitted, in which
case {\cmd sdb} selects a length and format according to the variable's
declaration.  The length specifiers are
\begin{itemize}
\item[{\cd b}] one byte
\item[{\cd h}] two bytes
\item[{\cd l}] four bytes
\item[{\ms number}]  length of a string
\end{itemize}
The  formats are
\begin{itemize}
\item[{\cd c}] character
\item[{\cd d}]  decimal
\item[{\cd u}] unsigned decimal
\item[{\cd o}] octal
\item[{\cd x}] hexadecimal
\item[{\cd f}] 32-bit floating point
\item[{\cd g}] 64-bit  floating point
\item[{\cd s}] string -- print characters  from address  contained in variable,
\item[{\cd a}] string -- print characters  from variable's  address
\item[{\cd p}] pointer to function
\end{itemize}

\hangitem{\cd {\ms variable\/}=lf\\
              {\ms linenumber\/}=lf\\
              {\ms number\/}=lf  } 
 Print the  address of  {\ms variable\/}, {\ms line
 number\/} or {\ms number\/} in the specified format.  {\cd l} and
 {\cd f} are as before.

\hangitem{\cd {\ms variable\/}!{\ms value\/} }
 Set {\ms variable\/} equal to {\ms value\/}

\end{hanglist}
A  program can be executed under  {\cmd sdb}'s control.  {\cmd
sdb} allows one to set breakpoints at  any point  in  the  program. 
When  an active  breakpoint  is encountered, {\cmd sdb } reports
that  it  has reached  the  breakpoint,  and  stops executing the
program. One can then use {\cmd sdb} freely, and continue executing
the program if  one wishes.  The following commands allow   control
of the execution of a program:
\begin{hanglist}
\hangitem{\cd {\ms count\/} R \\
   {\ms count\/} r {\ms args\/}}      
	Run the program (with the arguments {\ms args\/} if {\cd r}).
   After using  {\cd r} once, one can omit  {\ms args\/}, in which case
   the previous  {\ms args\/}  will  be  used  again.  The  optional
   {\ms count\/} specifies  the number  of breakpoints to be skipped
   over before making all breakpoints active.

\hangitem{\cd {\ms line\/} C {\ms count\/}\\
   {\ms line\/} c {\ms count\/} }
	Continue  execution   after   a   breakpoint   or
   interrupt. {\ms count\/}  is as before. {\cd c} continues, ignoring the
   signal that caused the program to stop. If {\ms line\/} is given,
   a temporary  breakpoint is  set at  that line,  and deleted
   when the command finishes.

\hangitem{\cd {\ms count\/} s \\
   {\ms count\/} S }
	Run the  program for  {\ms count\/} lines  (the default is one).  
	If {\cd S}, single step through called functions as well.

\hangitem{\cd {\ms function\/}({\ms args\/}) \\
   {\ms function\/}({\ms args\/})/f }
	Call   {\ms function\/}  with  {\ms args\/}.  If  the
   second form  is used,  the returned  value is  printed with
   format {\cd f}. The default format is {\cd d}.

\hangitem{\cd  {\ms line\/} b {\ms requests\/} }
   Set a  breakpoint at  {\ms line\/} (the  default is the
   current line).  If {\ms requests\/}  are  given,  then  when  the
   breakpoint is  encountered the  {\ms requests\/} are executed and
   the program  continues; otherwise  control returns  to  {\cmd sdb}
   when the  breakpoint is  encountered. Multiple requests can
   be separated by {\cd ;}.

\hangitem{\cd {\ms line\/} d}
   Delete the breakpoint at {\ms line\/}. If no {\ms line\/} is given,
   {\cmd sdb} prints each  breakpoint and  asks confirmation before
   deleting it.

\item[\cd B]   Print the current breakpoints.

\item[\cd D]   Delete all breakpoints.

\item[\cd I] Print the last line executed.

\hangitem{\cd {\ms line\/} a } 
 Announce. If  {\ms line\/} is  of the form  {\ms function\/}:, this 
is the same as  {\cd {\ms function\/}:\ b T}  else  if  {\ms line\/}  is 
of  form {\ms function\/}:{\ms line\/}, it  is the  same as {\cd {\ms
line\/}b I} The effect is to  print the  line or  function name  every
time  it is executed.
\end{hanglist}
 As in other UNIX utilitites, commands can be sent to the UNIX shell
by preceding them with {\cd !}.

\section{{\cmd dbx} -- A Symbolic Debugger}
\index{debugging!dbx@{\cmd dbx}}
\index{dbx@{\cmd dbx}}
{\cmd dbx} is a symbolic debugger that is used in BSD versions of UNIX, as
well as in IBM's AIX system. It is very user-friendly and easy to
use.

Like {\cmd sdb}, {\cmd dbx} requires you to compile your C program with
the {\bf -g} option of {\bf cc}. It will also make use of a
core dump file if there is one (see the section on {\cmd sdb} for a
description of what a core dump file is). Recall that you can force a
core dump at any stage with \CTRL \verb+\+.

{\cmd dbx} takes two optional arguments, the first being the name of
the file to be debugged, and the second is the name of the core dump
file. The defaults are {\fn a.out} and {\fn core} respectively. After
starting {\cmd dbx}, it will print a message, and then prompt you
with the prompt {\cd (dbx)}. You can type {\cd help} to get a screen
full of help, summarising the possible commands you can use. Some
(but by no means all) of these commands are described in the
following paragraphs.

When {\cmd dbx} starts, it looks for a file called {\fn .dbxinit}, in
the current directory, and if not found, in the user's home
directory. If it finds this file, it runs the commands in it. This
file can be used to customise {\cmd dbx} to an extent, as well as to
repeat a set of commands. Furthermore, the option {\cmd -c}{\ms
file\/} can be used to specify a file containing commands to be
executed before reading from the keyboard.

The following operators are all valid in {\cmd dbx} expressions:
\begin{display}
\begin{tabular}{@{}ll@{}}
Algebraic: & \cd +  -  *  / {\rm (floating)}  div {\rm (integral)}  
		mod  exp {\rm (exponentiation)} \\
Bitwise:   & \cd |  bitand  xor  \verb+~+  <<  >> \\
Logical:   & \cd or  and  not \\
Comparison:& \cd >  <  <=  >=  !=  == \\
Other:	  & \cd sizeof	
\end{tabular}
\end{display}
\noindent
 The command {\cd where} will print a {\kc stack trace}, showing you
where in your program you are. The stack trace lists, in reverse
order, the different functions that are currently active, including
the parameters that were passed to them. The last function listed is
always the first one called, namely {\cd main}. This command is
particularly useful after a UNIX-instigated core dump, as it tells
you where your program crashed.

The command {\cd call {\ms func\/}({\ms params\/})} executes the
function {\ms func\/} with the specified arguments. Alternatively,
{\cd print {\ms func\/}({\ms params\/})} will do similarly, and
also print the return value.

{\cd next} {\ms $[$num\/$]$} will execute the next line of the program (or
{\ms num\/} lines, if {\ms num\/} is specified. A similar command is {\cd
step}. The difference is that if the line(s) to be executed contain a
function call, {\cd step} will stop at that call, whereas {\cd next}
will execute it.

The {\cd run} command will execute the program. It can be followed by
command line arguments for the program, as well as redirection
commands for {\fn stdin} and {\fn stdout}. In other words, a command
line can be specified as usual, except that the name of the program
is omitted.  For example:
 \begin{code}
run -x <data >list
\end{code}
\noindent
 Values can be assigned to variables with the {\cd assign {\ms var\/} =
{\ms expr\/}} command, and the values of expressions can be printed with the
{\cd print }{\ms expr\/$[$,expr...$]$} command.

The {\cd trace} command lets you track variables as their values
change. You must specify which variable to trace. When you then run
the program, each time a traced variable is altered its value is
printed, together with the place in the program where the value was
changed. The {\cd trace} command is extremely flexible and powerful.
A similar command is {\cd stop}. For full details, see the UNIX
manuals.

To examine the program, the {\cd list} command is used. You can
specify the name of a function, or a range of lines to be listed,
for example, {\cd list 10,20}.

Finally, to exit {\cmd dbx}, we use the {\cd quit} command.

This is a very cursory description of {\cmd dbx}'s capabilities, but
it includes sufficient information to get started. Once you have the
feel for these commands, look at the {\cmd dbx} entry in the UNIX
manuals for full details of the entire {\cmd dbx} command set.

\section{{\cmd strip} -- Strip Symbolic Information}
\index{strip@{\cmd strip}}
\index{debugging!stripping symbolic information}
When a program is compiled with the {\cmd -g} option as required by
{\cmd sdb} and {\cmd dbx}, the compiler adds symbolic information to
the executable file that aids the symbolic debugger.  This
information takes up extra space which is not required once the
program is debugged.  UNIX  provides a  utility, {\cmd strip}, which
removes the symbolic information from the file without having to
recompile it.  {\cmd strip} allows two options to control what
information is stripped:
\begin{display}
\begin{tabular}{@{}ll@{}}
{\cmd -l} & Strip line number information only.\\
{\cmd -x} & Don't strip static and external symbol information.
\end{tabular}
\end{display}
\noindent
     The default action of {\cmd strip} is to remove all symbolic
information.

\section{{\cmd ctrace} -- The C Trace Utility}
\index{debugging!tracing program execution}
\index{trace@{\cmd ctrace}}
{\cmd ctrace} is  a utility that inserts statements into a  C program
to print each statement  before execution,  as well  as to print the 
values  of  any variables referenced or modified. {\cmd ctrace} is a
filter; however, the {\cmd cc} compiler is not,  therefore you  must
send the output  of {\cmd ctrace } to a  temporary file before you
can compile it with {\cmd cc}.  In other words:
\begin{display}\cmd
 ctrace {\cd <}myprog.c {\cd |} cc
\end{display}
\noindent
     is not allowed. Instead you must use something like
\begin{display}\cmd
          ctrace  {\cd <}myprog.c {\cd >}mytmp.c\\
          cc mytmp.c\\
          rm mytmp.c
\end{display}
\noindent
     {\cmd ctrace} allows a file argument to be specified, as in:
\begin{display}\cmd
          ctrace myprog.c {\cd >}mytmp.c
\end{display}
\noindent
     {\cmd ctrace} detects  loops and  stops the  trace until  the
loop  is exited. A warning message  is printed  every 1000  times
through the loop. The output of {\cmd ctrace} always  goes to the standard
output. A typical use would be to pipe the output of  a crashing 
program to the UNIX {\cmd tail} command, and thus see the last few
statements executed, as well as their effect.

{\cmd ctrace} has the following options:
\begin{hanglist}
\hangitem{\cmd -f  {\ms function\_list\/}  \\
			-v {\ms function list\/} }
       Select only ({\cmd -f})  or all except ({\cmd -v}) the  
functions named in {\ms function\_list\/} for  tracing. 

\hangitem{\cmd -o -x -u -e}
 Change the numeric formats from their  defaults to octal ({\cmd
-o}), hexadecimal {\cmd -x}) unsigned ({\cmd -u}), or floating point
({\cmd -e}).

\hangitem{\cmd -l  {\ms number\/}}
  Specify  the  number  of  consecutive statements that are checked 
to detect  loops (default 20). In other words, loops are assumed to
be no more than {\ms number\/} statements long. If you want every
statement traced (that is, no loop checking), use the value 0.

\hangitem{\cmd -t  {\ms number\/}}
  Specify the maximum number of variables that will be traced per
statement. The default is 10, and the maximum is 20.

\item[\cmd -P ]
 Run the preprocessor  on the  file before  adding  trace statements.
This  is useful  if the  file {\cd \#include}s some other file which must
also be  traced, or  if you  have defined macros which are expressions
and you want them traced as well.

\hangitem{\cmd -p '{\ms string\/}'}
   Change the default ({\cd 'printf('}) trace statement. This option
makes it possible to print the trace to other files.  For example, to  print 
all  trace statements to stderr,  one could  use the option  {\cd 
-p 'fprintf(stderr,'}.
\end{hanglist}
     We can control the trace from within our program. There are two
functions provided by  {\cmd ctrace} to turn tracing off and on.
These are {\cd ctroff()} and {\cd ctron()} respectively. {\cmd
ctrace} also defines  the symbol {\cd CTRACE}, thus  we  can  have 
conditionally compiled statements  controlling the  trace in an
arbitrarily complex way. For example:
\begin{code}
   \#ifdef CTRACE    \\
\>   ctroff();		  \\
   \#endif			  \\
	\vdots			  \\
   \#ifdef CTRACE	  \\
\>   if( feof(outfile) \&\& (n<100) ) ctron(); \\
   \#endif
\end{code}
\noindent
     We can  also use  these functions  from within {\cmd sdb} if we 
compile the output of {\cmd ctrace} with the {\cmd -g} option of
{\cmd cc}. For example:
\begin{display}\cd
          main:1b ctroff()\\
          main:10b ctron()\\
          r
\end{display}
\noindent
  in {\cmd sdb} will trace all but the first 10 lines of {\cd main}.

     Note that {\cmd ctrace} can't handle printing components of
aggregate data types such as structures, unions and arrays.

\section{{\cmd cxref} -- The C Program Cross-Reference Generator}
\index{cxref@{\cmd cxref}}
\index{identifiers!cross-referencing with {\cmd cxref}}
     This utility  analyses a  collection of  files and  attempts to 
build  a cross-reference for  them. This  cross-reference can contain
all  identifiers (including preprocessor  names),  either separately 
for  each  file,  or  in combination. The identifiers are listed
together with all their occurrences. Defining references are marked
with an asterisk {\cd *}.

     The syntax of the command is:
\begin{display}\cmd
          cxref  $[${\ms options\/}$]$  {\ms files}
\end{display}
\noindent
     {\cmd cxref} allows the {\cmd -D}, {\cmd -I}, and {\cmd -U}
options of {\cmd cc}, as well as:
\begin{display}
\begin{tabular}{@{}ll@{}}
{\cmd -c}  &      Produce a combined cross-reference for all files.\\
{\cmd -o} {\ms file\/} & Send output to the named file. \\
{\cmd -s}   &     Don't print the input file names.\\
{\cmd -t}   &    Format output for 80-column device.
\end{tabular}
\end{display}


\section{{\cmd cb} -- The C Program Beautifier}
\index{style!cb@{\cmd cb}}
\index{cb@{\cmd cb}}
     {\cmd cb} is  a filter   which takes a C program (either from its
argument list, or stdin),  and writes  it on stdout with appropriate
spacing and indentation.  Note that  punctuation which  is hidden  in
preprocessor  statements may cause indentation errors. The available
options are:
\begin{display}
\begin{tabular}{@{}ll@{}}
  {\cmd   -s} & Output code in Kernighan and Ritchie style. \\
  {\cmd   -j} & Rejoin split lines if necessary.\\
  {\cmd   -l}{\ms length\/} &   Split lines longer than {\ms length}.
\end{tabular}
\end{display}




\section{Building Your Own Libraries}
\index{libraries!creating your own}
     The advantage of writing small, generalised functions is that
they may be easily reused.  If you  write a  program, and  this
program  has, say,  a  few functions which  are of  general use  and
not specific to the program, you can separate these functions from
the program and put them in your own library for future use.

     A library (or {\kc archive}) is a single file which contains
other files. There are no restrictions on the contents of an archive;
within a particular archive we may keep documents, program listings,
compiled programs, and so on. Archive file names have the convention
that they begin with {\fn lib} and end with {\fn .a}.

     When you  compile a  program, you  can specify  the name of an
archive or archives for  {\cmd ld} to  search. {\cmd ld}  will extract
any files from the archive that resolve external references in the
program being compiled, and add them to the program. Note  that
library  functions stored in an archive must  have been compiled, but
not linked.  In  other words,  they must have been compiled with the
{\cmd -c} option of {\cmd cc}.

     The UNIX {\cmd ar} utility is used for archive maintenance. To
create an archive (say {\cmd libme.a}) containing files {\cmd fun1.o},
{\cmd fun2.o}, and {\cmd fun3.o}, we use:
\begin{display}\cmd
ar  rv  libme.a  fun1.o  fun2.o  fun3.o
\end{display}
\noindent
     To extract  {\cmd fun1.o} and {\cmd fun3.o} and place them in the
current directory, we use:
\begin{display}\cmd
ar  xv  libme.a  fun1.o  fun3.o
\end{display}
\noindent
     If we  use the {\cmd ar rv} command shown above after having
already created the archive, we  will simply add the  files {\cmd
fun1.o},  {\cmd fun2.o} and {\cmd fun3.o} to the archive.  If  there
are  already files with that name  in the archive, they will be
replaced.

     To examine the files in the archive,  use:
\begin{display}\cmd
ar  tv  libme.a
\end{display}
\noindent
     The  only  problem  with  these  archives  is  that  they  are
searched sequentially by  {\cmd ld}, so  functions which  call other
functions in the archive must precede  those functions in the
archive.  This allows {\cmd ld} to detect an unresolved reference in
a single pass, because when it  reaches the  called  function  it
can  resolve  the reference. If  the called  function preceded the
caller, and no other function called this  function, then  {\cmd ld}
will  skip the function since it does not resolve any references at
that point.  When {\cmd ld} eventually reaches the  caller, an
unresolved reference  will occur that could not be resolved in the
same pass. Fortunately there are solutions to this problem.

     The first  is for  those UNIX  systems that have the command
{\cmd ranlib}. This command builds  a table  of references of a
library, so that all the necessary functions can be quickly
extracted. This is the ideal option, but not all UNIX systems have
this command.

     An alternative  is to  use the {\cmd lorder} and {\cmd tsort}
commands. These produce a sorted list of function dependencies, which
eliminates any forward references, except in  the case  when two
functions refer  to each  other.  To  find  the ordering for all the
{\cmd .o} files in the current directory, we use:
\begin{display}\cmd
lorder {\cd *}.o {\cd |} tsort
\end{display}
\noindent
     UNIX allows  us to  replace commands  by their  output by  using
backward quotes {\cd `\,`}.  Thus we can create a sorted archive
(called, say, {\cmd libme}) of all the
{\cmd .o} files in the current directory, by using
\begin{display}\cmd
ar -cr libme  `lorder {\cd *}.o {\cd |} tsort`
\end{display}
\noindent
     The last  alternative, which  may   be necessary  if we
don't  have {\cmd ranlib}, is  for {\cmd ld} to search the library
twice. This can be done by using the {\cmd -Ime} option to  {\cmd ld}
twice (as well as {\cmd -L}{\ms dir\/} to indicate which
directory the library is stored in).

     Note that  UNIX System  V supports random libraries automatically
through {\cmd ar}.

\section{Obtaining Information On Compiled Files}

There are two commands that provide information on programs that have
been compiled:
\begin{description}
 \item[{\cmd size}]  prints  the sizes of the sections (.text, .bss,
.data, or .code) of files. The syntax is as follows:
 \begin{display}\cmd
size $[${\ms options\/$]$ files}
\end{display}
\noindent
\index{size@{\cmd size}}The options are as follows:
\begin{display}
\begin{tabular}{@{}lp{.7\textwidth}@{}}
  {\cmd -o} & Select  octal   output. \\
  {\cmd -x} & Select  hexadecimal output.\\
  {\cmd -V} & Print the version of {\cmd size} being used.
\end{tabular}
\end{display}
\noindent
     The sizes  are printed  in decimal by default.

\item[{\cmd nm}]\index{nm@{\cmd nm}}  prints selected  symbol table
information from compiled files.  The information include symbol
names,  addresses, storage  classes and  sections of symbols.  If
the files  were compiled  with the  {\cmd -g} option,  the types,
sizes and lines at which the symbols were declared are also printed.
The format of the command is
 \begin{display}\cmd
nm $[${\ms options\/$]$ files}
\end{display}
\noindent
where the options are:
\begin{display}
\begin{tabular}{@{}lp{.7\textwidth}@{}}
  {\cmd -o} & Select  octal   output \\
  {\cmd -x} & Select    hexadecimal output\\
  {\cmd -h} & Don't print a heading \\
  {\cmd -v} & Sort the symbols by value\\
  {\cmd -n} & Sort the symbols by  name\\
  {\cmd -e} & Select only  the static  and extern  symbols\\
  {\cmd -u} & Select only  the undefined symbols\\
  {\cmd -f} & Print all symbols in the symbol table\\
  {\cmd -V} & Print the version of {\cmd nm} being used\\
  {\cmd -T} & Truncate symbol names that are too long to
  				    be printed in the name column.
\end{tabular}
\end{display}
\end{description}
     One \index{time@{\cmd time}}other useful command is the {\cmd time} command.
\begin{display}\cmd
time {\ms command\/}
\end{display}
\noindent
     where {\ms command\/} is an  arbitrary command to the UNIX shell
(and thus could be the  execution of  a program). This will print how
long the command took to execute, as  well as  how much  of this time
was spent in the user part of the program and how much was spent in
the system (for example, during I/O).

