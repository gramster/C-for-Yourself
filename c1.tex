\cleardoublepage
\chapter{The C Programming Language}


A C  program consists  of {\kc preprocessor commands}, {\kc
comments}, some {\kc top-level declarations}, and  one or  more {\kc
functions},  possibly spread over more than one file. One  and only one of the 
functions must  be called  {\cd main}\index{main@{\cd main}}; this  is always  
the first
function to  be executed.  Functions may refer to (or call) one
another, with the exception of  {\cd main}, which cannot be called by
any other function. The order of the functions  in the  file is
incidental; they are typically grouped together according to their
purpose.  The next section describes the preprocessor and all but the
first paragraph may be skipped on the first reading of this book.

\section{The Preprocessor}
\index{cpp@{\cmd cpp}}

The C preprocessor is  a simple  macroprocessor, controlled  by
special  command  lines within the source program.  Each command line
begins with a hash ({\cd \#}); in most compilers, this must be the
first character on the line. The common commands are:
 \begin{display}
\begin{tabular}{@{}ll@{}}   
          {\cd \#define}       &  define a macro \\
          {\cd \#undef}        &  remove a macro definition \\
          {\cd \#include}      &  include another file \\
          {\cd \#if, \#ifdef,}  & conditional ... \\  
          {\cd \#ifndef, \#else,} & ...test... \\
          {\cd \#endif}         & ... commands
\end{tabular}
\end{display}
\noindent
 The preprocessor  executes all  such  commands  it  finds  in the 
source file, removing them as it does so. Preprocessor commands can
be split over more than one line by ending each line except the last
with a backslash (\verb+\+).

\subsection{Macro Definitions}

Macro definitions  provide a  way of  naming useful constants, as well
as improving readability  and abbreviating  common short  pieces of
text. A macro definition has the form \index{define@{\cd \#define}}
\begin{display}\cd
\#define {\ms macro\_name\/} $[$({\ms arguments\/})$]$  $[${\ms
macro\_body}$]$
\end{display}
\noindent
where $[]$ indicate optional parts.

Once a macro has been defined, and until the end of the file containing
the definition or  an {\cd \#undef} \index{undef@{\cd \#undef}} command for  the same  macro,
{\em all occurrences of the macro name  in the  file will  be replaced by
the macro body.}  If arguments are used, these will be substituted into
the macro body.

Some examples will help. Firstly, naming useful constants:
\begin{display}\cd   
\begin{tabular}{@{}ll@{}}   
      \#define TRUE   & 1 \\
      \#define FALSE      & 0 \\
      \#define PI         & 3.1415927 \\
      \#define BUFFERSIZE & 80 \\
      \#define STAR       & '*' \\
      \#define ERROR      & -1 
\end{tabular} 
\end{display}
\noindent
These are  reasonably obvious  examples. As  far as  the  first  two 
are concerned, C provides no reserved words for the Boolean\index{booleans}
values
true and false, but treats  zero as  being false  and everything else
as being true, with one being the canonical true  value. Thus,
by defining  these macros, we can use {\cd TRUE} and  {\cd FALSE} in
our program  instead of  the more  obscure 1  and  0.  The {\cd
BUFFERSIZE} example  illustrates a technique for improving
readability \index{style!readability}(as with {\cd TRUE} and  
{\cd FALSE}), as  well  as ease 
of  modification.  This  constant  would presumably be  used when 
declaring line  buffers, for example. If the desired size of  the
line  buffer changed,  it would  only be  necessary to change the
macro definition,  instead of  searching for  the number 80 and
changing it to the new number throughout the file, which is an
error-prone method.

Examples of macros taking arguments are:
\begin{display}\cd
\begin{tabular}{@{}lll@{}}   
 \#define  &  square(x)             &  ((x)*(x)) \\
 \#define  &  cube(x)               &  ((x)*(x)*(x)) \\
 \#define  &  hypotenu(a,b)         &   (sqrt(square(a)+square(b))) \\
 \#define  &  distance(x1,y1,x2,y2) &   (hypotenu(x1-x2,y1-y2)) 
\end{tabular}
\end{display}
\noindent
 Macros thus  offer a  simple alternative  to functions\footnote{A
function in C is a group of statements associated with a name, by
which they can be called and executed.}. The difference is that a 
function is  compiled once, and every time it is referenced, a call is
made to  that same  single function.  A macro,  however, is simply
{\em a textual replacement before compilation even  begins\/}. Thus 
every time the macro is used, the compiler will generate code  for
the  whole macro. 
\index{macros!compared to functions}
\index{functions!compared to macros}
The advantage of macros is that they  make
programs  easier to  read and  modify. In  general, the use of
complex macros  is discouraged,  as a  function would  almost always
serve the purpose better. However, a macro is generally better than a
trivial function.

Macros can  refer to  one another, \index{macros!within macros} as was  shown  above:  when  a 
macro reference occurs  in a  program,  the  macro  body  is 
substituted,  and  the preprocessor carries on from the
{\em beginning\/} of the substituted text. However, a macro cannot refer to
another macro which has not yet been defined.

{\bf Warning:} \index{macros!dangers of}You  should use  parentheses in  macros wherever 
they could  be ambiguous. This is surprisingly often. Take, for
example, the following macro:
\begin{code}
\#define  SQUARE(x) \tab  x*x
\end{code}
\noindent
If this was used in the program as:
\begin{code}
SQUARE(z+1)
\end{code}
\noindent
this would expand to:
\begin{code}
z+1*z+1
\end{code}
\noindent
or {\cd z+z+1}, which is hardly what we want. The safest definition here is
the one given earlier for {\cd square(x)}.

To remove a definition, use the {\cd \#undef} \index{undef@{\cd \#undef}}command as in
\begin{code}
\#undef SQUARE
\end{code}
\noindent
     By convention \index{macros!style}\index{style!macros}upper case letters are usually used for macros that
are simply constants (e.g.\ PI). Macros that take arguments are often
regarded as being similar to functions, and the convention then does not
apply.


\subsection{Including Text From Other Files}

The command\index{include@{\cd \#include}} 
\begin{code}
\#include  <filename>
\end{code}
\noindent
causes  the  entire content  of  the specified file  (the {\kc
include file\/}) to be processed as  if it had appeared  in  place of
the command. It may be  that the included file itself contains {\cd
\#include} commands; the depth to which this is possible varies from
system to system.

     File names  must be  in double quotes ({\cd ""}) or angle brackets
({\cd <>}). The quotes are used  for user  files, the  angle brackets 
for  UNIX  system  files.  
\index{include@{\cd \#include}!<>@{\cd <>} form}
\index{include@{\cd \#include}!""@{\cd ""} form}
The difference is that files enclosed in
quotes are first searched for in the same directory as  the program, 
and then  in the standard directories, while files enclosed in angle
brackets are searched for in the standard directories only.  Under
UNIX, the standard include directory is {\cd /usr/include}.

There is  a  standard input/output header file, called 
{\fn stdio.h}\index{stdio.h@{\fn stdio.h}},
which must be included  by any  C program that performs input or
output. As it is a short file, and defines a few useful constants, it
is best at first to always include it. You should print  a copy of your
system's {\fn stdio.h} file and examine it to see what constants it
defines (for example, some define the {\cd TRUE} and {\cd FALSE}
constants we defined above, while others do not).

     

\subsection{Conditional Compilation}\index{conditional compilation}
\index{style!portability}

     Conditional commands  allow parts  of  the  program  to  be 
included  or excluded from  compilation, depending  on some 
condition. As the preprocessor must be  able to determine the
condition, only simple constant expressions are allowed, with the
usual C convention that if the expression has value zero, it is
considered false, otherwise true. The main use of conditional
compilation is to make programs more portable, and to (conditionally)
include debugging code in a program.  The general form of a
conditional is:
\index{if@{\cd \#if}}
\index{else@{\cd \#else}}
\index{endif@{\cd \#endif}}
\begin{code}
\#if {\ms constant\_expression\/} \\
 \> {\ms text\_section\_one\/} \\
\#else \\
 \> {\ms text\_section\_two\/} \\
\#endif
\end{code}
\noindent
Constant expressions \index{expressions!constant}will be described in detail later; for now it is
enough to say that {\em constant\_expression\/} is an expression that can be
evaluated immediately from  the   information  known   by  the 
compiler  (or,  in  this  case,  the preprocessor). If  the
expression  is true,  {\ms text\_section\_one\/} is included by
the preprocessor  and {\ms text\_section\_two\/}  is ignored; if
the  expression  is false, it is the other way around.

Two other forms of {\cd \#if} are available, namely:
\index{ifdef@{\cd \#ifdef}}
\index{ifndef@{\cd \#ifndef}}
\begin{code}
\#ifdef  {\ms macro\_name\/}
\end{code}
\noindent
and
\begin{code}
\#ifndef {\ms macro\_name\/}
\end{code}
\noindent
These test  whether the  macro name  has been  defined or  is 
undefined, respectively. This could be of use, for example, if we are
unsure whether {\cd TRUE} has been defined. We could use:
\begin{display}\cd
\begin{tabular}{@{}ll@{}}
  \#ifndef &  TRUE \\
  \#define &  TRUE 1 \\
  \#endif &
\end{tabular}
\end{display}
     

\section{Comments}
\index{comments}

Comments in  C are  enclosed within  {\cd /*} and {\cd */}.  Comments
can continue over more than one line. You should not use comments within
comments; {\cd /*} is not recognised within a comment,
so (usually) the first {\cd */} found indicates the end of comments. 
However, this is compiler dependent and not all compilers may act
this way.

\section{Functions}
\index{functions}

A function  is a  collection of  {\kc declarations\/} and  {\kc
statements\/} enclosed in braces {\cd \{\}}, and preceded by a {\kc
function name}. The minimal C function, which does
nothing\footnote{This is not quite true, because the function, as
declared here, returns a value by default, but more about this
later.} (it contains no declarations or statements), is:
\begin{code}
 tiny() \\
 \{ \\
 \}
\end{code}
\noindent
 The parentheses  following the  name are  necessary;  they  are  used
to enclose the  names of  the function's  arguments\index{arguments}, if any. Even if
there are no arguments, they  serve to  indicate to the compiler that
the name is that of a function, and not a variable. Function names
\index{functions!names}must be valid C {\kc identifiers\/}.

     

\subsection{Identifiers}
\index{identifiers}

     Identifiers in  C are  names for functions and variables. They
may be any combination  of  upper  and  lower  case letters, 
digits  (0--9),  and  the underscore character (\_), the only
restriction being that they may not begin with a  digit (this  is so
that the compiler can recognise them as identifiers and not numbers).
In older compilers only the first eight characters are significant, thus the
compiler  might not  distinguish between  {\cd long\_name\_1} and  
{\cd long\_name\_2}. The case of the letters is significant,  thus 
{\cd Total}  and  {\cd total}  are  different identifiers (it  is
considered  bad style  to have two identifiers that differ only in
their case)\index{style!identifier names}.

     All identifiers are characterised by two attributes: their {\kc
type\/} \index{types}and their {\kc storage class\/}\index{storage classes}.  The  type  represents the 
identifier type,  for  example,  {\cd int} (integer), {\cd float} 
(floating point  number), {\cd char} (character), or function. The
storage class  determines the  identifier's {\kc scope\/}  
\index{scope}and {\kc visibility\/}; that is, how accessible the 
identifier is from other
parts of the program. The main storage classes are:
\begin{description}

\item[external:] 
\index{extern@{\cd extern}}\index{storage classes!extern@{\cd extern}}an external 
identifier can be referenced from
anywhere in the program, even  if the  program is  split over 
several  files.  The linker ({\cmd ld}\index{ld@{\cmd ld}})  knows the  details of  all
external  identifiers and can thus resolve  references across  files.
All functions and top-level declarations are  external unless 
otherwise specified.  Memory for external variables is allocated once
only, at compile time. Another name for external identifiers is {\kc
global identifiers\/}.

\item[automatic:] 
\index{auto@{\cd auto}}\index{storage classes!auto@{\cd auto}}this is the 
default storage class for all variables
declared within blocks.  Top-level declarations of automatic
variables are  not allowed.  An automatic variable can only be
accessed within the block \index{statements!compound}containing 
its  declaration.   It  is  
called  automatic because  memory  is automatically allocated  for it
when the block is entered and given up when the block is exited (that
is, at run time, not compile time as in external). This means that
the value of an automatic variable is lost  when the  function is 
exited (they are {\kc volatile}).  The linker \index{ld@{\cmd ld}}does 
not need any information about  automatic
variables.  Another name for automatic variables is {\kc local
variables\/}, as they are local to the block that contains them.

\item[static:] 
\index{static@{\cd static}}\index{storage classes!static@{\cd static}}static  
variables are much the same as external
variables. Like externals, they  have storage  allocated for  them
once, at compile time. Thus  the value  of a  static variable  is
{\em not}  lost when the containing block  is exited (it is 
{\kc non-volatile}).  However, a
static  identifier is not visible outside its containing block, is
not available to  the linker, and thus cannot be accessed across
files.  They thus  provide a  way of limiting  the  sharing  of 
non-volatile variables. A  typical use  of static  identifiers  is  for 
use  in execution profiles  -- counting the number  of times  a
function is executed. Each  function can have a  static variable
which has one added each time the function is entered.  Such a
counter can also be used by a function to determine the first time it
is being executed and act appropriately.  For example, an output routine 
can open a file for output the first time it is called.

\end{description}   

\subsection{Simple Declarations and Statements}
\index{declarations}

     Before we can begin using functions, we need variables to hold
the values we want  to work  upon. All  variables in  C must  be
declared before being used.  This tells the compiler necessary 
details about  the variable,  such as the type \index{types}and storage 
class\index{storage classes}. At 
this stage,  we will  just consider  very  basic  declarations of
integer and character variables. Such a declaration has the general
form
 \begin{display}
{\ms storage\_class\/} {\ms type\/} {\ms list\_of\_identifiers\/};
\end{display}
\noindent
 where {\ms storage\_class\/} is optional.  The default storage class
specifier is external for top  level  declarations  and  functions, 
and automatic  for  any  other declarations.  If {\ms
storage\_class\/} is present, it should be one of {\cd extern}, {\cd
auto}, {\cd static}, or {\cd register} (the latter will be described
later).  The {\ms type\/} specifier is essential  and
in this case should be either {\cd int} or {\cd char}.  {\ms
list\_of\_identifiers\/} must  contain at  least one identifier name,
but may have more, in which case each identifier must  be separated 
by  commas.  Like all  C  statements, a declaration is terminated by
a semicolon. For example:
 \begin{code}
int x,y,result; \\
static char c, next\_c; 
\end{code}
\noindent
Having declared  these variables,  we can  use them  in simple 
{\kc assignment statements\/}, for example:
\begin{code}
 result = x+y; \\
 c = '*';
\end{code}
\noindent
Note the  use of  {\cd =} for assignment as opposed to Pascal's {\cd
:=}, as well as single quotes  to enclose  {\kc character
constants\/} \index{constants!character}in C as opposed to double quotes, which are used for {\kc
string constants\/}\index{constants!string}.

     In order  to make  our programs  produce meaningful results, we
need some way of entering and printing the values of these variables.

\subsection{Basic Character and Integer Input/Output}

Input and  output statements,  strangely enough,  are not a part
of the C language itself.  C is a small language, and many of its
features are actually provided by  precompiled functions  in special 
{\kc system  libraries}\index{libraries}.  The linker 
\index{ld@{\cmd ld}}searches the libraries and
extracts only those functions used by the program, adding them to the
compiled program.  There  are libraries for  I/O functions, 
string handling,  math functions, and so on. To use the  standard
input/output  runtime  library,  the  standard  input/output header
file, {\fn stdio.h}\index{stdio.h@{\fn stdio.h}}, must be included in 
the source file using:
\begin{code}
 \#include  <stdio.h>
\end{code}
\noindent
 This provides access  to numerous  functions.  Two of
the most powerful (although their full power will only
become apparent later) are {\cd printf} 
\index{printf@{\cd printf}}for output and {\cd scanf}
\index{scanf@{\cd scanf}}for input.  

{\cd printf} takes at least one argument -- a string enclosed in
quotes. This is the string which {\cd printf} prints.  However,
within this string we can use {\cd \%d} to represent a place to
print  an integer in decimal,  and {\cd \%c}  to represent a place to
print a character. The appropriate variables  or constants  must
then  be included  as arguments  to {\cd printf}, in the correct 
order. For example,  assuming {\cd x} has value 3, {\cd y}
has value 27, and {\cd c} has value {\cd '*'}, then
 \begin{display}\cd
\begin{tabular}{@{}ll@{}}
 {\rm\bf The call:}           &          {\rm\bf will print:}\addVspace

 printf("Hello\verb+\+n");               &  Hello \\ 
 printf("\%d+\%d=\%d\verb+\+n",x,y,x+y); &  3+27=30 \\
 printf("x+\%d=\%d\verb+\+n",5,x+5);     &  x+5=8 \\
 printf("\%cc\%c\verb+\+n",c,c);         &  *c*
\end{tabular}
\end{display}
\noindent
Note the  use of  {\cd \verb+\+n} ({\em newline}) to indicate
that the line printed is to be terminated by  a line  feed. Several 
such {\kc escape  characters}\index{escape characters}  are  available, namely:
\begin{display}   
\begin{tabular}{@{}ll@{}}   
          {\cd \verb+\+b}   & backspace \\
          {\cd \verb+\+f}   & form feed \\
          {\cd \verb+\+n}   & newline \\
          {\cd \verb+\+r}   & carriage return \\
          {\cd \verb+\+t}   & tab \\
          {\cd \verb+\+'}   & single quote \\
          {\cd \verb+\+"}   & double quote \\
          {\cd \verb+\+\verb+\+}   & backslash \\
          {\cd \verb+\+0}   & ASCII null (0)
\end{tabular}
\end{display}
\noindent
{\cd scanf} is  very similar  in form  to {\cd printf}.  For the 
time being we will restrict ourselves  to two  forms, namely to  read
single integer  and  character variables respectively. The
simplest such forms are:
\begin{code}
 scanf("\%d",\&{\ms integer\_variable\_name\/}); \\
 scanf("\%c",\&{\ms character\_variable\_name\/});
\end{code}
\noindent
The use  of the  ampersand ({\cd \&})  before the variable names
will become clear later.

Let us consider a simple program to illustrate these concepts:
\begin{code}   
 \#include <stdio.h> \\
 /* Square/Cube Example 1 */ \addVspace

 main() \\
 \{ \+\\
   int x, x\_sq; \addVspace

   printf("Please enter a number\verb+\+n"); \\
   scanf("\%d",\&x); \\
   x\_sq = x*x; \\
   printf("\%d squared is \%d, cubed is \%d\verb+\+n", x, x\_sq, x*x\_sq); \-\\
 \}
\end{code}
\noindent
 All input  and output is done from and to files. The standard I/O
library provides three special files, called 
{\cd stdin}\index{stdin@{\cd stdin}}, {\cd
stdout}\index{stdout@{\cd stdout}} and {\cd stderr} 
\index{stderr@{\cd stderr}} (standard input, output, and  error files, 
respectively).  Under normal circumstances the  {\cd stdin} file is
the keyboard, and {\cd stdout} and  {\cd stderr} are  the screen.
{\cd printf} always prints to  {\cd
stdout}, while  {\cd scanf} reads  from {\cd stdin}.  {\cd stderr}
is  used for  error messages.

  UNIX allows  us to read from and write to data files, simply by using 
{\kc I/O redirection}\index{redirection of I/O} 
\index{stdin@{\cd stdin}!redirection}
and {\cd stdout}\index{stdout@{\cd stdout}!redirection}.
When  we execute  a compiled  program
(by typing its name) we can specify whether  to redirect the source of the
input, or  redirect the destination of the output, as follows:
\begin{display}
\begin{tabular}{@{}ll@{}}   
\cd > {\ms file} & Create {\ms file\/} and use it for {\fn stdout} output. \\
\cd < {\ms file} & Take {\fn stdin} input from {\ms file}. \\
\cd >> {\ms file}\tab & Append {\fn stdout} output onto the end of {\ms file}.
\end{tabular}
\end{display}
Thus, if  we wanted  {\cd printf}  to print  to a  file called
{\fn progout} and {\cd scanf} to read from a file called {\fn
progin}, we could for example use:
\begin{display}\cmd
a.out  {\cd <}progin {\cd >}progout
\end{display}
\noindent
If we wanted to send the output to both the screen and the file, we
could use the UNIX {\cmd tee}, as follows:
\begin{display}\cmd
a.out  {\cd <}progin {\cd |} tee progout
\end{display}

\subsection{Passing Arguments to Functions}
\index{arguments}

Our use of {\cd printf} above illustrates calling a function with
arguments. To define a  function with  arguments in  C, we  must list
the argument names, in order, in  the {\kc formal  parameter list} 
\index{arguments!formal}
(parameter=argument) which  follows the function name.  This must 
immediately be followed by the type declarations of the arguments. 
The order  of the arguments\index{arguments!order of} in the argument list is important, as the
same order must be used when calling the function, but the order of
the declarations \index{declarations!order of}is  unimportant. For  example, say we wanted to
print the squares and cubes of the first five integers. We could use:
\begin{code}   
/* Square/Cube Example 2 */ \addVspace
   
 main() \\
 \{ \\
  \>  doline(1); doline(2); doline(3); doline(4); doline(5); \\
 \} \addVspace
   
 doline(num) \\
 \> int num; \\
 \{ \\
  \> printf("\%d\verb+\+t\%d\verb+\+t\%d\verb+\+n", num, num*num, num*num*num ); \\
 \}
\end{code}
\noindent
 (Note: from  now on,  we  will  omit  the  {\cd \#include  <stdio.h>}
from  our examples, and assume it is always present).

  All arguments in C are {\kc passed by value}\index{arguments!passed by value}. In other words,
only the value of the argument  is passed  to the function. Upon
entry to the function, space is allocated for the formal parameters,
and these are initialised with the passed values. Using  a variable
as an argument in a function call will not result in the variable's
value being changed, even if the formal argument changes within the
function. Thus, executing:
\begin{code}
 main() \\
 \{ \\
	\> int x; \\
	\> x = 5; \\
	\> addone(x); \\
	\> printf("\%d\verb+\+n",x); \\
	\} \addVspace

 addone(n) \\
 \> int n; \\
 \{ \\
  \> n = n+1; \\
 \}
\end{code}
\noindent
 will print  the result  5, not 6. When the function is exited,
the values of the formal arguments are simply discarded.

\subsection{Returning Values from Functions}

\index{functions!returning values}
     We saw  earlier how  to  use  preprocessor  macros  to  implement
simple functions, such as squaring and cubing numbers. We can do this
with functions, as follows:
\begin{code}   
 /* Square/Cube Example 3 */ \addVspace

 square(n) \\
 \> int n; \\
 \{ \\
 \> return n*n; \\
 \} \addVspace

 cube(n) \\
 \> int n; \\
 \{ \\
  \> return n*n*n; \\
 \} 
\end{code}
\noindent
     The {\cd return} statement \index{statements!return@{\cd return}}causes 
the containing function to be
exited, and the function call effectively to be ``replaced''  by the 
returned value. We could call the above functions in different ways,
for example:
\begin{code}
 x\_sq = square(x); \\
 printf("\%d cubed is \%d\verb+\+n", n, cube(n)); \\
 square(2);
\end{code}
\noindent
     In the last example, the function will be called, and the value
4 will be returned. However,  as we  are not  using the  value
returned by the function, this value will simply be discarded.  It
turns  out that  this may  happen often.  Unless told otherwise, C
expects  all functions  to return an  integer, and  reserves space 
for this. This is  fine when  we are returning an integer, but there
are two other cases: we may not want to return any value, or we may
want to return a type other than integer.

     These problems are easily solved: we simply have to declare
the type of the function. For example:
\begin{code}          
 \#define   PI   3.1415927 \addVspace

 float circumf(radius) \\
 \> float radius; \\
 \{ \\
 \>   return 2*PI*radius; \\
 \}
\end{code}
\noindent
     C provides  a special  type, {\cd void},  to solve  the first  
problem. A  {\cd void} function is one which returns no value. For
example, our {\cd doline} function could be written:
\begin{code}
 void doline(n) \\
 \> int n; \\
 \{...\}
\end{code}
\noindent
     The advantages  of using {\cd void} are firstly that you are
explicitly stating that no  return value can be expected from the
function, and secondly, you save time and  space by  not allocating 
any unnecessary space and returning arbitrary values (which  is
what  happens if  you don't  specify {\cd void} and don't return a
value).

     A limitation  of {\cd return} is that you can only return a
single value from a function. Some  languages allow  you to call a
function and pass the arguments 
{\kc by reference}\index{arguments!passed by reference},  that is, passing
the addresses of variables rather than just their values. This
allows the function to affect the variable values directly.  While
all  C calls  are  by  value, there is a way to effectively make
calls by reference, which we will examine shortly.

