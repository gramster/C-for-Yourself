\cleardoublepage
\chapter{Program Development and Style}
\section{C Style}
\index{style}
    C  was  designed for  the development of large-scale applications
based on a uniform philosophy (UNIX!).  The result is that the
language lends itself well to writing modular programs.   Programs
may be split over several files which can be separately compiled and
updated, they can include text from other files, conditional
compilation is allowed, and the visibility of  functions and 
variables can  be easily controlled with storage class specifiers.

     This is helpful in the development of large programs.
However, to successfully develop  and maintain  a large program, good
programming style is essential. This aids readability, ease of
modification, and portability. Good, clear programming  style is 
generally much  more desirable than efficiency or brevity.
A normal, sane person would not, for example, want to write a program 
like the one in Figure~\ref{obfuscate}.
\begin{figure}\small
\begin{verbatim}
long h[4];E[80],S;t(){signal(14,t);if(S)longjmp(E,1);}c,d,l,v[]={(int)t,0,2},
w,s,I,K=0,i=276,j,k,q[276],Q[276],*n=q,*m,x=17,f[]={7,-13,-12,1,8,-11,-12,-1,9
,-1,1,12,3,-13,-12,-1,12,-1,11,1,15,-1,13,1,18,-1,1,2,0,-12,-1,11,1,-12,1,13,
10,-12,1,12,11,-12,-1,1,2,-12,-1,12,13,-12,12,13,14,-11,-1,1,4,-13,-12,12,16,-
11,-12,12,17,-13,1,-1,5,-12,12,11,6,-12,12,24};u(){for(i=11;++i<264;)if((k=q[i
])-Q[i]){Q[i]=k;if(i-++I||i%12<1)printf("\033[%d;%dH",(I=i)/12,i%12*2+28);
printf("\033[%dm  "+(K-k?0:5),k);K=k;}alarm(1);Q[263]=c=((S=1)&&!setjmp(E))?
getchar():-1;alarm(0);}G(b){for(i=4;i--;)if(q[i?b+n[i]:b])return 0;return 1;}g
(b){for(i=4;i--;q[i?x+n[i]:x]=b);}main(C,V,a)char**V,*a;{for(a=C>2?V[2]:
"jkl pq";i;i--)*n++=i<25||i%12<2?7:0;srand(getpid());system("stty raw -echo");
signal(14,t);t();puts("\033[H\033[J");for(n=f+rand()%7*4;;g(7),u(),g(0)){if(c<
0){if(G(x+12))x+=12;else{g(7);++w;for(j=0;j<252;j=12*(j/12+1))for(;q[++j];)if(
j%12==10){for(;j%12;q[j--]=0);u();for(;--j;q[j+12]=q[j]);u();}n=f+rand()%7*4;G
(x=17)||(c=a[5]);}}if(c==*a)G(--x)||++x;if(c==a[1])n=f+4**(m=n),G(x)||(n=m);if
(c==a[2])G(++x)||--x;if(c==a[3])for(;G(x+12);++w)x+=12;if(c==a[4]||c==a[5]){
printf("\033[H\033[J\033[0m%d\n",w);if(c==a[5])break;for(j=264;j--;Q[j]=0);
while(getchar()-a[4]);puts("\033[H\033[J\033[7m");}}system("stty cooked echo")
;d=popen("cat - HI|sort -rn|sed -n 1,20p>/tmp/$$;mv /tmp/$$ HI;cat HI","w");
fprintf(d,"%4d on level %1d by %s\n",w,l,getlogin());pclose(d);}
\end{verbatim}
\caption{\label{obfuscate} The winner of the 1989 obfuscated C contest.  It
is a valid, working C program that lets you play Tetris on UNIX -- it even
keeps a list of high scores!}
\end{figure}

     In this section, we will examine some stylistic conventions to
be used with C, as well as indicate how greater readability and
portability can be achieved.

     The great  advantage which  UNIX and  C provide  is the 
``software tool'' \index{software tool}\index{style!software tool}philosophy upon which they are based. A software
tool (as defined by Kernighan and Plauger in their excellent book 
{\em Software Tools\/}) is a program that:
\begin{itemize}
\item  solves a general problem, not a special case, and
\item  is so easy to use that people will use it rather than write their
own.
\end{itemize}
     Many of  the UNIX  utilities are  software tools.  In fact, 
unlike  most operating systems  which are  large complex programs,
UNIX consists of a small program called  the {\kc kernel},  which is 
responsible for  only  the  most  basic functions. All other
operating system tasks are performed by software tools written in C (or
the UNIX shell command language).

     In order  to write  a good UNIX software tool, there are some
conventions that should  be followed.  The tool should solve a
single, general problem, to keep its  use as simple as possible.
Wherever possible, it should be a {\kc filter}; that is,  it should
read its input from the standard input and send its output to the
standard output. This means that the program can be used in
conjunction with other  programs and the UNIX I/O redirection
commands and especially pipes, to create even more powerful tools. 
Some stylistic guidelines follow.  For a more thorough treatment of the
topic (as well as writing portable code), refer to appendices~\ref{style}
and \ref{portable} containing recent papers on the subject.

     First of all, \index{style!functions}functions should perform 
single logical operations
wherever possible. Functions  should be  kept short  (about 20 lines
is a good length), and should be preceded by comments explaining what
they do, what the arguments are for,  and what  is returned,  if
anything.  Any piece  of code  within the function whose  purpose
may  seem obscure  should also  be commented. Wherever possible,
allow  only one  entrance to  and one exit out of a function or compound
statement.

     Macro names  \index{style!macros}should be  used wherever  possible  to  represent 
important constants. It  is conventional  in C  to use upper case
names for such macros.  \index{style!identifier names}Identifier names  should be  a compromise 
between meaningfulness and brevity.  Similarly, {\cd typedef} should
be used to simplify declarations.

     Lines should  be kept  short, and  indentation used  to show the
relative nesting of  statements.  Three-column indentations are
recommended.
Complex  expressions that are often repeated should  be made 
into functions if possible, or named using macros.  For example:
\begin{code}
\#define QUADROOT1(a,b,c) \= \kill
\#define DISCRIM(a,b,c) \>   (sqrt((b)*(b)-4*(a)*(c))) \\
\#define QUADROOT1(a,b,c) \> ((-(b)-DISCRIM(a,b,c))/(2*(a))) \\
\#define QUADROOT2(a,b,c) \> ((-(b)+DISCRIM(a,b,c))/(2*(a)))
\end{code}
\noindent
     would allow  us to  use the  macros {\cd QUADROOT1}  and {\cd
QUADROOT2}  in place of complex expressions for the roots of
quadratic equations.

     The number of files {\cd \#include}d, apart from the standard
library .h files,  should be kept small.

     Closely related  functions should  be kept  in the  same file.
The use of {\cd continue}, and  especially {\cd goto}, is strongly
discouraged, as it is often a sign of a badly structured function.

     While these  points may  reduce the efficiency of programs
slightly, good program and  data structure  design are far more
important efficiency factors.  The small price to pay for readable
programs is well worth it.

     For a  software tool  to be  truly useful,  there are some
user-interface conventions which should be followed. 
\index{style!UNIX command line format}Firstly, you
should use {\cd argc} and {\cd argv} to enable the  program to be
invoked using the UNIX standard command format. This is:
\begin{display}\ms
 program  $[$options$]$  $[$files$]$
\end{display}
\noindent
     Options must begin with a dash {\cd -} and may take the form:
\begin{display}\ms
{\cd -}options
\end{display}
\noindent
or   
\begin{display}\ms
{\cd -}option\/ {\cd -}option \ldots
\end{display}
\noindent
 The first form is only allowed for options that don't take arguments.

     The convention  is that output from the program should be sent
to {\fn stdout}, unless the {\cmd -o}{\ms file\/}  option is  used,
which  specifies which file the output should be  sent to. Any error
messages or diagnostics should be sent to {\fn stderr}.  This
prevents interference with output that might be piped as input to some
other process.

     The optional  {\ms files\/} argument  specifies a  list of  
files to be used for input, one  at a  time. If  no files  are
given,  {\fn stdin} should  be used  as  a default. This  convention
allows  the use  of UNIX  wild  card characters  as  file arguments.

     UNIX programs  such as  {\cmd cc} follow  the convention  that
``no news is good news.''  While this can sometimes be disconcerting, 
it does have  its merits,  as  no unnecessary  information  is
produced to clutter output and prevent piping of output to other
routines.

     Many UNIX  utilities and  filters are  written specifically  to
deal with text files.  Thus, by  making the  input and  output data
from your files text wherever possible, you have access to all of
these utilities.

     \index{style!portability}Portability can  often be  improved  using  conditional 
compilation  and macros. A  good example  of this  is the  {\cd FILE}
type  in the {\fn stdio.h} file. Some systems use  pointers to 
structures for  file pointers (usually pointing to a structure 
containing  information  about  the  file,  including  the  current
position), while  others use  pointers to  int (usually  just an 
index into a table of  file descriptors).  While these  are
essentially the same (as we can regard an  index into a table as
being an address of an element of the table), they are declared
differently. To declare a file, we would use
\begin{code}
struct file\_info *fp;
\end{code}
\noindent
     in the first case and
\begin{code}
int *fp;
\end{code}
\noindent
     for the second. Suppose we wish  to write  portable programs  for  two
machines,  XYZ and 
 ABC, respectively.  Assume furthermore that on XYZ {\cd int}s are 16 bits,
whereas on ABC they are 32 bits, and that we want them to be 32 bits. On
XYZ we would use
\begin{code}
typedef struct file\_info *FILE;
\end{code}
\noindent
     in the {\bf stdio.h} file, while on  ABC we would use
\begin{code}
typedef int *FILE;
\end{code}
\noindent
     We could  incorporate this  into our own program as follows
(there is not much point to the  {\cd FILE} example,  as it  is done  for
us  by {\fn stdio.h},  but it illustrates the point):
\begin{code}
       \#define XYZ  \tab /* Define which machine we are on */ \addVspace

       \#ifdef XYZ					 \\
       typedef int long; \tab /* redefine ints to be 32 bits */ \\
       typedef struct file\_info *FILE;			   \\
       \#else						   \\
       \#ifdef ABC					   \\
       typedef int *FILE;				   \\
       \#else						   \\
       No machine defined!				   \\
       \#endif						   \\
       \#endif
\end{code}
\noindent
     The line  {\cd No machine  defined!} will be rejected by the
compiler, thus you will have to define the machine before you can
successfully compile and run.

     The \index{cpp@{\cmd cpp}}preprocessor option {\cmd -D}{\ms identifier\/} can also be
used here. This option causes {\ms identifier\/} to be defined to the
preprocessor. Instead of explicitly defining the  machine in  the
program,  we could specify it as the {\cmd -D} argument to {\cmd cc}, 
for example, 
\begin{display}\cmd
 cc -DXYZ myprog.c
\end{display}
     

\section{Common Mistakes}
\index{debugging}
     Debugging a  C program  can be  as easy  or difficult  as you
make it for yourself. Well-structured,  readable code  simplifies the
task considerably to start with.  Another good  practice is  to
develop the program bit-by-bit (say three or four functions at a
time), thoroughly testing each part before moving on. C  aids this 
process, by allowing us to put all top-level declarations in an {\cd
\#include} file, and then keep all well-tested functions in one file,
and the latest untested  additions in  a second  temporary file. Once
we are satisfied that the  latest additions  are working correctly,
we can add them to the main file (using,  for example,  the UNIX {\cmd
cat}  command), and  begin a new part in the temporary file.

     C also  provides a  number of  useful tools to aid the debugging
process; these are  described in the next section. Although it is
impossible to predict likely errors  in  {\em a priori\/},
there are a few features of C that are often incorrectly used. These
include\index{debugging!common mistakes}:
\begin{itemize}

\item Using  {\cd =} instead  of {\cd ==}  in  equality  tests,  thus 
performing  an unwanted assignment, and returning an arbitrary value
(the right-hand side of the assignment).

\item Failing  to  terminate  a  statement  with  a  semicolon;  this 
is particularly the case just before {\cd else} statements.

\item Failing  to use {\cd break} statements within a {\cd switch}; 
in most cases, we don't want  to continue  processing all  statements
to the end of the switch, but only those in the particular {\cd
case}; thus we need a {\cd break} statement at the end of the case.

\item Using  a semicolon  at the  end of  the argument list in a
function definition, thus  making that  line look  like  a  function 
call instead.

\item Forgetting to use the {\ms address of\/} operator ({\cd \&}) in
calls to {\cd scanf} and other functions requiring pointer arguments.

\item Forgetting  to close  {\cd FILE}s that  have been opened.  While
this will not cause an error, some systems may not close the files or
flush the buffers, leading to incomplete I/O.

\item Incorrect use of pointers.

\item Failing  to ensure  that operator precedences in expressions are
correct (if in doubt, use parentheses).

\item \index{macros!dangers of}Using arguments with 
{\em side-effects} within macros. For example, if we have:
\begin{code}
\#define toupper(c) \tab (c>='a' \&\& c<='z') ?\ c+'A'-'a' :\ c
\end{code}
\noindent
and then call the macro with
\begin{code}
void copy\_upper(s1, s2) \\
\>char *s1, *s2; \\
\{ \\
\>while( *s1++ = toupper(*s2++) ); \\
\}
\end{code}
\noindent
only every second character will be copied and, if we are unlucky,
the null byte at the end of {\cd s2} will be missed with the result that the 
loop will not terminate properly.

\item Failing  to propagate changes made  to the  program throughout  the
whole program. This  is particularly  so with  functions; if you
change the parameter  list of a function, make sure you change all
calls to the  functions as  well, as  C performs  no checks on
function calls, but simply performs them.

\end{itemize}
     Bearing these  points in  mind could save a reasonable amount of
time and effort. Of course, the best way to improve debugging skill
is through practice -- the  more you program, the less mistakes you
are likely to make in the first place, and  the more  experience you 
will have with identifying those that do occur.
