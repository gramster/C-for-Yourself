% remove [portableC] from the \documentstyle command below if you
% prefer the old format.  However, be sure to somehow include the
% section marked `% incorporate any additional commands I find necessary'.

\chapter{Portable Programs}
\bibliographystyle{alpha}
\label{portable}

% The number between brackets is the minor revision number which
% must be removed when we finally agree on the contents.

\title{{\bf Notes on Writing\\Portable Programs in C}\\
       {\small (Nov 1990, 8th Revision)}
      }

\author{A. Dolenc%
  \protect\thanks{Internet: \id{ado@sauna.hut.fi}.}
       \\ A. Lemmke \\
  {\em Helsinki University of Technology} \\
         D. Keppel%
  \protect\thanks{Internet: \id{pardo@cs.washington.edu}.} \\
  {\em CS\&E, University of Washington} \\
         {\normalsize and} \\
	G. V. Reilly%
  \protect\thanks{Internet: \id{gvr@cs.brown.edu}.} \\
  {\em Dept.\ of Computer Science, Brown University}
}
\porttitle

{
\abstract
\parskip=4pt plus 1pt
\parindent=0pt

This documents describes the features and non-features of 
different C~preprocessors, compilers, and environments.  As such,
it is an incomplete document, growing as information is gathered.
It contains some material concerning ANSI~C but it is not a
substitute for the Standard itself.
We assume the reader is familiar with the C~programming language.
\endabstract
}

\pagebreak
%\parskip=4pt plus 1pt
%\parindent=0pt
\raggedbottom

%+++++++++++++++++++++++++++++++++++++++++++++++++++++++++++++++++++++++++++++
\section{Foreword}
%+++++++++++++++++++++++++++++++++++++++++++++++++++++++++++++++++++++++++++++

We will call a program {\em portable\/} if adapting it to a new
environment is easier than rewriting it for that environment.
This document is mainly for those who have {\em never\/} ported
a program to another platform --- a specific hardware and
software environment --- and, evidently, for those who plan to
write large systems which must be used across different vendor
machines.  If you have already done some porting, you may not
find the information herein very useful.

We suggest that \cite{style} be read in conjunction with this
document.\footnote{\cite{style} can be obtained via {\em
anonymous FTP\/} from \site{cs.washington.edu} in
\file{\twiddle{}ftp/pub/cstyle.tar.Z}\@.} Posters to the newsgroup
\ng{comp.lang.c} have repeatedly recommended \cite{MH} and
\cite{AK} (none of the information herein has been taken from
those two references).

{\bf Disclaimer:} We will attempt to keep the information herein
updated, but it can happen that some of it may be incorrect at
the time of reading. The code fragments presented are intended
to make applications ``more'' portable, meaning that they may
fail with some compilers and/or environments.

{\footnotesize

This document can be obtained via anonymous FTP from
\site{sauna.hut.fi} [130.233.251.253] in
\file{\twiddle{}ftp/pub/CompSciLab/doc}.  The files
\file{portableC.tex}, \file{portableC.sty},
\file{portableC.bib}, and \file{portableC.ps.Z} are the \LaTeX\
source and style files, {\sc Bib}\TeX\ and the compressed {\sc PostScript},
respectively. Alternatively, there is a site in the US
from which one can obtain all four
files, \site{cs.washington.edu} [128.95.1.4] in
\file{\twiddle{}ftp/pub/cport.tar.Z}\@. All files are in the
public domain. Comments, suggestions, flames, eggs, and requests
for copies via e-mail should be directed to
\id{ado@sauna.hut.fi}.
}


%+++++++++++++++++++++++++++++++++++++++++++++++++++++++++++++++++++++++++++++
\section{Introduction}
%+++++++++++++++++++++++++++++++++++++++++++++++++++++++++++++++++++++++++++++

The aim of this document is to collect the experience of several
people who have had to write and/or port programs written in~C
to more than one platform.

In order to keep this document within reasonable bounds, we must
restrict ourselves to programs which must execute under
Unix-like operating systems and those which implement a
reasonable Unix-like environment. The only exception we will
consider is VMS\@.

A wealth of information can be obtained from programs that have
been written to run on several platforms. This is the case of
publicly available software such as that developed by the Free
Software Foundation and the MIT X~Consortium.

When discussing portability, one focuses on two issues:

\newcommand{\newblock}{}
\bibliography{portC}

