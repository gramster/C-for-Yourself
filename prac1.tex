Practical Session 1


1) Use vi to enter the program below. The program replaces tabs in a
file with spaces. Use tabs to indent the program - this will allow us
to test the program with its own listing:


#include <stdio.h>
#define TABSTOP 8

main()
{
int col;
char c;

col=0;
while (!feof(stdin)) {
     c=getchar();
     if (c=='\t')
          do putchar(' '); while (++col%TABSTOP);
     else {
          col++;
          putchar(c);
          if (c=='\n') col=0;
          }
     }
}

2) Use cc to compile the program, and test it with the program
listing.

3) Define three vi macros. The first, 'w', should insert 'while ()' at
the current position, and put vi into insert mode at the right
parenthesis ')'. The second, 'd', is similar, except with text 'do
()'. The third, 'c', should insert '{ }' and position vi in insert
mode at the '}'. The last, 'b', should insert '( )' and position vi in
insert mode at the ')'. Think how these macros could help us enter
programs such as the one above. Using a macro such as 'b' for brackets
has the added benefit that we are less likely to have missing brackets
in expressions.

4) Write, enter and test a shell script 'develop <program> <test
file>', to repeatedly execute vi on <program>, compile <program>, and,
if no errors occured in compilation, execute a.out with <test file> as
input.


