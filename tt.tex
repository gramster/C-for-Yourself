\documentstyle[11pt,A4,lists,twoside,miihead]{report}
%\showboxbreadth=20
%\showboxdepth=20
\pagestyle{biiheadings}
\renewcommand{\lefthead}{Wheeler and Smit}
%
\newcommand{\fn}{\bf}	% file name
\newcommand{\kc}{\em}	% key concept
\newcommand{\ms}{\it}	% meta symbol
\newcommand{\cd}{\tt}	% program code
\newcommand{\cmd}{\sf}	% command
\newcommand{\tab}{\hspace*{2em}}
%
\newcommand{\addVspace}{\\[.5\baselineskip]}
\newcommand{\CR}{$\hookleftarrow$}
\newcommand{\CTRL}{$\uparrow$}
\newcommand{\ESC}{{\sc esc}}
%
\def\caret{{\tt\char'136}}
%
\newenvironment{compact}% single spaced environment
{\renewcommand{\baselinestretch}{1}\footnotesize\normalsize}%
{\renewcommand{\baselinestretch}{1.5}\footnotesize\normalsize}
%
%\newenvironment{display}[1]{#1\begin{tabbing}\hspace*{3em}\=\+\kill}%
%{\end{tabbing}}
\newenvironment{display}{%
\begin{list}{}{\setlength{\leftmargin}{3em}}
\item[]}{\end{list}}
%
\newenvironment{code}{\cd\begin{tabbing}%
\hspace*{3em}\=mmm\=mmm\=mmm\=mmm\=mmm\=mmm\=mmm\=mmm\=\+\kill}%
{\end{tabbing}}


\begin{document}
%\tracingall

\chapter{The C Program Structure}


A C  program consists  of {\kc preprocessor commands}, {\kc
comments}, some {\kc top-level declarations}, and  one or  more {\kc
functions},  possibly spread over more than one file. One  of the 
functions must  be called  {\cd main}; this  is always  the first
function to  be executed.  Functions may refer to (or call) one
another, with the exception of  {\cd main}, which cannot be called by
any other function. The order of the functions  in the  file is
incidental; they are typically grouped together according to their
purpose.  The next section describes the preprocessor and all but the
first paragraph may be skipped on the first reading of this book.

\section{The Preprocessor}

The C preprocessor is  a simple  macroprocessor, controlled  by
special  command  lines within the source program.  Each command line
begins with a hash ({\cd \#}); in most compilers, this must be the
first character on the line. The common commands are:
 
The preprocessor  executes all  such  commands  it  finds  in the 
source file, removing them as it does so. Preprocessor commands can
be split over more than one line by ending each line except the last
with a backslash (\verb+\+).

\subsection{Macro Definitions}

Macro definitions  provide a  way of  naming useful constants, as well
as improving readability  and abbreviating  common short  pieces of
text. A macro definition has the form
where $[]$ indicate optional parts.

Once a macro has been defined, and until the end of the file containing
the definition or  an {\cd \#undef}  command for  the same  macro,
{\em all occurrences of the macro name  in the  file will  be replaced by
the macro body.}  If arguments are used, these will be substituted into
the macro body.

Some examples will help. Firstly, naming useful constants:
These are  reasonably obvious  examples. As  far as  the  first  two 
are concerned, C provides no reserved words for the boolean values
true and false, but treats  zero as  being false  and everything else
as being true, with one being the canonical true  value. Thus,
by defining  these macros, we can use {\cd TRUE} and  {\cd FALSE} in
our program  instead of  the more  obscure 1  and  0.  The {\cd
BUFFERSIZE} example  illustrates a technique for improving
readability (as with {\cd TRUE} and  {\cd FALSE}), as  well  as ease 
of  modification.  This  constant  would presumably be  used when 
declaring line  buffers, for example. If the desired size of  the
line  buffer changed,  it would  only be  necessary to change the
macro definition,  instead of  searching for  the number 80 and
changing it to the new number throughout the file, which is an
error-prone method.

Examples of macros taking arguments are:
 Macros thus  offer a  simple alternative  to functions\footnote{A
function in C is a group of statements associated with a name, by
which they can be called and executed.}. The difference is that a 
function is  compiled once, and every time it is referenced, a call is
made to  that same  single function.  A macro,  however, is simply
{\em a textual replacement before compilation even  begins\/}. Thus 
every time the macro is used, the compiler will generate code  for
the  whole macro. The advantage of macros is that they  make
programs  easier to  read and  modify. In  general, the use of
complex macros  is discouraged,  as a  function would  almost always
serve the purpose better. However, a macro is generally better than a
trivial function.

Macros can  refer to  one another,  as was  shown  above:  when  a 
macro reference occurs  in a  program,  the  macro  body  is 
substituted,  and  the preprocessor carries on from the
{\em beginning\/} of the substituted text. However, a macro cannot refer to
another macro which has not yet been defined.

{\bf Warning:} You  should use  parentheses in  macros wherever 
they could  be ambiguous. This is surprisingly often. Take, for
example, the following macro:
To remove a definition, use the {\cd \#undef} command as in
     By convention upper case letters are usually used for macros that
are simply constants (e.g.\ PI). Macros that take arguments are often
regarded as being similar to functions, and the convention then does not
apply.


\subsection{Including Text From Other Files}

The command
causes  the  entire content  of  the specified file  (the {\kc
include file\/}) to be processed as  if it had appeared  in  place of
the command. It may be  that the included file itself contains {\cd
\#include} commands; the depth to which this is possible varies from
system to system.

     File names  must be  in double quotes ({\cd ""}) or angle brackets
({\cd <>}). The quotes are used  for user  files, the  angle brackets 
for  UNIX  system  files.  The difference is that files enclosed in
quotes are first searched for in the same directory as  the program, 
and then  in the standard directories, while files enclosed in angle
brackets are searched for in the standard directories only.  Under
UNIX, the standard include directory is {\cd /usr/include}.

There is  a  standard input/output header file, called {\fn stdio.h},
which must be included  by any  C program that performs input or
output. As it is a short file, and defines a few useful constants, it
is best to always include it. You should print  a copy of your
system's {\fn stdio.h} file and examine it to see what constants it
defines (for example, some define the {\cd TRUE} and {\cd FALSE}
constants we defined above, while others do not).


The command
causes  the  entire content  of  the specified file  (the {\kc
include file\/}) to be processed as  if it had appeared  in  place of
the command. It may be  that the included file itself contains {\cd
\#include} commands; the depth to which this is possible varies from
system to system.

     File names  must be  in double quotes ({\cd ""}) or angle brackets
({\cd <>}). The quotes are used  for user  files, the  angle brackets 
for  UNIX  system  files.  The difference is that files enclosed in
quotes are first searched for in the same directory as  the program, 
and then  in the standard directories, while files enclosed in angle
brackets are searched for in the standard directories only.  Under
UNIX, the standard include directory is {\cd /usr/include}.

There is  a  standard input/output header file, called {\fn stdio.h},
which must be included  by any  C program that performs input or
output. As it is a short file, and defines a few useful constants, it
is best to always include it. You should print  a copy of your
system's {\fn stdio.h} file and examine it to see what constants it
defines (for example, some define the {\cd TRUE} and {\cd FALSE}
constants we defined above, while others do not).

     

\subsection{Conditional Compilation}

     Conditional commands  allow parts  of  the  program  to  be 
included  or excluded from  compilation, depending  on some 
condition. As the preprocessor must be  able to determine the
condition, only simple constant expressions are allowed, with the
usual C convention that if the expression has value zero, it is
considered false, otherwise true. The main use of conditional
compilation is to make programs more portable, and to (conditionally)
include debugging code in a program.  The general form of a
conditional is:
Constant expressions will be described in detail later; for now it is
enough to say that {\em constant\_expression\/} is an expression that can be
evaluated immediately from  the   information  known   by  the 
compiler  (or,  in  this  case,  the preprocessor). If  the
expression  is true,  {\ms text\_section\_one\/} is included by
the preprocessor  and {\ms text\_section\_two\/}  is ignored; if
the  expression  is false, it is the other way around.

Two other forms of {\cd \#if} are available, namely:
These test  whether the  macro name  has been  defined or  is 
undefined, respectively. This could be of use, for example, if we are
unsure whether {\cd TRUE} has been defined. We could use:
Comments in  C are  enclosed within  {\cd /*} and {\cd */}.  Comments
can continue over more than one line. You should not use comments within
comments; {\cd /*} is not recognised within a comment,
so (usually) the first {\cd */} found indicates the end of comments. 
However, this is compiler dependent and not all compilers may act
this way.

\subsection{My Own Heading}

A function  is a  collection of  {\kc declarations\/} and  {\kc
statements\/} enclosed in braces {\cd \{\}}, and preceded by a {\kc
function name}. The minimal C function, which does
nothing(it contains no declarations or statements), is:
 The parentheses  following the  name are  necessary;  they  are  used
to enclose the  names of  the function's  arguments, if any. Even if
there are no arguments, they  serve to  indicate to the compiler that
the name is that of a function, and not a variable. Function names
must be valid C {\kc identifiers\/}.
     Identifiers in  C are  names for functions and variables. They
may be any combination  of  upper  and  lower  case letters, 
digits  (0--9),  and  the underscore character (\_), the only
restriction being that they may not begin with a  digit (this  is so
that the compiler can recognise them as identifiers and not numbers).
In older compilers only the first eight characters are significant, thus the
compiler  might not  distinguish between  {\cd long\_name\_1} and  
{\cd long\_name\_2}. The case of the letters is significant,  thus 
{\cd Total}  and  {\cd total}  are  different identifiers (it  is
considered  bad style  to have two identifiers that differ only in
their case).

     All identifiers are characterised by two attributes: their {\kc
type\/} and their {\kc storage class\/}.  The  type  represents the 
identifier type,  for  example,  {\cd int} (integer), {\cd float} 
(floating point  number), {\cd char} (character), or function. The
storage class  determines the  identifier's {\kc scope\/}  and {\kc
visibility\/}; that is, how accessible the identifier is from other
parts of the program. The main storage classes are:
     Conditional commands  allow parts  of  the  program  to  be 
included  or excluded from  compilation, depending  on some 
condition. As the preprocessor must be  able to determine the
condition, only simple constant expressions are allowed, with the
usual C convention that if the expression has value zero, it is
considered false, otherwise true. The main use of conditional
compilation is to make programs more portable, and to (conditionally)
include debugging code in a program.  The general form of a
conditional is:
Constant expressions will be described in detail later; for now it is
enough to say that {\em constant\_expression\/} is an expression that can be
evaluated immediately from  the   information  known   by  the 
compiler  (or,  in  this  case,  the preprocessor). If  the
expression  is true,  {\ms text\_section\_one\/} is included by
the preprocessor  and {\ms text\_section\_two\/}  is ignored; if
the  expression  is false, it is the other way around.

Two other forms of {\cd \#if} are available, namely:
These test  whether the  macro name  has been  defined or  is 
undefined, respectively. This could be of use, for example, if we are
unsure whether {\cd TRUE} has been defined. We could use:
Comments in  C are  enclosed within  {\cd /*} and {\cd */}.  Comments
can continue over more than one line. You should not use comments within
comments; {\cd /*} is not recognised within a comment,
so (usually) the first {\cd */} found indicates the end of comments. 
However, this is compiler dependent and not all compilers may act
this way.


A function  is a  collection of  {\kc declarations\/} and  {\kc
statements\/} enclosed in braces {\cd \{\}}, and preceded by a {\kc
function name}. The minimal C function, which does
nothing(it contains no declarations or statements), is:
 The parentheses  following the  name are  necessary;  they  are  used
to enclose the  names of  the function's  arguments, if any. Even if
there are no arguments, they  serve to  indicate to the compiler that
the name is that of a function, and not a variable. Function names
must be valid C {\kc identifiers\/}.
     Identifiers in  C are  names for functions and variables. They
may be any combination  of  upper  and  lower  case letters, 
digits  (0--9),  and  the underscore character (\_), the only
restriction being that they may not begin with a  digit (this  is so
that the compiler can recognise them as identifiers and not numbers).
In older compilers only the first eight characters are significant, thus the
compiler  might not  distinguish between  {\cd long\_name\_1} and  
{\cd long\_name\_2}. The case of the letters is significant,  thus 
{\cd Total}  and  {\cd total}  are  different identifiers (it  is
considered  bad style  to have two identifiers that differ only in
their case).

     All identifiers are characterised by two attributes: their {\kc
type\/} and their {\kc storage class\/}.  The  type  represents the 
identifier type,  for  example,  {\cd int} (integer), {\cd float} 
(floating point  number), {\cd char} (character), or function. The
storage class  determines the  identifier's {\kc scope\/}  and {\kc
visibility\/}; that is, how accessible the identifier is from other
parts of the program. The main storage classes are:
\end{document}


