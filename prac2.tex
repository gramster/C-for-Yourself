Practical Session 2

1) Modify the detab program so that we can set the value of TABSTOP at
compile-time. The program should default to TABSTOP 8 if not otherwise
specified.

2) Test this program with TABSTOP defined to be 4.

3) Modify the detab program so that it prompts for and reads the
tabstop value. Test this program.

4) Change the detab program so that the main loop is put into a
function called 'detab'. This function should take an integer argument
'tabstop'. The main function should just prompt for and read tabstop,
and call detab.

5) Change the detab FUNCTION so that it returns the number of tab
characters in the file. main() should print this out at the end.

6) Change the detab FUNCTION so that it returns the (number of spaces
inserted instead of tabs - number of tab characters in file). Change
main() to print this out at the end, using the format "%d more
bytes\n".

