\cleardoublepage
\chapter{ANSI extensions to K\&R C}
\index{ANSI C}
\index{K\&R C}
This chapter considers some of the ANSI extensions to Kernighan and
Ritchie C. These extensions are only supported by fairly recent
compilers.

\section{Preprocessor Enhancements}
\index{cpp@{\cmd cpp}!ANSI enhancements to}

The preprocessor operator {\cd \#\#} allows concatenation of
preprocessor arguments. Two simple examples should suffice
as illustration.  Given
\begin{code}     
     \#define JOIN(a,b) \tab \=a\#\#b \\
     \#define PLURAL(a) \>  a\#\#s 
\end{code}
\noindent
then
\begin{display}\cd
\begin{tabular}{@{}ll@{}}
\rm\bf The call & \rm\bf Expands to \addVspace
     JOIN(cat,mat)   &    catmat\\
     PLURAL(cat)     &    cats
\end{tabular}
\end{display}
\noindent
     The operator {\cd \#} converts a macro argument to a string. For
example, given the definition
\begin{code}
     \#define WRITE(s) \tab printf(\#s)
\end{code}
\noindent
then {\cd WRITE(cat)} will expand to {\cd printf("cat")}
     
     The {\cd \#error} preprocessor command causes the preprocessor
(and compiler) to print the specified message and stop compiling.
For example,
\begin{code}     
     \#ifndef NULL \\
     \#error "stdio.h must be included" \\
     \#endif
\end{code}
\noindent
     There is also a {\cd \#pragma} preprocessor command used to pass
information to the compiler, but that is beyond the scope of this
book.  

The ANSI standard includes several predefined macros:
\begin{display}
\begin{tabular}{@{}ll@{}}
\cd  \_\_DATE\_\_ &  the date of compilation \\
\cd  \_\_TIME\_\_ &  the time of compilation \\
\cd  \_\_FILE\_\_ &  the name of the source file being compiled \\
\cd  \_\_STDC\_\_ &  a nonzero constant \\
\cd  \_\_LINE\_\_ &  the current line number
\end{tabular}
\end{display}
\noindent
 The first four of these are strings; the last two are integers. The
\_\_STDC\_\_ macro is used to indicate that the version of C being used
complies with the ANSI standard; if this macro is not defined, then your
C compiler does not comply with the standard.
     
\section{Type Modifiers -- {\cd signed}, {\cd const} and
{\cd volatile}}
\index{signed@{\cd signed}}
\index{const@{\cd const}}
\index{volatile@{\cd volatile}}
\index{types!ANSI enhancements to}
\index{constants!using {\cd const}}
In addition to the {\cd unsigned} type modifier, a {\cd signed}
modifier has been added. This was previously unnecessary, as the
default was always to use signed representation for characters and
integers. The use of {\cd signed} allows this to be made explicit.
Some compilers also have options allowing the user to specify whether
the default {\cd char} type should be signed or unsigned, and in the
latter case, the {\cd signed} modifier is needed to override the
default.

ANSI C supports constants by using the type modifier {\cd
const}. In K\&R C, constants were always handled by using preprocessor
macros. The {\cd const} type modifier is more flexible, allowing
arbitrary type constants (for example, constant arrays can be
declared). Obviously, a constant must be initialised when it is
declared.

Pointer {\cd const}s are also supported. In this case, the pointer
points to a fixed address.  The object at that address may be
modified, but the pointer cannot be changed.
For example:
\begin{code}
const unsigned long *clock\_tick = (unsigned long *)0x512;
\end{code}
\noindent
 declares a constant pointer which points to a {\cd long} situated at
address $512_{16}$ (which happens to be the clock tick counter on an
IBM PC). The following would then be legal under
MS-DOS\footnote{MS-DOS is a trademark of Microsoft Corporation.}:
\begin{code}
*clock\_tick = 0l; \tab	/* Reset the tick counter */
\end{code}
\noindent
since it is assigning to a variable which is pointed to by a constant.
However, the following is {\bf not} legal:
\begin{code}
clock\_tick = NULL;
\end{code}
\noindent
as it is attempting to assign to a constant.

The type modifier {\cd volatile} tells the compiler not to perform
any optimisations on the specified variables. The reasons for using
this modifier are beyond the scope of this book.

\section{Function Prototypes}

\index{functions!ANSI enhancements to}
\index{functions!prototypes}
A characteristic of K\&R C is the poor type checking that is performed
by the compiler. In other words, the compiler will typically not report
any problems with the following:
\begin{code}
void funk(a,b) \\
\> char a;		\\
\> int b;		\\
\{  				\\
 \> \vdots 		\\
\} 				\addVspace		

long x, y;    \\
\vdots		  \\
funk(x,y);	  \\
\vdots
\end{code}
\noindent
 This is obviously incorrect, as we are passing a total of eight bytes
(two {\cd long}s) as parameters to a function which expects four bytes (a
{\cd char}, which is cast to {\cd int}, and an {\cd int}. It is this
vulnerability to errors which resulted in the development of C program
checkers such as {\cmd lint}.

ANSI C supports a different syntax for function declarations which
allows strong type checking and avoids these types of errors. Pascal
programmers will find the new syntax, called {\kc function
prototyping}, much more familiar. To declare a function using
prototypes, the parameter types are specified within the argument
list itself. The function must be declared before any in order calls for the
type checking to be used. This may require forward declarations in
some cases. Often, each global function has a prototype declaration
in a header file, which is included by files which call the
functions. For example, compilers that support ANSI C will use
function prototypes for the input and output routines, and declare
these prototypes in the {\fn stdio.h}\index{stdio.h@{\fn stdio.h}} file. Thus, 
by including {\fn stdio.h}, all calls to standard I/O functions have strong type
checking applied.

The use of prototypes is best illustrated by an example. Consider the
function declaration:
\begin{code}
void funk(a,b,c)	\\
\> int a;			\\
\> char *c;			\\
\> float b[];		\\
\{						\\
 \> \vdots			\\
\}
\end{code}
\noindent
In K\&R C, a forward declaration of such a function would look like
\begin{code}
void funk(a,b,c);
\end{code}
\noindent
Notice that this gives no information about the parameter types.
Using prototypes, the function would be declared as:
\begin{code}
void funk(int a, float b[], char *c) \\
\{												 \\
 \> \vdots									 \\
\}
\end{code}
\noindent
and a forward declaration would be
\begin{code}
void funk(int a, float b[], char *c);
\end{code}
\noindent
 giving complete type information on the function type as well as the
parameters. Our original erroneous example would immediately be
reported by the compiler if we had used prototypes. In fact,
prototypes eliminate much of the need for checkers such as {\cmd
lint}.

