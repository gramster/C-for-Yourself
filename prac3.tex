Practical Session 3

We now  begin to  build functions  which will eventually form an interactive,
menu-driven file-examination  utility which  will allow  us to  examine files
screen-by-screen in different formats.

In order  both to  simplify  writing  the  program  and  to  ensure  as  much
uniformity as  possible, the  design of  the program  has already  been done.
Roughly, the program will operate as follows:

1 - the top-level menu will ask for a file name, or 'Q' to quit.

2 -  if a file name is entered, the program will attempt to open the file for
       reading

3 -  if it  fails, an  error message  will be generated, and we return to the
       top-level menu (1)

4 - otherwise, the first screen of the file will be displayed as text

5 - at the end of the screen, the program will wait for a response

6 -  if the  response is  just a  <return>, we  display the  next screen, and
       continue at (5)

7 -  if the  response is  '-<return>', we  display the  previous screen,  and
       continue at (5)

8 -  otherwise we  display a menu (the file menu) allowing the user to change
       the view of the file (choice of text, or octal, decimal or hexadecimal
       dump), or  go to  an option menu (9) or the top-level menu (1). At the
       end of the file menu we redisplay the screen, and continue at (5).

9 -  if the  option menu  is selected, we display this menu, which allows the
       screen size  to be  changed, and  line numbers  (text view) or offsets
       (dumps) to  be turned  on or  off. At  the end of the options menu, we
       return to the file menu (8)

An include  file, view.h,  is available for use in the program. This contains
the following definitions:

#include <stdio.h>
#include <ctype.h>

/******************************/
/* Useful macros for booleans */
/******************************/

#define SWITCH(s)   (s?"on":"off") /* "on" if s true; else "off" */
#define TOGGLE(s)   {s=1-s;}       /* Switch true/false          */

/**********************/
/* Menu return values */
/**********************/

#define QUIT        0         /* Return value for quit      */
#define CONT        1         /* Return value for continue  */

/*****************/
/* Views of file */
/*****************/

#define OCTAL       8         /* Base 8 view           */
#define DECIMAL     10        /* Base 10 view          */
#define HEXADECIMAL 16        /* Base 16 view          */
#define TEXT        26        /* ASCII Text view       */

/******************/
/* Line numbering */
/******************/

#define NUMWIDTH    (numbr?6:0)    /* Columns of screen used for */
                                   /* line number (6 if numbering*/
                                   /* is on; otherwise 0.        */
#define NUMFMT      "%4d> "        /* Format for printing line   */
                                   /* numbers (can change this   */
                                   /* easily if a macro)         */

/****************/
/* Clear Screen */
/****************/

#define CLEARSCREEN printf("%c[H%c[J",27,27) /* This works! */

/****************************************************************/


1) Begin  by editing  the file menu.c. This file already exists, and contains
functions to  print the three menus, as well as a function void pause() which
causes a  short delay  (this is useful for after printing error messages, and
so on).

2) Add  a function, error(msg) char *msg; which prints ERROR followed by msg,
beeps and pauses.

3) Add a function getvalue() which prints a message requesting a value, reads
an integer value, and returns this value.

4) Look  at the  menu printing functions. Each begins by printing the name of
the menu,  followed by Enter: and a list of options. Write a function heading
to print  a large  heading (whatever you want to call your program), and call
this function  at the  start of  each menu. Now, change the functions so that
heading prints the menu name and ENTER: itself.

5) Save  the file.  Edit a  new file, main.c. Write a main() function to test
the functions you have written. Save this file.

6) Compile and run the program.
