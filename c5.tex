
\section{Structures}
\index{structures}
     A {\kc structure\/}  is a  collection of  named components of
various types. They are thus  similar to  records in  a data
file.  For example,
\begin{code}
struct\{  \+\\
 char *name; \\
 int dept; \\
 long tel; \-\\
\} staff[50];
\end{code}
\noindent
 defines {\cd staff} as an array with 50 elements, where each element
consists of an employee name, department number, and telephone
number.

Each time we define a structure type\index{structures!types}, we introduce a new type
different from all others. If we wish to declare more than  a few
structures of a particular type, we must name the
structure type  so that  we can  refer to  it again.  We can do this
using a 
{\cd typedef}\index{storage classes!typedef@{\cd typedef}}
\index{typedef@{\cd typedef}}, 
for example,
\begin{code}
typedef struct\{  \+\\
 char *name; \\
 int dept; \\
 long tel; \-\\
\} an\_employee; \addVspace
an\_employee staff[50];
\end{code}
\noindent
Alternatively we can name the structure with a 
{\kc structure tag}\index{structures!tags}.
For example:
\begin{code}
struct employee \{  \+\\
 char *name; \\
 int dept; \\
 long tel; \-\\
\}; \addVspace
struct employee staff[50];
\end{code}
\noindent
 gives the name {\cd employee} to the structure we used earlier, and
then declares an array of 50 of these structures. We could also have
used:
\begin{code}
struct employee \{  \+\\
 char name; \\
 int dept, \\
 long tel; \-\\
\} staff[50];
\end{code}
\noindent
for the same effect. The general form of a structure definition is
therefore
\begin{code}
struct {\ms structure\_tag\/} \{  \\
\> {\ms structure\_description\/}  \\
\} {\ms variable\_list\/};
\end{code}
\noindent
 where {\ms structure\_tag\/} and {\ms variable\_list\/} are optional
(although it would be pointless to leave them both out).

     Structures can  be  forward  referenced  
\index{forward references}provided  their  sizes
are  not required. This allows structures to refer to one another,
for example:
\begin{code}
struct P\{ struct Q *qp; \}; \addVspace

struct Q\{ struct P *pp; \};
\end{code}
\noindent
     To access  the components  of a 
structure\index{structures!component selection}
\index{unions!component selection}, C provides the two
operators: {\cd .} ({\kc direct component  selection\/}) and {\cd ->}
({\kc indirect component selection\/}).  The first is used  on
structures, the second  on  pointers  to  structures.  Using  our
{\cd employee} example above, we can write:
\begin{code}
  struct employee x, *y, z;   \addVspace

  y = z; \tab\tab\tab /* y now points to z */ \\
  x.name = "The Boss";	   \\
  x.dept = 1;		   \\
  (\&x)->tel = 666;	   \\
  y->name = "Joe Soap";	   \\
  (*y).dept = 99;
\end{code}
\noindent
     Structures can be initialised; for example:
\begin{code}
struct S\{ int a; char *b; \} x = \{1,"abcd"\};
\end{code}
\noindent
 declares a variable {\cd x} with structure {\cd S} and initialises
{\cd x.a} to {\cd 1}, and {\cd x.b} to point to the string {\cd
"abcd"}.

     Older C  compilers only  allow pointers to structures to be
passed to and from 
functions\index{functions!arguments}\index{functions!returning values},  
and do not allow structure assignment
(this must be done on a com\-po\-nent-by-component basis).  Newer
compilers,  however, allow structures to be assigned to, and passed to
and fro.  Structures may  not be  compared for equality (if you want
to do this, you must write your own function to do it on a
com\-po\-nent-by-component basis).

     Structure components  may have  any type  except function types.
Note, however, that a structure component may have the type {\ms
pointer to function}.
Structures may not contain  instances of themselves (an infinite
regression!) but may contain pointers to  instances of  themselves.
The  only  overloading  restriction  on component names  is  that
the  component  names  within  any  one  particular structure
definition must be distinct.

     Small integer components can be packed into small spaces using
{\kc bitfields\/}\index{structures!bitfields}\index{bitfields}.  The 
component  name in  the definition is followed
by a colon ({\cd :}) and the number of bits  to be  used. Unnamed
bitfields can  be used  for padding. The use of bitfields is  likely
to be non-portable; the main use is in matching some data structure
exactly.  As an example, a structure to match the general form of an
(assembler) instruction of the Motorola MC68000 microprocessor chip is:
\begin{code}
struct instr \{ \+\\
  int \=op:4, \\
  \>  reg1:3, \\
  \>  mode1:3,\\
  \>	mode2:3,\\
  \>	reg2:3;   \-\\
\};
\end{code}
\noindent
     Such a  structure would  occupy 16  bits, which  is  the  length
of  the standard MC68000  instruction. Thus  our structure  would
match an instruction exactly, and give us easy access to the various
fields of such an instruction.

Something worth noting is that the {\cd sizeof} operator cannot be
applied to bitfields. The operator returns a size in bytes, so its
application to bitfields is meaningless. Likewise it can be dangerous
to apply the {\cd \&} operator to a bitfield. Arrays of bitfields are
not allowed. Unions may not contain bitfields, but may contain
structures which contain bitfields, so this is no real restriction.

As most  16-bit computers  do not allow 16 and 32-bit data items
(such as {\cd short} and  {\cd long int}s)  to be  on odd-address
boundaries, while bytes (that is, {\cd char}s) may, a structure may
contain holes\index{structures!holes in}\index{unions!holes in}. For example:
\begin{code}
struct eg \{ int a; char b; int c \};
\end{code}
\noindent
     will not  be able  to be  packed into 5 bytes (assuming 16-bit
ints), but will need  6 bytes.  Thus a one byte hole will occur in
the structure (this is why we  must compare  structures for  equality
component-for-component, as the contents of  the holes  are
undefined).  Thus, {\cd sizeof}  must always be used when determining the
size of a structure.

     Even rearranging the above definition to read:
\begin{code}
struct eg \{ int a,c; char b; \};
\end{code}
\noindent
     will not help, as the compiler works out the size of a structure
based on the assumption  that you may declare arrays of the
structure. Conceptually, we can think  of this  as: if  we stored
three of these structures in memory one after the  other, how  much
memory  would the  middle one have to occupy if no invalid addresses
are allowed. No matter how we arrange the above example, the answer
is always 6 bytes.

\subsection{I/O with Structures}
\index{structures!I/O}

It is possible to read and write structures, unions, and so on, using
the {\cd fread} and {\cd fwrite} library functions. However, things
are not always as simple when we wish to read a file with a
particular format into an array of records. Consider a data file
which has a strict line layout, such as shown in the following table.
\begin{center}
\begin{tabular}{l|l}
Position & Data\\
\hline
1-13 & Employee's ID number\\
14-32 & Employee's Name\\
33-40 & Employee's Salary\\
\end{tabular}
\end{center}
Assume we want to read the employee's birth dates (obtained from
their ID number) and names into an array of structures defined as:
\begin{code}
struct BirthData \{    \\
\>	int day;    \\
\>	int month;    \\
\>	int year;    \\
\>	char name[20];    \\
\};
\end{code}
\noindent
One problem that may occur is with the names -- they can contain
whitespace. If you use {\cd scanf} with a {\cd \%s} conversion
specifier for the string, the read will terminate at the first space
character found in the person's name. This is not what we want. The
solution is to use the {\cd \%c} conversion specifier instead, with
the code:
\begin{code}
struct BirthData buf; \\ \addVspace

scanf(\="\%2d\%2d\%2d\%*7c\%19c\%*8c", \\
\>	\&buf.year,\&buf.month,\&buf.day,buf.name);\\
$\vdots$
\end{code}
\noindent
Note however that unlike with {\cd \%s}, {\cd scanf} does not add the
terminating {\cd `\verb+\+0'} byte automatically if {\cd \%c} is used.
Also, {\cd \%c} reads in and keeps all trailing white space, so that
{\cd strlen(buf.name)} would always be nineteen, regardless of the
name read (of course, if we haven't added the terminationg byte
ourselves, then the string may well be longer than 19 characters).

If you want to read and write complete structures (as opposed for a
field-by-field method as used above), you should bear in mind that
the whole structure will be written, including any `holes'. You can
use the {\cd sizeof} operator to find out how many bytes are occupied
by a structure. This is the number of bytes that will be read and
written. The advantage of reading and writing whole structures is
that we are guaranteed that the field will be read back in correctly,
as a specific number of bytes will be written/read at a time. You
should open the file in binary mode, not text mode, to ensure that
this works correctly. Also bear in mind that structures which contain
pointers will be read in with the old pointer values -- if the memory
allocated for the structures that are read in is not at precisely the
same address as the structures that were written out, these pointer
fields will no longer be valid.

The following function and macros are useful for copying whole
structures\index{structures!copying}. They are independent of the 
structure to be copied:
\begin{code}
void \_scopy(dest, source, len) \\
\> char *dest, *source; \\
\> int len; \\
\{ \+\\
	if(  len > 0  )\{ \+\\
		while( len-- )\{ \+\\
			*dest++ = *source++; \-\\
		\} \-\\
	\} \-\\
\} \addVspace

\#define structcopy(d,s) \tab\tab\=\_scopy(\&d,\&s,sizeof(s)) \\
\#define pstructcopy(d,s)        \>\_scopy(d,s,sizeof(*s))
\end{code}
\noindent
{\cd structcopy} takes structures as arguments, while {\cd
pstructcopy} takes pointers to structures as arguments.

\subsection{Structures and Absolute Addresses}
\index{structures!absolute addressing}
Structures can be a convenient way to access tables at fixed
locations in memory. For example, IBM PC's have a table called the
disk base table at address {\cd 0xFEFC7}. We can define a structure
with fields corresponding to this table as:
\begin{code}
struct\{ \+\\
	char \=specify1, \+\\
		specify2,	\\
		wait\_time,	\\
		sector\_size,	\\
		last\_sector,	\\
		RW\_gap\_length,	\\
		data\_length,	\\
		format\_gap\_length,	\\
		format\_value,	\\
		settle\_time,	\\
		startup\_time;	\-\-\\
\} far *DskBaseTbl = 0xFEFC7;
\end{code}
\noindent
The actual details of this table, as well as the PC-specific
keyword {\cd far}, is beyond the scope of this discussion. Having
created our table structure, we can now access the elements of the
table using the usual structure pointer operations. (Hackers, take
note -- the address of the table as given, is the default table in the
ROM BIOS; thus you cannot change any of these values. The location of
the RAM version of the table is given by interrupt vector 1E).

\section{Unions}
\index{unions}
     Much of  what we  have said  about  structures  also applies
to the  type {\cd union}. Unions too have a number of components, and
have declarations which are (superficially, at least) the same as
structure declarations, except using the keyword {\cd union}  instead
of  {\cd struct}. However,  a union can only store one of its
components at  a time;  in other words, the components are overlaid in
memory.  This allows us to access areas of memory in different ways.
In other words, a C {\cd union} is similar to a Pascal variant
record.

     C does  not provide  a way  to determine which component  of a union
was last assigned; it  is up  to the  programmer to  keep track of
this. This
is unlike Pascal variant records,  which have a tag to indicate which
type is present. This can easily be  done in  C as well; C just
doesn't demand it (thus allowing greater flexibility). For example,
\begin{code}
  \#define STRING 0 \\
  \#define VALUE  1 \\
  \#define CHR    2 \addVspace

  struct reply \{			\+\\
       short tag;			\\
       union \{ 				\+\\
            char str[8]; 	\\
            long val; 		\\
            char ch;			\-\\
       \} is;				\-\\
  \} answer;
\end{code}
\noindent
defines a variable, {\cd answer}, to have the {\cd reply} structure.
The latter consists of two fields, {\cd tag}, and {\cd is}, where
{\cd is} can be any of three things: a {\cd char} array, a {\cd
long}, or a single character.

     To assign to this union, we can use:
\begin{code}
  answer.tag = STRING; \\
  strcpy( answer.is.str, "Yes");
\end{code}
\noindent
     or
\begin{code}
 answer.tag = VALUE; \\
 answer.is.val = 43;
\end{code}
\noindent
     or
\begin{code}
  answer.tag = CHR;     \\
  answer.is.ch = 'N';
\end{code}
\noindent
     The advantage of using tags is obvious: we can now determine what
type of data the  union is  holding by  examining {\cd answer.tag}.
We can then easily write functions such as:
\begin{code}
        print\_answer( a ) \\
        \> struct reply *a;  \\
		  \{ \+\\
	       switch( a->tag )\{ \\
          case STRING: 									\\
			 \> printf("\%s\verb+\+n", a->is.str); \\
			 \> break; \\
          case VALUE: 									\\
			 \> printf("\%ld\verb+\+n",a->is.val); \\
			 \> break; \\
          case CHR: 										\\
			 \> printf("\%c\verb+\+n", a->is.ch ); \\
			 \> break; \\
          default: 										\\
			 \> printf("ERROR - invalid reply tag!\verb+\+n"); \\
          \} \-\\
        \}
\end{code}
\noindent
     Bit fields\index{unions!restrictions}  are not  allowed in 
unions, and unions cannot be
initialised.  In other respects,  unions and  structs are
very similar.

     Note that structures and unions cannot be cast to other 
types\index{types!conversions}. However, pointers to structures and unions 
can be cast to pointers of
other types (including to structures and unions of other types). 
Thus the following
code is legal (although it is pointless and dangerous):
\begin{code}
struct A \{ int i; \}; \\
struct B \{ float f; \}; \addVspace

void jam\_float\_into\_int(val,a) \\
\>float val; \\
\>struct A a; \\
\{ \\
\>struct B *b = (struct B *)\&a; \\
\>b->f = val; \\
\}
\end{code}

\section{Enumerations}
\index{enumerations}
     Enumerations are  a recent  addition to  C and  are not
supported by many older compilers.  Although they are redundant,
since macros can be used for precisely  the same  purpose  (symbolic
names for  constants), they are more convenient and less error
prone.  Furthermore, symbolic debuggers\index{debugging}  give enumerations an edge over
macros, as they cannot deal symbolically with macros but can with
enumerations.

An enumeration  is a  set of  values represented  by identifiers
known as {\kc enumeration constants\/}\index{enumerations!constants}, and like structures  and
unions, an enumeration declaration  may include a tag name 
\index{enumerations!tag names}for the
enumeration.  For example
 \begin{code}
  enum committee \{ \\
  \> president, vice\_president, secretary, treasurer \\
  \} member; \addVspace

  member=secretary; \addVspace

  if( member==secretary ) printf("The Secretary\verb+\+n");
\end{code}
\noindent
 declares an enumerated type with name {\cd committee} and a variable
of that type named {\cd member}, assigns an enumerated constant {\cd
secretary} to the variable, and tests it for equality.

Enumerations are essentially lists of integer constants,
and the compiler simply replaces references to enumeration elements
with the appropriate constant value.
It happens  that C represents enumeration constants with integers,
numbered (by default)  from 0 
upwards\index{enumerations!constants!default values}. This  numbering implies
an order  on  the constants, so that in our example
\begin{code}
president < vice\_president < secretary < treasurer
\end{code}
\noindent
     (which might reflect  the real  power wielded  in a  committee).
This could have been done with macros, using
\begin{code}
  \#define VICE\_PRESIDENT \= 1 \kill
  \#define PRESIDENT       \> 0 \\
  \#define VICE\_PRESIDENT \> 1 \\
  \#define SECRETARY       \> 2 \\
  \#define TREASURER       \> 3 \addVspace

  unsigned short member;
\end{code}
\noindent
     It is possible to specify which  values must be associated with
enumeration constants\index{enumerations!constants!overriding default values}. 
The  rules used are  that the  first
constant  has value  zero, unless otherwise  specified, and  all
others  have a value of one greater than the previous constant,
unless otherwise defined. Thus if we declare:
\begin{code}
enum colour\{  \+\\
  red=2, green, blue, \\
  yellow=8, black, white \-\\
 \};
\end{code}
\noindent
     The associated values will be:
\begin{code}
red=2;   green=3;   blue=4;   yellow=8;  black=9;  white=10;
\end{code}
\noindent


\section{Functions Revisited}
\index{functions}
     A {\kc function  definition\/} (not  to be  confused with a
function declaration) introduces a new function and provides some
information about it, namely:
 \begin{itemize}
\item the type of the value returned, if any;

\item the types, names and number of formal parameters;

\item the visibility of the function outside of the file; and

\item the function body (code to be executed).
\end{itemize}
     Functions \index{functions!visibility}may only have the storage 
classes {\cd extern} (visible
outside file) or {\cd static} (not visible outside file).

The default  return type is {\cd int}\index{functions!returning values}. 
Functions may not return
arrays, structs\footnote{Modern C compilers might allow structs as
return values.}, or other functions, but may return any other type. For
example:
\begin{code}
float ( *(*fn(x,y))[] )()\\
\vdots
\end{code}
\noindent
     declares {\cd fn} as a  function taking  two parameters  {\cd x}
and {\cd y}, and returning a pointer to an array of pointers to
functions returning {\cd float}s.

     The order  of the  arguments in the formal parameter list and in
calls to the function  must agree  (although C  does not actually
check this); however, the order  of the  formal parameter
declarations  is  unimportant.  The  only explicit storage  class
specifier  allowed for  formal parameters is {\cd register} (the
default is {\cd auto}).  Parameters may  be of  any type  except {\cd
void}  and ``function returning...'' (the latter will  simply be
treated as function calls, with the returned value used as the actual
argument).  To repeat, however, pointers to functions may be used as
parameters.

     A {\cd return}  statement without  an expression  is allowed
(although if  the function is  not of  type {\cd void},  the
returned  value is  unpredictable).  This allows us  to request  a
function to be exited before the end. If the function is of  type
{\cd void},  it is an error to have an expression in a {\cd return}
statement.  Furthermore, {\cd void}  functions may never be called in a
context which requires a value. Thus:
\begin{code}
     extern void factor(); \\
     \vdots    \\
     a = 2*factor();
\end{code}
\noindent
     is not  allowed. In  all other  cases of  {\cd return}, C
attempts to cast the  return expression  to the return type of
the function; if this cannot be done, an error occurs.

\subsection{Variable Parameter Lists}
\index{functions!arguments!variable number of}
The discerning reader might be wondering by now how functions such as
{\cd printf} and {\cd scanf} work, as they take variable numbers of
parameters. C allows us to define such functions, and they can be an
extremely powerful construct. The only requirement is that there must
be some way of telling how many parameters there are, and what their types are. In {\cd
printf} and {\cd scanf}, this is done by examining the format string,
as it is assumed that there are as many arguments (of the apropriate types) as there are
conversion specifiers.

For example, say we wish to write a function which takes several
strings as parameters, and writes each one on a new line. The
simplest way to specify how many strings there are is to pass the
number of strings as the first parameter. The following example
function will do the task:
\begin{code}
\#include <varargs.h> 								\addVspace

void print\_strings(va\_alist)				   \\
\> va\_dcl \tab	/* Note: no semicolon */  \\
\{	\+														\\
	va\_list args;										\\
	int arg\_count;									\\
	char *string;										\addVspace

	va\_start(args);									\\
	arg\_count = va\_arg( args, int);			\\
	while( arg\_count-- )\{ \+						\\
		string = va\_arg( args, char* );			\\
		puts(string);								 \-\\
	\}														\\
	va\_end(args);									 \-\\
\}
\end{code}
\noindent
The example shown is for UNIX systems; other operating systems may
use slightly different forms. The header file {\fn varargs.h}
contains the definitions of {\cd va\_dcl}, etc. The meaning of each of
these is:
\begin{description}
\item[\cd va\_alist] is used as the parameter list in the
function header.
\item[\cd va\_dcl] declares {\cd va\_alist}. There should be
no semicolon.
\item[\cd va\_list] defines the type of a special variable
which is used to traverse the argument list. This variable should
never be modified directly.
\item[\cd va\_start] initialises the special variable to point
to the start of the argument list.
\item[\cd va\_arg] returns the next argument in the list. It
takes two arguments itself, the first being the special variable
described above, and the second being the type of the argument we are
about to get from the list.
\item[\cd va\_end] cleans up any mess at the end.
\end{description}

\subsection{Passing Arguments to {\cd main()} from UNIX}
\index{command line arguments}
\index{main@{\cd main}!passing arguments to}
As we have mentioned at the biginning,
each C program must have one and only one function called {\cd main}.
UNIX allows  arguments to  be passed  to {\cd main()}, which can 
be defined as:
\begin{code}
          main(argc,argv) \\
          int argc; \\
          char *argv[];
\end{code}
\noindent
 where {\cd argc} is  the  argument  count (the number of arguments
passed), and {\cd argv}  is the  argument vector, an array of
pointers to strings which hold the arguments.  The first of these
arguments is  always the  name of  the program.  For example, say  we
have  a program  {\cmd renum} which renumbers BASIC programs, and we
wish to be able to invoke it with
 \begin{display}\ms
{\cmd renum} input\_file output\_file start\_number step
\end{display}
\noindent
     In other  words, to  renumber the program {\fn myprog.b} starting
with line 100, using steps of 10, and sending the output to {\fn
newprog.b}, one would use:
\begin{display}\cmd
renum myprog.b newprog.b 100 10
\end{display}
\noindent
     If we  had declared  the {\cd main}  function as shown earlier, we
would have the following:
\begin{display}\cd
\begin{tabular}{@{}ll@{}}
  argc = 5  & \\
  argv[0]="renum"    &  argv[3]="100" \\
  argv[1]="myprog.b" &  argv[4]="10"  \\
  argv[2]="newprog.b" &
\end{tabular}
\end{display}
\noindent
     These arguments  can be used by {\cd main} instead of having to
prompt for and read them from the keyboard. This, in fact, is how the
UNIX commands such as {\cmd cc} get their arguments.

Conceptually, {\cd main} is the function with which execution begins. 
In
reality, all C implementations include a special piece of code, the
{\kc startup code}\index{startup code}, where execution actually begins. 
This code is
responsible for calling the {\cd main} function
and passing it the array of command line arguments, opening the {\cd
stdin}, {\cd stdout} and {\cd stderr} files, and setting up the program
stack and memory heap. It usuallly also cleans up after {\cd main}
exits, and passes back any values returned by {\cd main} or calls to
{\cd exit} to the operating system.

An example of the use of the startup code is as follows: on MS-DOS
systems, unlike UNIX, the command shell {\fn COMMAND.COM} does not
expand wild characters. In other words, if we run a program {\cmd
prettyprint} with the command line:
\begin{display}\cmd
prettyprint *.c
\end{display}
\noindent
then {\cd argv[1]} will have the value ``{\cd *.c}''. On a UNIX
system, the shell first expands the {\cd *.c} pattern into a list of
matching files, and passes these to the program. If we wanted this to
occur in MS-DOS, we would have to do the work of matching the files
in our {\cd main} function. An alternative provided by some C
compilers is to have two versions of the startup code -- one which
does no wildcard expansion, and one which does. The latter is
obviously bigger than the former, so by default the former will be
used. If we need wild character expansion, then we can link the
program with the alternative startup code. We then do not have to
make any changes to the {\cd main} function.

