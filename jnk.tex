\chapter{The C Program Structure}


A C  program consists  of {\kc preprocessor commands}, {\kc
comments}, some {\kc top-level declarations}, and  one or  more {\kc
functions},  possibly spread over more than one file. One  of the 
functions must  be called  {\cd main}; this  is always  the first
function to  be executed.  Functions may refer to (or call) one
another, with the exception of  {\cd main}, which cannot be called by
any other function. The order of the functions  in the  file is
incidental; they are typically grouped together according to their
purpose.  The next section describes the preprocessor and all but the
first paragraph may be skipped on the first reading of this book.

\section{The Preprocessor}

The C preprocessor is  a simple  macroprocessor, controlled  by
special  command  lines within the source program.  Each command line
begins with a hash ({\cd \#}); in most compilers, this must be the
first character on the line. The common commands are:
 
The preprocessor  executes all  such  commands  it  finds  in the 
source file, removing them as it does so. Preprocessor commands can
be split over more than one line by ending each line except the last
with a backslash (\verb+\+).



