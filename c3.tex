
\section{Type Conversions and Pointers to Functions}
\index{types!conversions}
\index{functions!pointers to}
\index{pointers!to functions}

     We have  seen that the differences between types in C are
actually small:  integers, pointers  and arrays  all have
similarities, and  characters can be treated as  small integers. This
fact means that converting between types in C is easy.

     C performs  many implicit  type conversions itself. For example,
an array declaration is  converted into a pointer declaration, and a
string constant is converted into  a pointer  to {\cd char}.  When a
function is defined, a  constant having the function name is actually
created. This constant is of type {\ms pointer to function returning 
t\/}, where  {\ms t\/} is  the return  type of  the function. Thus, a
function   call  in  C  actually  consists  of  a constant pointer name, 
followed  by parentheses which  may contain expressions representing
arguments. The effect of this is the following:
 \begin{itemize}
\item Any arguments are evaluated and saved on the program stack.

\item Execution  passes to  the code  in the  {\bf .text} section
\index{text@{\bf .text}}    being pointed to by the pointer.
\end{itemize}
     Furthermore, when calling a function with arguments, 
\index{arguments!type conversions of}
\index{types!conversions!of function arguments}{\cd char}s
and {\cd short}s are converted to  {\cd int}s, {\cd float}s  are
converted to {\cd double}s, and arrays are converted to pointers
before they  are stacked  and the  function  called.  This  makes
argument passing much simpler for the compiler to handle.

Note that we can declare a variable of type {\ms pointer to function 
returning a type t} with:
 \begin{code}
t    (*pf)();
\end{code}
\noindent
     We can then assign a value (the address of a function) to this 
variable and call the function which is being pointed to with:
\begin{code}
(*pf)({\ms arguments\/});
\end{code}
\noindent
Note the difference between the above declaration and {\cd t pf()}
and {\cd t *pf()}.  The latter two declarations both define functions
as opposed to a pointer to a function.  The first defines a function
returning a value of type {\cd t}, while the second defines a function
returning a value of type {\ms pointer to\/} {\cd t}.

     Assignment statements often cause type conversions. You can
mix most of the  types we  have so  far considered,  but
indiscriminate type mixing may cause errors, particularly if
converting long types to shorter ones, or {\cd float}s to {\cd int}s.

     You can specifically request C to perform a type conversion for
you\index{types!conversions!explicit}. This is known  as {\ms type  casting}. To  cast a  variable to a
type,  we  precede  the variable with the desired type name in
parentheses, for example:
\begin{code}
mmm\= truncate(n) \tab\tab \= \kill
float truncate(n) \>\> /* Remove fractional part */ \\
\> float n;         \> /* of a float             */ \\
\{ \\
 \>   return (float)((long)n); \\
\}
\end{code}


\section{More Complex Declarations}
\index{declarations!complex}
\index{arrays!multi-dimensional}
\label{complxDecl}
     We now  have sufficient  tools for more complex declarations.
Firstly, we can have  {\kc multi-dimensional arrays}.  There is  no
fixed limit on the number of dimensions (although  some systems may
have restrictions). To declare an array of more  than one  dimension,
we  simply add the size specifiers for the other dimensions. In other
words, to declare {\cd a}, a $d_{1} \times  d_{2}
\times \ldots \times  d_{n}$ array of type {\cd t}, we use:
\begin{code}
t  a[$d_{1}$][$d_{2}$]\ldots[$d_{n}$];
\end{code}
\noindent
     We access the element at $(i_{1}, i_{2}, \ldots, i_{n})$ with
\begin{code}
a[$i_{1}$][$i_{2}$]\ldots[$i_{n}$]
\end{code}
\noindent
     The array/pointer  equivalence still  holds; in particular, we
could also access the above element with:
\begin{code}
*( *(...*( *(a + $i_{1}$) + $i_{2}$)...) + $i_{n}$)
\end{code}
\noindent
     We have  seen before  that we  can access  variables declared in
other files by using {\cd extern} \index{extern@{\cd extern}}declarations. 
We can have any
number of {\cd extern} declarations for a variable, provided  there
is  one and only one declaration  of that  variable which  is the
{\kc defining declaration}\index{declarations!defining}; that is, there is a declaration
somewhere which creates the variable  and tells the  compiler
everything it needs  to  know about  that variable\footnote{The
defining declaration must not have the keyword {\cd extern} in it}. 
With  the exception of  defining declarations, we do not have to
specify the size of the first  dimension  of  a  C array, but all 
other dimensions  must  be specified.  This is  allowed since only
the defining declaration  causes storage to be allocated for the
variable.  Any other  declaration (including formal parameters in
functions) does not cause storage to be allocated; nevertheless,  the
dimension sizes must be known for all except the first dimension  so
that  the compiler  can tell how to access the elements of the
array.  The first  dimension is  unnecessary as  C does  not check
whether subscripts are out of range.
				  
To make this a bit clearer, consider the following code:
\begin{code}
extern int tbl[][ENTRYSIZE]; /* Not a defining declaration */
\addVspace

printdown(str) \\
char str[]; \tab  /* Function argument */ \\
\{ \\
 \> \vdots	\\
\} \addVspace

int tbl[ENTRIES][ENTRYSIZE]; \tab  /* defining decl. */
\end{code}
\noindent
 If we want to access the element {\cd tbl[i][j]}, the compiler
must be able to calculate the address of that element. The array is
stored in the row order:
\begin{display}\cd
  tbl[0][0], tbl[0][1], \ldots, tbl[0][ENTRYSIZE-1], tbl[1][0], \\
  \ldots, tbl[1][ENTRYSIZE-1], \ldots, tbl[ENTRIES-1][0], \ldots, \\
  tbl[ENTRIES-1][ENTRYSIZE-1]
\end{display}
\noindent
     To access the element {\cd table[i][j]}, the compiler evaluates:
\begin{code}
tbl + (i * ENTRYSIZE + j) * sizeof(int)
\end{code}
\noindent
     This is  the address  of the  required element. As can be seen,
we do not need the value of {\cd ENTRIES} to calculate it.

     Note the values returned by {\cd sizeof} in the following examples:
\begin{display}\cd
\begin{tabular}{@{}ll@{}}
  sizeof(tbl) &      {\rm is}  ENTRIES * ENTRYSIZE * sizeof(int) \\
  sizeof(*tbl) &    {\rm is}  ENTRYSIZE * sizeof(int)	    \\
  sizeof(tbl[0]) &  {\rm is}  ENTRYSIZE * sizeof(int)	    \\
  sizeof(**tbl) &   {\rm is}  sizeof(int) \\
  sizeof(tbl[0][0]) & {\rm is}  sizeof(int)	
\end{tabular}
\end{display}
\noindent
     We can form compound declarators from the array ({\cd
[]}), function ({\cd ()}) and pointer ({\cd *}) declarators. The only
restrictions are:
\begin{itemize}
\item {\cd void} can only be used for functions returning {\cd void};
\item arrays  of functions  are not  allowed, although  arrays of
pointers to functions are; 
\item functions may not return arrays or functions.
\end{itemize}
     Function and  array  declarators  have  higher  precedence  than 
pointer declarators, thus:
\begin{display}
\begin{tabular}{@{}lp{.6\textwidth}@{}}
 {\bf The declaration}  & {\bf Declares:} \addVspace
 {\cd float s();}      & a function returning a {\cd float} \\
 {\cd float *s();}     & a function returning a pointer to a {\cd float} \\
 {\cd float (*s)();}   & a pointer to a function returning a {\cd float} \\
 {\cd float *(*s)();}  & a pointer to a function returning a pointer
								 to a {\cd float} \\
 {\cd char (*p[])();}  & an array of pointers to functions 
 								 returning {\cd char} \\
 {\cd int (*pf)();}    & a pointer to a function returning an {\cd int} \\
 {\cd void *f();}      & {\bf illegal:} the type ``pointer to {\cd
								 void}'' is not allowed
\end{tabular}
\end{display}
\noindent
     While some of the more complex  declarations are not often used,
it is useful to know how to read and write them correctly.

     We are  now in a position where we know how to define and call
functions, how to  send and  receive values to and from functions,
and how to create many different types  of variables. It is time to
increase our ability to do things with these. Enter the C {\kc
operators}.


\section{Operators}
\index{operators}
     There are  places in  our program  that we  need values  (for
example, as arguments, or  for the  right-hand side  of
assignments).  Wherever a value is needed, we  can use  an  {\kc
expression}.  For  example,  a  function  call  is  an expression,
provided  that the function returns a value. In order to have more
complex expressions  than just  values, we  can use  operators.
We have already seen some of C's {\kc arithmetic operators}.  The
complete list is\index{operators!arithmetic}:
\begin{display}
\begin{tabular}{@{}ll@{}}
       {\cd a+b}  &   add {\cd a} and {\cd b} \\
       {\cd a-b}  &   subtract {\cd b} from {\cd a} \\
       {\cd a*b}  &   multiply {\cd a} by {\cd b} \\
       {\cd a/b}  &   divide {\cd a} by {\cd b} \\
       {\cd a\%b}  &  find the remainder if {\cd a} is divided by {\cd b} 
							 (i.e.\ {\cd a} {\it mod\/} {\cd b}) \\
       {\cd -a}   &   negate {\cd a} (i.e.\ unary minus)
\end{tabular}
\end{display}
\noindent
     C provides  a set  of {\kc bitwise operators\/} for manipulating
bit patterns in memory. The operators are\index{operators!bitwise}:
\begin{display}
\begin{tabular}{@{}ll@{}}
       {\cd a<<b} &   shift {\cd a} left by {\cd b} bits \\
       {\cd a>>b} &   shift {\cd a} right by {\cd b} bits \\
       {\cd a | b} &  find {\cd a} {\sc inclusive-or} {\cd b} \\
       {\cd a \caret b} &  find {\cd a} {\sc exclusive-or} {\cd b} \\
       {\cd a \& b} &  find {\cd a} {\sc and} {\cd b} \\
       {\cd   \verb+~+a}  &  complement {\cd a}
\end{tabular}
\end{display}
\noindent
     Similar to these, there are {\kc logical operators\/} for
combining truth values (Booleans).  Recall that C regards zero as
being  false and any other value as true, with  1 being  the  
canonical  true.  Using  this convention, truth  values  in  C  may 
be  any expression, although they will typically be the result of
some comparison. The logical operators are\index{operators!logical}:
\begin{display}
\begin{tabular}{@{}ll@{}}
 {\cd a \&\& b} & true if both {\cd a} and {\cd b} are true; false otherwise \\
 {\cd a || b} &  true if either {\cd a} or {\cd b} is true; false otherwise \\
 {\cd !a} &  true if {\cd a} is not true; false otherwise
\end{tabular}
\end{display}
\noindent
  In order to compare values  there are {\kc comparison operators}.
These are\index{operators!comparison}:
\begin{display}
\begin{tabular}{@{}lp{.6\textwidth}@{}}
 {\cd a > b}   &  true if {\cd a} is greater than {\cd b}; false otherwise \\
 {\cd a >= b}  &  true if {\cd a} is greater than or equal to {\cd b}; 
 						false otherwise \\
 {\cd a == b}  &  true if {\cd a} equals {\cd b}; false otherwise \\
 {\cd a != b}  &  true if {\cd a} is not equal to {\cd b}; false otherwise \\
 {\cd a <= b}  &  true if {\cd a} is less than or equal to {\cd b}; 
 						false otherwise \\
 {\cd a < b}   &  true if {\cd a} is less than {\cd b}; false otherwise
\end{tabular}
\end{display}
\noindent
     One should be particularly careful about the equality comparison operator
{\cd ==}, as it is easy to use {\cd =} by mistake. If this
happens, an assignment is done instead of a comparison, often with
disastrous results.

We have  already met  the {\kc indirection  operators\/} {\cd *}
and {\cd \&}, as well as the {\kc cast operator\/}, {\cd ({\ms
type\/})}, and  the basic assignment operator, {\cd =}. C allows us
to abbreviate a common form of assignment. The assignment
\begin{display}\ms
lvalue {\cd =} lvalue operation expression {\cd ;}
\end{display}
\noindent
     may be abbreviated to:
\begin{display}\ms
lvalue operation\/{\cd =} expression {\cd ;}
\end{display}
\noindent
     For  example,  {\cd a=a+5;} can be written as {\cd a+=5;}  An
{\ms lvalue\/} is an expression that refers  to memory  that can  be
examined and altered (the name comes from {\ms l\/}eft  {\ms value},  
as  only  {\em lvalues}  are  allowed  on  the  left-hand  side  of
assignments).  Thus  all variables  are lvalues,  as are  all pointer 
contents (that is, expressions of the form {\cd *p} where {\cd p} is
a pointer).

     The operations which are permissible as {\kc assignment
operators\/} are\index{operators!assignment}:
\begin{display}\cd
\begin{tabular}{@{}ccccc@{}}
 +=  & -=  & *=  & /= & \%= \\
 >>= & <<= & \&= & |= & \caret=
\end{tabular}
\end{display}
\noindent
     It often  happens in  a program that you wish to add or subtract
one from an lvalue.   C  provides special {\kc increment and 
decrement operators}\index{operators!increment}\index{operators!decrement}:
\begin{display}
\begin{tabular}{@{}ll@{}}
 {\cd ++a} & preincrement: add one to a and return its new value \\
 {\cd --a} & predecrement: subtract one from a, returning its new value \\
 {\cd a++} & postincrement: add one to a but return its old value \\
 {\cd a--} & postdecrement: subtract one from a but return its old value
\end{tabular}
\end{display}
\noindent
     These are  very useful  for loops and for manipulating array
indexes. For example:
\begin{code}
 printdown(s) \\
 \> char *s; \\
 /* Prints a string down the screen */ \\
 \{ \\
 \> while( *s ) printf("\%c\verb+\+n",*s++); \\
 \}
\end{code}
\noindent
(The {\cd while} statement is described in Section~\ref{while_st}. {\cd while}
loops terminate when the parenthesised expression has the value zero.)
     The increment and decrement operators  also support  C pointer
arithmetic\index{pointers!arithmetic}.   If we apply them to pointer types, the increment
or decrement is by the size of the object pointed to in bytes (i.e.\
by one ``object''). In our example above, the {\cd s} pointer is incremented by
single characters (bytes) at a time, moving through the string until it reaches
the null character, at which point the {\cd while} loop terminates.

Beware of using the increment and decrement operators in arguments to
macros. Because macros simply involves textual replacements, the
increment/decrement operator might be applied more times than what is
intended\index{macros!dangers of}.


\section{Precedence of Operators}
\index{operators!precedence of}
\index{operators!associativity of}
     Now that we have operators to work with, we can form complex
expressions.  However, we  must take into account  operator 
precedence and  associativity. The {\kc precedence} of  an
operator  indicates how tightly it groups. When there is a choice
in  the order  in which  to apply  operators, the  operator with
higher precedence is always performed first. For example,
multiplication has a higher precedence  than   addition,  as  it  is 
always  performed  first;  thus  the expression:
\begin{code}
a+b*c
\end{code}
\noindent
     groups as:
\begin{code}
a+(b*c)
\end{code}
\noindent
     When there is a choice of operators and they have the  same
precedence, we use the  associativity to  determine grouping. 
An  operator  which  is {\kc left-associative\/} always  groups left 
first,  while  a {\kc right-associative\/}  operator groups right
first. Thus:
\begin{code}
 a+b+c
\end{code}
\noindent
     groups as:
\begin{code}
 (a+b)+c
\end{code}
\noindent
     because addition is left-associative, while:
\begin{code}
 **p
\end{code}
\noindent
     groups as:
\begin{code}
 *(*p)
\end{code}
\noindent
     because  {\cd *}   (indirection)  is  right-associative.  A 
table summarising  the operators, their precedences and
associativities, is given in Table~\ref{precedence}.
\begin{table}\centering
\index{operators!precedence of}
\index{operators!associativity of}
\caption{\label{precedence}C Operators and Precedence.}
\par\medskip\noindent
\begin{tabular}{|llll|} \hline
 & & & \\
    Precedence & Associativity & Operators    &       Description \\ 
 & & & \\
\hline
 & & & \\
   16 & &           identifiers, constants & \\
   16 & Left  &             {\cd []}   &         array subscripting \\
   16 & Left   &  {\cd ()}      &         function call \\
   16 & Left   &  {\cd .}       &         direct selection \\
   16 & Left    &  {\cd ->}       & indirect selection \\
   15 & Right   &  {\cd ++\tab--} & post-inc/decrement \\
   14 & Right   &  {\cd ++\tab--} & pre-inc/decrement \\
   14 &         &  {\cd sizeof}     &         size \\
   14 & Right   &  {\cd  ({\ms type})}   &         type cast \\
   14 & Right   &  {\cd \verb+~+}         &         bitwise {\sc not} \\
   14 & Right   &  {\cd  !}          &         logical {\sc not} \\
   14 & Right   &  {\cd -}          &         arithmetic negation \\
   14 & Right   &  {\cd \&}         &          address of \\
   14 & Right   &  {\cd *}          &         contents of \\
   13 & Left    &  {\cd * \tab/ \tab\%} &        multiplicative \\
   12 & Left    &  {\cd + \tab-}     &        additive \\
   11 & Left    &  {\cd <<\tab>>}     &        left and right shift \\
   10 & Left    &  {\cd <=\tab>=\tab<\tab>} &  inequality \\
   9  & Left    &  {\cd ==\tab!=}      &       equality/inequality \\
   8  & Left    &  {\cd \&}          &         bitwise {\sc and} \\
   7  & Left    &  {\cd \caret}          &        bitwise {\sc xor} \\
   6  & Left    &  {\cd |}           &        bitwise {\sc or} \\
   5  & Left    &  {\cd \&\&}        &          logical {\sc and} \\
   4  & Left    &  {\cd ||}          &        logical {\sc or} \\
   3  & Right   &  {\cd ?:}          &        conditional \\
   2  & Right   & {\cd \begin{tabular}[t]{@{}llll@{}}
                    =  & +=  & -=  & *= \\
		  /= & \%= & \&= & \caret= \\
		  |= & <<= & >>=
		  \end{tabular}}      &  assignment \\  
   1  & Left    & {\cd ,}                 &  sequential \\
 & & & \\ \hline 
\end{tabular} 
\end{table} 
 The array operator {\cd []} and  the function operator {\cd ()} have
been discussed in Section~\ref{complxDecl}, while the operators {\cd
.}\ and {\cd ->} (which are used  for accessing  the components  of
structures and unions) will be described later.


\section{Expressions}
\index{expressions}
     An expression in C can be any one of the following:
\begin{itemize}
\item a numeric, character or string constant
\item an identifier
\item an expression in parentheses {\cd ()}
\item an array element
\item a function call
\item a  unary expression
\item a  binary expression
\item a  conditional expression
\item an  assignment expression
\item a  sequential expression
\item a  constant expression
\end{itemize}
     There are other types; these involve more complex data types that
we have not introduced yet.

     A {\kc unary expression\/} \index{expressions!unary}consists of a  unary operator and a
single operand. The unary operators are:
\begin{display}
\begin{tabular}{@{}lll@{}}
 type casts & {\cd sizeof} & {\cd ++} \\ 
 {\cd --} &  {\cd -} (unary minus) & {\cd !} (logical {\sc not}) \\ 
 {\cd \verb+~+} (bitwise {\sc not}) &  {\cd \&} (address of) 
   & {\cd *} (contents of)
\end{tabular}
\end{display}
\noindent
     A {\kc binary  expression\/} \index{expressions!binary}consists  of a  two operands
separated by a binary operator. The binary operators are:
\begin{display}
\begin{tabular}{@{}l@{\ }ll@{\ }ll@{\ }l@{}}
 {\cd +} & add       & {\cd -} & subtract    & &  \\
 {\cd *} & multiply  & {\cd /} & divide      &    {\cd \%} & remainder \\
 {\cd <<} & shift left& {\cd >>} & shift right & & \\
 {\cd !=} & not equal to & {\cd <} & less than & {\cd >} & greater than \\
 {\cd ==} & equal to & {\cd <=} & less or equal 
   & {\cd >=} & greater or equal \\
 {\cd \&} & bitwise {\sc and} & {\cd |} & bitwise {\cd or} 
   & {\cd \caret} & bitwise {\sc xor} \\
 {\cd \&\&} & logical {\cd and} & {\cd ||} & logical {\cd or}  & &
\end{tabular}
\end{display}
\noindent
     A {\kc conditional expression\/} \index{expressions!conditional}has the form:
\begin{display}\ms
 expression1 {\cd ?} expression2 {\cd :} expression3
\end{display}
\noindent
     The first  expression is  evaluated; if  it  is  true,  then the
second expression is  used as  the value of the conditional
expression. If the first expression is false, the third expression is
used instead.  We can  therefore read the conditional expression  
{\ms e1\/{\cd ?}e2{\cd :}e3} as:  ``If {\ms e1\/} then {\ms e2\/}
else {\ms e3\/}'' For example, we can declare macros to return the
greater and smaller of two numbers as:
 \begin{code}
 \#define  min(a,b)\tab(a>b ? b :\ a) \\
 \#define  max(a,b)\tab(a>b ? a :\ b)
\end{code}
\noindent
     We have  used {\kc assignment expressions\/} 
\index{expressions!assignment}already.  If you 
are familiar with some other  language, you  may wonder why we have
assignment expressions, when other languages  have assignment 
statements. The  answer is  that  in  C,  an assignment is  both an 
expression and  a statement:  an assignment expression returns the
value of the right-hand side of the expression as the value of the
whole expression.  If we  use the  assignment expression  as a
statement, this value is  simply discarded.  The use of assignments
as expressions will become more obvious later; two simple example will
suffice for the time being.  

Firstly, it allows us to perform multiple assignments, e.g.\
\begin{code}
a=b=c=-1;
\end{code}
\noindent
The value of the expression {\cd c=-1} (which also happens to assign
the value -1 to the variable {\cd c}) is -1, which is the value
assigned to {\cd b}, etc.

As a second example, consider the C run-time library 
function {\cd getchar()}, which reads and returns a character from
standard input. We could write the following:
\begin{code}
    char c; \\
    int size; \addVspace

	 printf("Do you have a 132 character printer? (y/n)\verb+\+n"); \\
    size = ((c=getchar())=='y') ? 132 :\ 80;
\end{code}
\noindent
     Here we  call {\cd getchar}, assign the result to {\cd c}, and
use the value assigned to  {\cd c}  immediately  within  the 
conditional  expression.  Without  assignment expressions, we would
have to use:
\begin{code}
         c = getchar(); \\
         size = (c=='y') ? 132 :\ 80;
\end{code}
\noindent
     in place of the last line.

     A {\kc sequential  expression\/} \index{expressions!sequential}is  a list  of expressions,
separated by commas.  Each expression  is evaluated in turn, and the
value of each is discarded, except for the  last expression,  whose
value  is returned  as the value of the whole sequential expression. 
The main  use of  sequential expressions is for loops; these will  be
explained later. Sequential expressions are also known as {\kc comma
expressions}.


     A {\kc constant  expression\/} \index{expressions!constant}is an expression that can be
evaluated at compile-time (that  is, before the program is actually
executed). These are needed for {\cd \#if} preprocessor  commands,
array  size declarations,  and in a  few other places (which have  yet
to  be dealt  with). Most  C operators are allowed for use in
constant expressions, the main proviso being that they can be
evaluated before execution. Constant  expressions in  {\cd \#if}
commands are more restricted, as they must be  able to  be evaluated
by the preprocessor, whereas all other constant expressions must  be
evaluated by the compiler (which is far more powerful and has
considerably  more information available to it. For example, the
addresses of variables are known by the compiler but not by the
preprocessor; thus if we have declared  {\cd static int  x;} we  can
use  {\cd \&x} as  a constant expression in the program but not in
preprocessor statements).

